% !TeX root = ../apuntes-ea.tex

\chapter{Aplicaciones prácticas}

\begin{ej}[Grupo de biyecciones $S_3$]
	Ver figura \ref{fig:s3elemento12}. Llamamos $S_3$ al grupo de las biyecciones $f:\{1,2,3\} \to \{1,2,3\}$. También podemos pensar en este grupo como el grupo de las permutaciones de 3 elementos. De hecho, utilizamos la siguiente notación para las biyecciones de $S_3$:
	\begin{itemize}
		\item $(1)$ indica que $f(1) = 1$. Por defecto, $f(2) = 2$ y $f(3) = 3$.
		\item $(12)$ indica que $f(1) = 2$ y $f(2) = 1$. Por defecto $f(3) = 3$.
		\item $(123)$ indica que $f(1) = 2,\ f(2) = 3,\ f(3) = 1$.
		\item $(13)$ indica que $f(1) = 3,\ f(3) = 1$ y por defecto $f(2) = 2$.
	\end{itemize}
	
	\begin{figure}[h]
		\centering
		\begin{tikzpicture}[scale=0.6]
		\node (1) at (0,1) {$1$};
		\node (2) at (0,0) {$2$};
		\node (3) at (0,-1) {$3$};
		
		\node (f1) at (3,1) {$1$};
		\node (f2) at (3,0) {$2$};
		\node (f3) at (3,-1) {$3$};
		
		\draw (0,0) ellipse (.7 and 2);
		\draw (3,0) ellipse (.7 and 2);
		
		\draw[-{Latex[length=2mm]}] (1) -- (f2);
		\draw[-{Latex[length=2mm]}] (2) -- (f1);
		\draw[-{Latex[length=2mm]}] (3) -- (f3);
		\end{tikzpicture}
		\caption{Elemento (12) de $S_3$}
		\label{fig:s3elemento12}
	\end{figure}
	
	En este grupo ocurre algo parecido a lo que ocurre en $D_4$. Sea $a = (123), b = (12)$. Podemos presentar el grupo con
	\begin{align}
	S_3 = \langle a, b\rangle\text{ donde } o(a) = 3,\ o(b) = 2,\ ba = ab^2
	\end{align}
	y por tanto $S_3 = \{1, a, a^2, b, ab, a^2b\} = \{(1), (12), (13), (23), (123), (132)\}$. %TODO comprobar
\end{ej}

