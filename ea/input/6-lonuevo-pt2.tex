% !TeX root = ../apuntes-ea.tex

\chapter{Lo nuevo - Parte 2}

Apuntes de Santorum desde la definición de grupo simple.
Las dos definiciones siguientes no están explicitas en los apuntes de Santorum y puede que no sean de mi parte pero las necesito para la legibilidad de la misma.
\begin{dfn}[Grupo de biyecciones]
	\label{dfn:grupobiy}
	Sea $X$ un conjunto, definimos $Biy(X)=\left\{f\mid f : X \longrightarrow X  \right\}$ como el conjunto de biyecciones de $X$ en $X$. Si $|X| \not \eq \infty$, entonces $Biy(X) = S_n$ siendo $S_n$ el conjunto de simetrías o el conjunto de permutaciones de n elementos. $(Biy(X), \circ)$ es un grupo.
\end{dfn}

\begin{dfn}[Grupo alternante]
	\label{dfn:grupopermpar}
	Sea $(S_n,\circ)$ el grupo de permutaciones de $n$ elementos. Llamamos grupo alternante $A_n \subseteq S_n$ al subgrupo de $S_n$ formado por las permutaciones que resultan de componer un número par de transposiciones.
\end{dfn}

\begin{dfn}[Grupo simple]
	Sea $G$ un grupo, decimos que $G$ es un grupo simple si los únicos grupos normales son $G$ y el grupo neutro $\{e\}.$
\end{dfn}

A continuación demostraremos que el grupo alternante $A_n$, es simple para $n\geq 5$. La demostración de este resultado requiere distintas proposiciones.
\begin{pro}
	Sea $G$ un grupo. Si $G$ es finito y abeliano $\implies\ G$ es simple.
\end{pro}
%TODO: demostracion de la proposición.
\begin{pro}
	Sea $A_n$ un grupo alternante, $A_n$ es generado por 3-ciclos para $n\geq 3$.
\end{pro}
\begin{proof}
	Sea $\sigma \in A_n$, entonces $\sigma = (i_1^1\ i_2^1)(i_1^2\ i_2^2)\ldots(i_1^{2n}\ i_2^{2n})$ una composición de un número par de composiciones. Vamos a ver que para cualquier par de transposiciones $(i\ j)(k\ l)$ podemos expresarla como un $3-ciclo$.
	\begin{align*}
		(i\ j)(k\ l) &= (i\ k\ j)(i\ k\ l)\ &si\ los\ elementos\ son\ diferentes.\\
		(i\ j)(i\ l) &= (i\ l\ j)\ &si\ tienen\ un\ elemento\ en\ comun.
	\end{align*}
	Por tanto, como $\forall \sigma \in A_n$ puede ser expresado como un $3-ciclo$ o una composición de estos, $A_n$ está generado por los ciclos de longitud 3.
\end{proof}
\begin{pro}
	\label{pro:alternante3ciclos}
	Sea $A_n$ el grupo alternante de un conjunto de $n$ elementos, $A_n$ es generado por 3-ciclos de la forma $(s\ t\ i)$ con $s,t\in \{1\ldots n\}$ fijos e $i\in\{1\ldots n\}\setminus\{s,t\}$
\end{pro}
\begin{proof}
	Cada 3-ciclo es el producto de 3-ciclos del tipo $(s\ t\ i)$ con $s,t$ fijos e $i$ variable, pues:
	\begin{align*}
		(s\ a\ t) &= (s\ t\ a)^2\\
		(s\ a\ b) &= (s\ t\ b)(s\ t\ a)^2\\
		(t\ a\ b) &= (s\ t\ b)^2(s\ t\ a)\\
		(a\ b\ c) &= (s\ t\ a)^2(s\ t\ c)(s\ t\ b)^2(s\ t\ a)
	\end{align*}
	Entonces, como $A_n$ está generado por 3-ciclos, $A_n$ está generado por ciclos de la forma $(s\ t\ i)$
\end{proof}
\begin{thm}[Igualdad entre subgrupos y grupos alternantes]
	\label{thm:subequalsalternate}
	Si un subgrupo normal $H$ de $A_n$ contiene un 3-ciclo $\implies H = A_n$
\end{thm}
\begin{proof}
	Supongamos que $H$ es no trivial y contiene un 3-ciclo de la forma $(s\ t\ a)$. Usando la normalidad de $H$ vemos que:
	\[
		[(s\ t)(a\ i)](s\ t\ a)^2[(s\ t)(a\ k)]^{-1} = (s\ t\ i)
	\]
	está en $H$. Luego, $H$ debe contener todos los ciclos $(s\ t\ i)$ para $1 \geq i \geq n$. Por la proposición \ref{pro:alternante3ciclos}, estos 3-ciclos generan $A_n$; luego $H = A_n$.
\end{proof}
\begin{pro}
	\label{pro:3cicloinsubgralter}
	Para $n\geq 5$, todo $H \normsub A_n$ contiene un 3-ciclo.
\end{pro}
\begin{proof}
	Sea $e \not \eq \sigma \in H$, existen varias posibles estructuras de ciclos para $\sigma$.
	\begin{itemize}
		\item $\sigma$ es un 3-ciclo.
		\item $\sigma$ es el producto de ciclos disjuntos, $\sigma = \tau(a_1\ a_2 \cdots a_r)\in H$, con $r\geq 3$.
		\item $\sigma$ es el producto de ciclos disjuntos, $\sigma = \tau(a_1\ a_2\ a_3)(a_4\ a_5\ a_6)$.
		\item $\sigma = \tau(a_1\ a_2\ a_3)$, donde $\tau$ es el producto de 2-ciclos disjuntos.
		\item $\sigma = \tau(a_1\ a_2)(a_3\ a_4)$, donde $\tau$ es el producto de un número par de 2-ciclos disjuntos.
	\end{itemize}
	La demostración sigue con el desarrollo de cada uno de los casos, utilizando la normalidad de $H$ para ver que en todos los casos se llega a que $H$ contiene un 3-ciclo.
\end{proof}
\begin{thm}[Simplicidad del grupo alternante]
	Sea $(A_n, \circ)$ el grupo alternante de un conjunto de $n$ elementos. $A_n$ es simple $\forall n \geq 5$.
\end{thm}
\begin{proof}
	Sea $H$ un subgrupo normal no trivial de $A_n$, por la proposición \ref{pro:3cicloinsubgralter}, $H$ contiene un 3-ciclo. Por el teorema \ref{thm:subequalsalternate}, $H = A_n$; por tanto, $A_n$ no contienen ningún subgrupo normal que sea propio y no trivial para $n\geq 5$.
\end{proof}
Falta la semana fatídica de Estadística

% 20181029
Vez pasada considerabamos $G_1 \times G_2$ y fijado un homomorfismo de grupos $\phi: G_1 \to Aut(G_2)$ hacíamos lo siguiente. En $G_1 \times_{\phi} G_2$ viven los elementos $(a,b) \times_{\phi} (c,d)$ donde la operación cambiaba en la primera coordenada $(a \phi_b(c), bd)$. Probamos la última clase que $G_1 \times_{\phi} G_2$ era un grupo (probar la asociatividad no es trivial).

% 20181030

Observación:

\begin{align*}
\gamma: G \xrightarrow{Int} Aut(G)
\end{align*}
$\gamma$ es un homomorfismo de grupos que lleva cada elemento $g \in G$ al automorfismo conjugación $\gamma_g(x) = gx\inv{g}$. Observamos que si $N \normsub G,\ \forall g \in G, \gamma_g(N) = gN\inv{g} = N$.

\begin{pro}
	$N$ es normal en $G$ ($N \normsub G$) sí y solo sí al restringir $\phi_g$ a $N$ la imagen es $N$:
	\begin{align*}
	G \xrightarrow{\gamma_g} G \\
	N \xrightarrow{\gamma_g \vert_N} N
	\end{align*}
	Es decir, que si $N$ es normal, $\gamma_g\vert_N$ induce un isomorfismo $\gamma_g\vert_N : N \to N$.
\end{pro}

\begin{proof}
	Cristalina de la definición de subgrupo normal.
\end{proof}

En general, al restringir $\gamma_g$ a un subgrupo de $G$ no tenemos esta propiedad.

Además, si $N \normsub G$ tiene sentido restringir $\gamma: G \xrightarrow{Int} Aut(G)$ a $Aut(N)$ y la restricción da un homomorfismo.

\section{Nuevas estructuras de grupo en el producto directo}

Sean $G_1, G_2$ grupos, queremos definir nuevas estructuras de grupo en el producto $G_1 \times G_2$.
Para ello comenzaremos definiendo una operación $\ast_\alpha$. Fijamos un homomorfismo de grupos $\alpha:G_2 \longrightarrow Aut(G_1)$, con $Aut(G_1)$ el grupo de automorfismos de $G_1$.\\\\
Sean $(a,b),(c,d) \in G_1\times G_2$, definimos $\ast_\alpha$ como:
\[
	(a,b)\ast_\alpha(c,d) = (a\cdot\alpha(b)\cdot c, b\cdot d).
\]
Donde $b\in G_2,\ \alpha(b) \in G_1$ y $\alpha(b)\cdot c \in G_1$.\\\\
Vamos a ver que $(G_1 \times G_2, \ast_\alpha)$ es un grupo.
\begin{thm}[Grupo producto directo]
	$(G_1 \times G_2, \ast_\alpha)$ es un grupo.
\end{thm}
Vamos a demostrar cada una de las propiedades del grupo:
\begin{itemize}
	\item Asociatividad.
		\begin{proof}
			\begin{align*}
				(a\cdot\alpha(b)\cdot c, bd) \ast_\alpha (h,f) &= (a\cdot\alpha(b)\cdot c\cdot \alpha(bd)\cdot h, b\cdot d\cdot h)\\
				(a,b)\ast_\alpha(c\cdot\alpha(d)\cdot h, df) &= (a\cdot\alpha(b)\cdot c\cdot \alpha(d)\cdot h, b\cdot d\cdot h)
			\end{align*}
			Entonces, falta ver que $\alpha(d)\cdot h = \alpha(bd)\cdot h$. Definimos el isomorfismo de grupo:
			\begin{align*}
				\alpha(b) : G_1 &\longrightarrow G_1\\
				c &\longmapsto \alpha(b)\cdot c\\
				\alpha(d)\cdot h &\longmapsto \alpha(b)\cdot(\alpha(d)\cdot h) = \alpha(bd) \cdot h.
			\end{align*}
			Por tanto, son iguales y la operación es asociativa.
		\end{proof}	
	\item Existencia del elemento neutro.
		\begin{proof}
			Sean $e_1$ y $e_2$ elementos neutros de $G_1$ y $G_2$ respectivamente. Recordamos que por el argumento anterior $\alpha(b)\cdot e_1 = e_1$.
			\begin{align*}
				(a,b) \ast_\alpha (e_1, e_2) = (a \cdot \alpha(b) \cdot e_1, b \cdot e_2) = (a,b)
			\end{align*}
		\end{proof}
	\item Existencia del inverso.
		\begin{proof}
			Hemos de hallar $(c,d) \mid (a,b)\ast_\alpha(c,d) = (e_1,e_2)$.  Entonces, hemos de hallar $c$ y $d$ tal que:
			\begin{align*}
				a \cdot \alpha(b) \cdot c &= e_1\\
				b \cdot d &= e_2
			\end{align*}
			Es fácil ver que $\exists d$ y $d = b^{-1}$. Como $\alpha(b)$ es un isomorfismo $\implies \exists (\alpha(b))^{-1}$, entonces, $c = \alpha(b^{-1}) \cdot a^{-1} = a^{-1}$, por tanto $\exists c$ y $c = a^{-1}$.
		\end{proof}
		
\end{itemize}
Por tanto, el par $(G_1 \times G_2, \ast_\alpha)$ tiene estructura de grupo.\\

Vamos a ver ahora ciertas relaciones del producto cruz con la operación que acabamos de definir. Para abreviar, al par $(G_1 \times G_2, \ast_\alpha)$ lo denominaremos por $G_1 \times_\alpha G_2$.\\\\
Sean $G_1, G_2$ grupos finitos, definimos:
\begin{align*}
	G_1^\ast &= \{(a, e_2) \mid a \in G_1\}\\
	G_2^\ast &= \{(e_1, b) \mid a \in G_2\}
\end{align*}
Es fácil ver que $G_1^\ast < G_1 \times_\alpha G_2$ y $G_2^\ast < G_1 \times_\alpha G_2$. Además,
\begin{align*}
	|G_1^\ast\cdot G_2^\ast| &= \frac{|G_1^\ast|\cdot |G_2^\ast|}{|G_1^\ast \cap G_2^\ast|} = \frac{|G_1^\ast|\cdot |G_2^\ast|}{1} = |G_1|\cdot |G_2| = |G_1 \times_\alpha G_2|\\
	G_1^\ast \cap G_2^\ast &= {(e_1, e_2)}
\end{align*}
Y podemos probar que $G_1^\ast$ es normal, sean $g_1 \in G_1$ y $g_2 \in G_2$:
\begin{align*}
	(g_1, g_2) \ast_\alpha (a, e_2) \ast_\alpha (g_1, g_2)^{-1} = (g_1,g_2)\ast_\alpha(\ldots, e_2\cdot g_2^{-1}) = (\ldots, e_2).
\end{align*}
\begin{cor}
	\label{cor:propiedadesgrupdirecto}
	Por lo que acabamos de ver:
	\begin{itemize}
		\item $\hat{G_1}$ y $\hat{G_2}$ son subgrupos.
		\item $\hat{G_1}$ es normal.
		\item $G_1^\ast \cap G_2^\ast = \{(e_1,e_2)\}$
		\item $G_1^\ast \cdot G_2^\ast = G_1 \times_\alpha G_2$
	\end{itemize}
	Si ahora tomamos $G_1 = N, G_2 = H$ con $N \normsub G, H < G$, entonces:
	\begin{itemize}
		\item $H \cap N = \{e\}$
		\item $H \cdot N = G$
		\item $\alpha: H \longrightarrow Aut(N)$
		\item $G \cong H \times_\alpha N$
	\end{itemize}
\end{cor}
En particular, podemos definir:
\begin{align*}
	\phi : H &\longrightarrow Aut(N)\\
	h &\longmapsto \gamma_h\mid_N(n) = h\cdot n\cdot h^{-1}
\end{align*}
\begin{ej}
	Sea el famoso grupo $D_4 = \{1,B,B^2,B^3,A,AB,AB^2,AB^3\}$ (ver ejemplo \ref{ej:famosogrupod4}). Tomamos $N = \langle B \rangle =\{1,B,B^2,B^3\},\ H = \langle A \rangle =\{1,A\}$. Entonces:
	\begin{align*}
		\phi: H &\longrightarrow \autom{N}\\
		A &\longmapsto ABA^{-1} = B^3
	\end{align*}
Entonces como hemos visto: $D_4 \cong \{1,A\} \ast_\phi \{1,B,B^2,B^3\}$.
\end{ej}
\section{Clase de equivalencia por el grupo de biyecciones}
\begin{dfn}[Clase de equivalencia por el grupo de biyecciones]
Sea $(G, \ast)$ un grupo, $X$ un conjunto, $Biy(X) = \{f\mid f: X\longrightarrow X\ biyeccion\}$, y $\alpha: G \longrightarrow Biy(X)$ es un homomorfismo de grupos. Definimos la siguiente relación de equivalencia:
\begin{align*}
	a\mathcal{R}b\ si\ \exists g \mid \alpha(g)a=b
\end{align*}\\
Por esta relación, definimos la \textbf{clase} de equivalencia de un elemento $a \in X$ como:  \[cl(a)=\{ \alpha(g)(a)\mid g\in G \} \ni a \]
\end{dfn}
Para poder definirlo mejor nos gustaría saber cuantos elementos existen en $cl(a)$. Para ello nos ayudaremos del $centralizador\ de\ a$ (definición \ref{dfn:centralizador}). En nuestro caso particular, el centralizador es:
\[ C_G(a) \{g\in G \mid \alpha(g)(a) = a \} \]
Es fácil ver que si $g \in C_G(a)$ y $g' \in C_G(a)$ entonces $g \ast g' = C_G(a)$.
\begin{thm}[Orden de la clase de equivalencia de un elemento]
Sea $(G, \ast)$ un grupo, $C(a)$ el centralizador de $a$ y $cl(a)$ la clase de equivalencia de $a$:
\[ |cl(a)| = \left[ G:C(a) \right] \]
\end{thm}
%TODO: Demostración
\begin{itemize}
	\item Recordemos que fijado $\sigma \in S_5$ podemos dar una descomposición en ciclos $\sigma = (123)(45)$ que es única aunque los ciclos se escriban diferente (por ejemplo $(123) = (231)$).
	
	\item Fijado $\tau \in S_5$, $\tau \sigma \inv{\tau} = (\tau(1)\tau(2)\tau(3))(\tau(4)\tau(5))$ la descomposición se mantiene
	
	\item Si dos permutaciones $\sigma, \sigma'$ tienen descomposiciones del mismo tipo (un 3-ciclo y un 2-ciclo como antes) entonces existe un $\tau$ que hace pasar de una a otra.
\end{itemize}

\begin{ej}[Posibles descomposiciones en cíclos de $S_4$]$ $ \newline
	\begin{itemize}
		\item Para $(1234)$
		\begin{align*}
		cl((1234)) = \{\tau(1234)\inv{\tau} \mid \tau \in S_4\}
		\end{align*}
		\item A la hora de definir $\tau$ tenemos varias posibilidades. En este caso, si empezamos por el $1$, para fijar el segundo elemento solo tenemos 3 posibilidades, para el tercero 2 y para el último una. Por tanto
		\begin{align*}
		|cl((1234))| = 4
		\end{align*}
		
		\item Recordemos que el centralizador
		\begin{align*}
		C_{S_4}((1234)) = \{\sigma \in S_4 \mid \sigma (1234) \inv{\sigma} = (1234)\} < S_4
		\end{align*}
		
		\item Como $S_4$ tiene $|S_4| = 4! = 24$ y tenemos que $|cl((1234))| = [S_4 : C_{S_4}((1234))] = 6$ necesariamente $|C_{S_4}((1234))| = 4$.
		
		\item Nos proponemos calcular el grupo $C((1234))$. Un candidato para $\sigma \in C((1234))$ es $\sigma = (1234)$. En efecto $(1234)(1234)(1234) \in C((1234))$. Siempre ocurre que un elemento conmuta consigo mismo. Además, $\langle (1234) \rangle < C((1234))$ pero como $|\langle (1234) \rangle| = 4 = |C((1234))$ tiene que ocurrir que $\langle (1234) \rangle = C((1234))$. Es decir que de tipo 4 solo tenemos $(1234)$.
		
		\item ¿Qué tipos tenemos? Pues tantos como maneras de descomponer 4 en suma de números positivos, a saber
		\begin{itemize}
			\item (1234) de tipo 4
			\item (123) de tipo 3+1
			\item (12)(34) de tipo 2+2
			\item (12) de tipo 2+1+1
			\item $Id$ de tipo 1+1+1+1 (que es la única que tiene descomposición en 4 unos)
		\end{itemize}
		
		\item En general no es difícil calcular cuantos hay, por lo que a menudo utilizamos este argumento para calcular el grupo centralizador.
		
		\item Lo importante es que estamos descomponiendo $S_4$ de la siguiente manera:
		\begin{align*}
		S_4 &= cl((1234)) \cap cl((1223)) \cap cl((12)(34)) \cap cl((12)) \cap cl(Id) \\
		|S_4| &= |cl((1234))| \cap |cl((1223))| \cap |cl((12)(34))| \cap |cl((12))| \cap |cl(Id)|
		\end{align*}
		\item Ahora analizamos la clase $cl((123))$ de los ciclos de tipo 3+1. Lo primero es saber cuantos hay. Pues tenemos que elegir 3 elementos de entre 4 y luego ordenar los dos que nos quedan por tanto
		\begin{align*}
		|cl((123))| = \binom{4}{3} \times 2 = 8
		\end{align*}
		Por otro lado lo que sabemos es que $(123) \in C((123))$ (porque todos conmutan consigo mismos) y como antes $|C((123))| = 3$ (de la fórmula $|cl((123))| = [S_4:C((123))]$), luego $C((123)) = \langle (123) \rangle$.
		
		\item Igual es un poco más interesante la clase de tipo 2+2. \textbf{Pregunta de examen:} halla generadores del subgrupo centralizador del elemento (12)(34).
		\begin{itemize}
			\item Sabemos que el conjugado de un elemento de tipo 2 tiene que ser otro de tipo 2, por tanto tenemos que ver qué elementos distintos de tipo 2 tenemos. Pues fijamos el 1 por ejemplo y vemos qué parejas podemos hacer. Nos salen 3, a saber, 1 con 2, 1 con 3 y 1 con 4 de lo que concluímos que $|cl((12)(34))| = 3$.
			\item De la misma fórmula que antes sacamos que $|C((12)(34))| = 8$. De orden 8 sabemos que hay solo unos pocos grupos (ver la clasificación en \ref{gruposfinitosnotables}). Veamos con cuál de ellos es isomorfo.
			\item Como siempre sabemos que $(12)(34) \in C((12)(34))$. Tenemos que encontrar los demás $\tau$ que conmutan $\tau \sigma \inv{\tau} = \tau (12)(34) \inv{\tau} = (\tau(1)\tau(2))(\tau(3)\tau(4))$. Probamos con $\tau = (1324)$\footnote{La idea de probar con este viene de decir: pues a ver qué pasa si cambio el 1 con el 3 y el 2 con el 4, que nos daría la permutación (1324). En cualquier caso esto es prueba y error, y parar de probar cuando tengamos un grupo generado de orden 8.}.
			\begin{align*}
			(1324)&(12)(34)\inv{(1324)} \\
			&(34)(21)
			\end{align*}
			Que es el mismo, luego hemos probado que $\tau$ conmuta y por tanto $\tau \in C((12)(34))$. Lástima que no valga porque nos damos cuenta de que $\tau ^2 = (12)(34)$. Vaya. Drácula ha hecho chiste con esto y todo $(X,d)$.\footnote{Aquí se ve claramente que la elección del $\tau$ es casi al azar. Hemos elegido uno que prometía pero hemos tenido la mala suerte de que su cuadrado nos daba un elemento que suponíamos estaba en el grupo ($\tau^2 = (12)(34)$. Podríamos haber descartado este $\tau = (1324)$ pero hemos preferido descartar el elemento (12)(34) que sabíamos que estaba en el grupo. La razón de la sustitución de este último por el (12) es un misterio hasta la fecha.}
			
			Lo que hacemos es quitar el $(12)(34)$ y cambiarlo por el $(12)$. Para evitar $\tau^2 \neq (12)$. En resumen, ya tenemos $(12) \in C((12)(34))$ y $\tau = (1324) \in C((12)(34))$. Si vemos sus grupos generados:
			\begin{align*}
			\langle (1324)\rangle = \{(1324), (12)(23), (4321), Id\} \\
			\langle (12) \rangle = \{(12), Id\}
			\end{align*}
			La intersección de ambos subgrupos es solo la identidad y por la fórmula del producto libre averiguamos que $|\langle (1324)\rangle \langle (12) \rangle| = 8$ por lo $C((12)(34)) = \langle (1324), (12) \rangle$.
			
			Tiene toda la pinta de ser $D_4$ porque está generado por dos elementos, no es abeliano y los órdenes de los generadores son $o((1324)) = 4,\ o((12)) = 2$. Solo nos quedaría probar que se sigue cumpliendo la ecuación de la presentación de $D_4$:
			\begin{align*}
			BA = AB^3 \iff (1324)(12) = (12)(1324)^3
			\end{align*}
			Lo comprobamos y al final sale.
		\end{itemize}
		
		\item Ahora hacemos lo mismo con $C((12))$. Siguiendo un razonamiento similar, llegamos a que $C((12))$ es isomorfo con el grupo de Klein y por extensión con $\Z/2\Z \times \Z/2\Z$.
	\end{itemize}
\end{ej}


Falta la semana fatídica de Estadística

% 20181029
Vez pasada considerabamos $G_1 \times G_2$ y fijado un homomorfismo de grupos $\phi: G_1 \to Aut(G_2)$ hacíamos lo siguiente. En $G_1 \times_{\phi} G_2$ viven los elementos $(a,b) \times_{\phi} (c,d)$ donde la operación cambiaba en la primera coordenada $(a \phi_b(c), bd)$. Probamos la última clase que $G_1 \times_{\phi} G_2$ era un grupo (probar la asociatividad no es trivial).

% 20181030

Observación:

\begin{align*}
\gamma: G \xrightarrow{Int} Aut(G)
\end{align*}
$\gamma$ es un homomorfismo de grupos que lleva cada elemento $g \in G$ al automorfismo conjugación $\gamma_g(x) = gx\inv{g}$. Observamos que si $N \normsub G,\ \forall g \in G, \gamma_g(N) = gN\inv{g} = N$.

\begin{pro}
	$N$ es normal en $G$ ($N \normsub G$) sí y solo sí al restringir $\phi_g$ a $N$ la imagen es $N$:
	\begin{align*}
	G \xrightarrow{\gamma_g} G \\
	N \xrightarrow{\gamma_g \vert_N} N
	\end{align*}
	Es decir, que si $N$ es normal, $\gamma_g\vert_N$ induce un isomorfismo $\gamma_g\vert_N : N \to N$.
\end{pro}

\begin{proof}
	Cristalina de la definición de subgrupo normal.
\end{proof}

En general, al restringir $\gamma_g$ a un subgrupo de $G$ no tenemos esta propiedad.

Además, si $N \normsub G$ tiene sentido restringir $\gamma: G \xrightarrow{Int} Aut(G)$ a $Aut(N)$ y la restricción da un homomorfismo.

\section{Producto semidirecto}

Sea $G$ un grupo. Sea $N \normsub G$, $H < G$, $N \cap H = \{e\}$ y $NH = G$ (recordemos que por ser $N$ normal, $NH$ es grupo). Entonces $G \isom N \times H$.

Veamos quién es ese isomorfismo $\gamma : G \to N \times H$. Recordemos que considerando dos grupos $G_1, G_2$ y su producto directo $G_1 \times G_2$ existe un $\alpha : G_2 \to Aut(G_1)$. Veremos quien es este $\alpha$ para $H$ y $N$, es decir, quién es $\alpha: H \to Aut(N)$.

Construye $\alpha$ a partir de 4 isomorfismos.

\begin{proof}$ $\newline
	\begin{itemize}
		\item Comenzamos por definir una función $j: N\times H \to G,\ (n, h) \mapsto nh$. Es función está bien definida por teoría de conjuntos pero no es un homomorfismo de grupos\footnote{Ojo con por qué no es homomorfismo. Si tomamos $(n,h),(n', h') \in N \times H$ tenemos que $j((n,h)(n',h')) = nn'hh'$. Podríamos pensar que como $N$ es normal, podemos conmutarlo y obtener $nn'hh' = nhn'h' = j((n,h))j((n',h'))$. \textbf{Pero esto está mal.} Lo que significa ser normal es que para $h \in H$, se tiene que $nh = hn''$ para algún $n'' \in N$.}\footnote{Si los grupos son abelianos entonces sí es claro que es un homomorfismo. Lo que vamos a hacer es ver que dando una estructura especial, sí que es un homomorfismo de grupos incluso para grupos no abelianos}.
		\item Recordemos que por el teorema \ref{thm:cardinalidadproductolibre} tenemos que $|G| = |N||H| = |N \times H|$ por ser $N \cap H = \{e\}$.
		\item Volviendo a lo de la estructura especial. Dar una estructura especial es dar una operación para $N \times H$.
		\begin{itemize}
			\item Sea $A$ un conjunto. Es claro que si tenemos una biyección $\phi : A \to G$ podemos dotar a $A$ de alguna estructura para que sea un grupo.
			\item Para dotar a $A$ de estructura tenemos que definir la operación. Forzamos que para cada $a, a' \in A$ para los que se tiene $\phi(g) = a, \phi(g') = a'$ la operación sea $a a'  = \phi(gg')$.
			\item En este caso nuestro $A$ es $N \times H$. En lugar de utilizar la operación habitual del producto directo definimos otra operación. Para llegar a ella nos fijamos en $(n,h)(n',h') \mapsto nhn'h' = nhn'\inv{h}hh' = n(hn'\inv{h})hh' = nn'hh'$ (intercalamos el neutro, que es legal).
			\item Redefinimos la operación en $N \times H$ para que cuadre con este resultado. Llamaremos al nuevo grupo con la nueva operación $N \times_\phi H$: para $(n,h), (n',h')$ definimos $(n,h)\cdot (n',h') = (n(hn'\inv{h}), hh')$.
			\item Comprobamos que en este caso $j$ es un homomorfismo de grupos:
			\begin{align*}
			j : N \times_\phi H &\to G \\
			(n,h) &\mapsto nh \\
			(n',h') &\mapsto n'h' \\
			(n,h)\cdot(n',h') &\mapsto n(hn'\inv{h})hh' = nn'hh'
			\end{align*}
		\end{itemize} 
	\end{itemize}
\end{proof}

Es muy interesante por que es muy natural llegar a situaciones de esta manera. ¡Y les voy a dar una!\footnote{Sugerencia: leelo con voz de tomatito.}

\begin{ej}
	Sea $|G| = p \cdot q$ y supongamos $p < q$ primos. Por el teorema de Lagrange (\ref{thm:lagrange}) tenemos que existe un subgrupo $H_p < G$ con $|H_p| = p$ y análogamente $\exists H_q \mid |H_q| = q$. A primera vista podríamos pensar que puede haber varios grupos de orden $q$. Pues no.
\end{ej}

\begin{proof}
	Supongamos hay dos grupos $H, H'$ de orden $q$ distintos. La intersección tiene que dar un subgrupo y si los dos grupos tienen un número primo de elementos entonces la intersección solo puede ser el neutro, $H \cap H' = \{e\}$. Entonces por el teorema \ref{thm:cardinalidadproductolibre} tenemos que $|HH'| = q^2 > p\cdot q$ lo que es imposible. Luego sabemos que a lo sumo hay un grupo de orden $q$.
\end{proof}

(Sigue el ejemplo) Supongamos que ese único grupo de orden $q$ se llama $N$. Entonces $\phi_g(N) = N$ ya que un isomorfismo tiene que mandar un subgrupo de $q$ elementos en otro subgrupo de $q$ elementos y $N$ es el único. Por tanto $N \normsub G$. Aplicando el teorema de antes, tenemos que $G \isom N \times H$.

\begin{ej}
	Veamos un ejemplo con más pinta de problema. Demostrar que todo grupo de orden $77$ es cíclico.
\end{ej}

\begin{proof}
	Comenzamos por observar que $77 = 7 \cdot 11$. Por el teorema de Lagrange (\ref{thm:lagrange}) tenemos que existen $H, N < G \mid |H| = 7,\ |N| = 11$ y por lo visto en el ejemplo anterior, $N \normsub H$. Como antes llegamos a que $H \cap N = \{e\}$ y a que $|H N| = pq$. Para aplicar el teorema anterior vemos qué estructura tiene que tener $N \times_\phi H$, con $\phi:H\longrightarrow Aut(N)$.
	\\\\
	Vemos que $Aut(N) = Aut(\mathbb{Z}/11\mathbb{Z}) = \mathcal{U}(\mathbb{Z}/11\mathbb{Z}) = \mathbb{Z}/10\mathbb{Z}$, es decir, un grupo cíclico de 10 elementos.
	\\\\
	Entonces, $\phi$ es de la forma: $H = \mathbb{Z}/7\mathbb{Z} \longrightarrow Aut(\mathbb{Z}/11\mathbb{Z})$, por tanto, solo podemos definir el homomorfismo de grupos trivial. Esto hace que $N \times_\phi H$ es igual a $\mathbb{Z}/7\mathbb{Z} \times \mathbb{Z}/11\mathbb{Z}$.\\\\Por el corolario \ref{cor:propiedadesgrupdirecto} sabemos que $G \cong N \times_\phi H \implies G\cong \mathbb{Z}/7\mathbb{Z} \times \mathbb{Z}/11\mathbb{Z}$ que es cíclico por ser producto de cíclicos de órdenes coprimos.
\end{proof}