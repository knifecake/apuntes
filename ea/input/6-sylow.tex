% !TeX root = ../apuntes-ea.tex

\chapter{Teoremas de Sylow}

\section{Nuevas estructuras de grupo en el producto directo}

Sean $G_1, G_2$ grupos, queremos definir nuevas estructuras de grupo en el producto $G_1 \times G_2$.
Para ello comenzaremos definiendo una operación $\ast_\alpha$. Fijamos un homomorfismo de grupos $\alpha:G_2 \longrightarrow Aut(G_1)$, con $Aut(G_1)$ el grupo de automorfismos de $G_1$.\\\\
Sean $(a,b),(c,d) \in G_1\times G_2$, definimos $\ast_\alpha$ como:
\[
(a,b)\ast_\alpha(c,d) = (a\cdot\alpha(b)\cdot c, b\cdot d).
\]
Donde $b\in G_2,\ \alpha(b) \in G_1$ y $\alpha(b)\cdot c \in G_1$.\\\\
Vamos a ver que $(G_1 \times G_2, \ast_\alpha)$ es un grupo.
\begin{thm}[Grupo producto semidirecto]
	$(G_1 \times G_2, \ast_\alpha)$ es un grupo.
\end{thm}
Vamos a demostrar cada una de las propiedades del grupo:
\begin{itemize}
	\item Asociatividad.
	\begin{proof}
		\begin{align*}
		(a\cdot\alpha(b)\cdot c, bd) \ast_\alpha (h,f) &= (a\cdot\alpha(b)\cdot c\cdot \alpha(bd)\cdot h, b\cdot d\cdot h)\\
		(a,b)\ast_\alpha(c\cdot\alpha(d)\cdot h, df) &= (a\cdot\alpha(b)\cdot c\cdot \alpha(d)\cdot h, b\cdot d\cdot h)
		\end{align*}
		Entonces, falta ver que $\alpha(d)\cdot h = \alpha(bd)\cdot h$. Definimos el isomorfismo de grupo:
		\begin{align*}
		\alpha(b) : G_1 &\longrightarrow G_1\\
		c &\longmapsto \alpha(b)\cdot c\\
		\alpha(d)\cdot h &\longmapsto \alpha(b)\cdot(\alpha(d)\cdot h) = \alpha(bd) \cdot h.
		\end{align*}
		Por tanto, son iguales y la operación es asociativa.
	\end{proof}	
	\item Existencia del elemento neutro.
	\begin{proof}
		Sean $e_1$ y $e_2$ elementos neutros de $G_1$ y $G_2$ respectivamente. Recordamos que por el argumento anterior $\alpha(b)\cdot e_1 = e_1$.
		\begin{align*}
		(a,b) \ast_\alpha (e_1, e_2) = (a \cdot \alpha(b) \cdot e_1, b \cdot e_2) = (a,b)
		\end{align*}
	\end{proof}
	\item Existencia del inverso.
	\begin{proof}
		Hemos de hallar $(c,d) \mid (a,b)\ast_\alpha(c,d) = (e_1,e_2)$.  Entonces, hemos de hallar $c$ y $d$ tal que:
		\begin{align*}
		a \cdot \alpha(b) \cdot c &= e_1\\
		b \cdot d &= e_2
		\end{align*}
		Es fácil ver que $\exists d$ y $d = b^{-1}$. Como $\alpha(b)$ es un isomorfismo $\implies \exists (\alpha(b))^{-1}$, entonces, $c = \alpha(b^{-1}) \cdot a^{-1} = a^{-1}$, por tanto $\exists c$ y $c = a^{-1}$.
	\end{proof}
	
\end{itemize}
Por tanto, el par $(G_1 \times G_2, \ast_\alpha)$ tiene estructura de grupo.\\

Vamos a ver ahora ciertas relaciones del producto cruz con la operación que acabamos de definir. Para abreviar, al par $(G_1 \times G_2, \ast_\alpha)$ lo denominaremos por $G_1 \times_\alpha G_2$.\\\\
Sean $G_1, G_2$ grupos finitos, definimos:
\begin{align*}
\underline{G_1} &= \{(a, e_2) \mid a \in G_1\}\\
\underline{G_2} &= \{(e_1, b) \mid a \in G_2\}
\end{align*}
Es fácil ver que $\underline{G_1} < G_1 \times_\alpha G_2$ y $\underline{G_2} < G_1 \times_\alpha G_2$. Además,
\begin{align*}
|\underline{G_1}\cdot \underline{G_2}| &= \frac{|\underline{G_1}|\cdot |\underline{G_2}|}{|\underline{G_1} \cap \underline{G_2}|} = \frac{|\underline{G_1}|\cdot |\underline{G_2}|}{1} = |G_1|\cdot |G_2| = |G_1 \times_\alpha G_2|\\
\underline{G_1} \cap \underline{G_2} &= {(e_1, e_2)}
\end{align*}
Y podemos probar que $\underline{G_1}$ es normal, sean $g_1 \in G_1$ y $g_2 \in G_2$:
\begin{align*}
(g_1, g_2) \ast_\alpha (a, e_2) \ast_\alpha (g_1, g_2)^{-1} = (g_1,g_2)\ast_\alpha(\ldots, e_2\cdot g_2^{-1}) = (\ldots, e_2).
\end{align*}
\begin{cor}
	\label{cor:propiedadesgrupdirecto}
	Por lo que acabamos de ver:
	\begin{itemize}
		\item $\hat{G_1}$ y $\hat{G_2}$ son subgrupos.
		\item $\hat{G_1}$ es normal.
		\item $\underline{G_1} \cap \underline{G_2} = \{(e_1,e_2)\}$
		\item $\underline{G_1} \cdot \underline{G_2} = G_1 \times_\alpha G_2$
	\end{itemize}
	Si ahora tomamos $G_1 = N, G_2 = H$ con $N \normsub G, H < G$, entonces:
	\begin{itemize}
		\item $H \cap N = \{e\}$
		\item $H \cdot N = G$
		\item $\alpha: H \longrightarrow Aut(N)$
		\item $G \cong H \times_\alpha N$
	\end{itemize}
\end{cor}
En particular, podemos definir:
\begin{align*}
\phi : H &\longrightarrow Aut(N)\\
h &\longmapsto \gamma_h\mid_N(n) = h\cdot n\cdot h^{-1}
\end{align*}
\begin{ej}
	Sea el famoso grupo $D_4 = \{1,B,B^2,B^3,A,AB,AB^2,AB^3\}$ (ver ejemplo \ref{ej:famosogrupod4}). Tomamos $N = \langle B \rangle =\{1,B,B^2,B^3\},\ H = \langle A \rangle =\{1,A\}$. Entonces:
	\begin{align*}
	\phi: H &\longrightarrow \autom{N}\\
	A &\longmapsto ABA^{-1} = B^3
	\end{align*}
	Entonces como hemos visto: $D_4 \cong \{1,A\} \ast_\phi \{1,B,B^2,B^3\}$.
\end{ej}

\section{Producto semidirecto}

% --------------------
De \cite{dor96}



% ---------------------

Sea $G$ un grupo. Sea $N \normsub G$, $H < G$, $N \cap H = \{e\}$ y $NH = G$ (recordemos que por ser $N$ normal, $NH$ es grupo). Entonces $G \isom N \times H$.

Veamos quién es ese isomorfismo $\gamma : G \to N \times H$. Recordemos que considerando dos grupos $G_1, G_2$ y su producto directo $G_1 \times G_2$ existe un $\alpha : G_2 \to Aut(G_1)$. Veremos quien es este $\alpha$ para $H$ y $N$, es decir, quién es $\alpha: H \to Aut(N)$.

Construye $\alpha$ a partir de 4 isomorfismos.

\begin{proof}$ $\newline
	\begin{itemize}
		\item Comenzamos por definir una función $j: N\times H \to G,\ (n, h) \mapsto nh$. Es función está bien definida por teoría de conjuntos pero no es un homomorfismo de grupos\footnote{Ojo con por qué no es homomorfismo. Si tomamos $(n,h),(n', h') \in N \times H$ tenemos que $j((n,h)(n',h')) = nn'hh'$. Podríamos pensar que como $N$ es normal, podemos conmutarlo y obtener $nn'hh' = nhn'h' = j((n,h))j((n',h'))$. \textbf{Pero esto está mal.} Lo que significa ser normal es que para $h \in H$, se tiene que $nh = hn''$ para algún $n'' \in N$.}\footnote{Si los grupos son abelianos entonces sí es claro que es un homomorfismo. Lo que vamos a hacer es ver que dando una estructura especial, sí que es un homomorfismo de grupos incluso para grupos no abelianos}.
		\item Recordemos que por el teorema \ref{thm:cardinalidadproductolibre} tenemos que $|G| = |N||H| = |N \times H|$ por ser $N \cap H = \{e\}$.
		\item Volviendo a lo de la estructura especial. Dar una estructura especial es dar una operación para $N \times H$.
		\begin{itemize}
			\item Sea $A$ un conjunto. Es claro que si tenemos una biyección $\phi : A \to G$ podemos dotar a $A$ de alguna estructura para que sea un grupo.
			\item Para dotar a $A$ de estructura tenemos que definir la operación. Forzamos que para cada $a, a' \in A$ para los que se tiene $\phi(g) = a, \phi(g') = a'$ la operación sea $a a'  = \phi(gg')$.
			\item En este caso nuestro $A$ es $N \times H$. En lugar de utilizar la operación habitual del producto directo definimos otra operación. Para llegar a ella nos fijamos en $(n,h)(n',h') \mapsto nhn'h' = nhn'\inv{h}hh' = n(hn'\inv{h})hh' = nn'hh'$ (intercalamos el neutro, que es legal).
			\item Redefinimos la operación en $N \times H$ para que cuadre con este resultado. Llamaremos al nuevo grupo con la nueva operación $N \times_\phi H$: para $(n,h), (n',h')$ definimos $(n,h)\cdot (n',h') = (n(hn'\inv{h}), hh')$.
			\item Comprobamos que en este caso $j$ es un homomorfismo de grupos:
			\begin{align*}
			j : N \times_\phi H &\to G \\
			(n,h) &\mapsto nh \\
			(n',h') &\mapsto n'h' \\
			(n,h)\cdot(n',h') &\mapsto n(hn'\inv{h})hh' = nn'hh'
			\end{align*}
		\end{itemize} 
	\end{itemize}
\end{proof}

Es muy interesante por que es muy natural llegar a situaciones de esta manera. ¡Y les voy a dar una!\footnote{Sugerencia: leelo con voz de tomatito.}

\begin{ej}
	Sea $|G| = p \cdot q$ y supongamos $p < q$ primos. Por el teorema de Lagrange (\ref{thm:lagrange}) tenemos que existe un subgrupo $H_p < G$ con $|H_p| = p$ y análogamente $\exists H_q \mid |H_q| = q$. A primera vista podríamos pensar que puede haber varios grupos de orden $q$. Pues no.
\end{ej}

\begin{proof}
	Supongamos hay dos grupos $H, H'$ de orden $q$ distintos. La intersección tiene que dar un subgrupo y si los dos grupos tienen un número primo de elementos entonces la intersección solo puede ser el neutro, $H \cap H' = \{e\}$. Entonces por el teorema \ref{thm:cardinalidadproductolibre} tenemos que $|HH'| = q^2 > p\cdot q$ lo que es imposible. Luego sabemos que a lo sumo hay un grupo de orden $q$.
\end{proof}

(Sigue el ejemplo) Supongamos que ese único grupo de orden $q$ se llama $N$. Entonces $\phi_g(N) = N$ ya que un isomorfismo tiene que mandar un subgrupo de $q$ elementos en otro subgrupo de $q$ elementos y $N$ es el único. Por tanto $N \normsub G$. Aplicando el teorema de antes, tenemos que $G \isom N \times H$.

\begin{ej}
	Veamos un ejemplo con más pinta de problema. Demostrar que todo grupo de orden $77$ es cíclico.
\end{ej}

\begin{proof}
	Comenzamos por observar que $77 = 7 \cdot 11$. Por el teorema de Lagrange (\ref{thm:lagrange}) tenemos que existen $H, N < G \mid |H| = 7,\ |N| = 11$ y por lo visto en el ejemplo anterior, $N \normsub H$. Como antes llegamos a que $H \cap N = \{e\}$ y a que $|H N| = pq$. Para aplicar el teorema anterior vemos qué estructura tiene que tener $N \times_\phi H$, con $\phi:H\longrightarrow Aut(N)$.
	\\\\
	Vemos que $Aut(N) = Aut(\mathbb{Z}/11\mathbb{Z}) = \mathcal{U}(\mathbb{Z}/11\mathbb{Z}) = \mathbb{Z}/10\mathbb{Z}$, es decir, un grupo cíclico de 10 elementos.
	\\\\
	Entonces, $\phi$ es de la forma: $H = \mathbb{Z}/7\mathbb{Z} \longrightarrow Aut(\mathbb{Z}/11\mathbb{Z})$, por tanto, solo podemos definir el homomorfismo de grupos trivial. Esto hace que $N \times_\phi H$ es igual a $\mathbb{Z}/7\mathbb{Z} \times \mathbb{Z}/11\mathbb{Z}$.\\\\Por el corolario \ref{cor:propiedadesgrupdirecto} sabemos que $G \cong N \times_\phi H \implies G\cong \mathbb{Z}/7\mathbb{Z} \times \mathbb{Z}/11\mathbb{Z}$ que es cíclico por ser producto de cíclicos de órdenes coprimos.
\end{proof}

% 20181031

\section{Teoremas de Sylow}

Son muchos teoremas para grupos finitos en los que el orden se puede expresar como
\begin{align}
	|G| = p^s m,\ mcd(p, m) = 1, s \geq 1
\end{align}
Veremos y discutiremos 3 de ellos. Sirven sobre todo para contar cosas.

\begin{dfn}[P-subgrupo de Sylow]
	Dado $G$ con $|G| = p^sm$ con $mcd(p,m) = 1,\ s \geq 1$, un p-subgrupo de Sylow de $G$ es un subgrupo $P < G$ con $|P| = p^s$.
\end{dfn}

\begin{thm}
	[Primero de Sylow]
	\label{thm:sylow1}
	Sea $G$ un grupo tal que $|G| = p^s m,\ mcd(p, m) = 1, s \geq 1,\ p$ primo. Entonces existe un p-subgrupo de Sylow $H_1 < G$ con $|H_1| = p^s$.\footnote{Este teorema es el recíproco de algo que ya sabíamos. Podíamos afirmar que si $P < G$ y $|P| = p^s$ entonces $p^s$ dividía a $|G|$. Lo que dice el primer teorema de Sylow es que el recíproco es cierto.}
\end{thm}

El teorema de Cauchy (\ref{thm:cauchy}) es una versión más débil de este primer teorema de Sylow.

\begin{thm}[Segundo de Sylow]
	\label{thm:sylow2}
	Sea $G$ grupo con $|G| = p^s m, mcd(p, m) = 1, s \geq 1$. Sea $P$ un p-subgrupo de Sylow fijado. Si $Q$ es un p-subgrupo de Sylow de $G$ entonces $\exists g \in G \mid Q \subset gP\inv{g}$.
\end{thm}

\begin{thm}[Tercero de Sylow]
	\label{thm:sylow3}
	Sea $F = \{g P \inv{g} \mid g \in G \} = \{P = P_{1}, \dots, P_{n_p}\}$ el conjunto de p-subgrupos de Sylow de $G$. Entonces $n_p$ divide a $m$ y $n_p \equiv 1 \mod p$.
\end{thm}

Hemos hecho mucho hincapié en los subgrupos normales y tenemos que si $N \normsub G$ entonces existe $\pi:G \to G/N$ homomorfismo de grupos\footnote{Por teoría de conjuntos tenemos que $\pi$ es una función que existe y está bien definida, pero aquí interesa que además es homomorfismo.}. Además teníamos que $|G| = |G/N| \cdot |N|$.

También establecíamos una biyección entre los submódulos de $G$ que contienen a $N$ y los submódulos de $G/N$. Si $K$ es uno de ellos entonces $N \normsub G \implies N \normsub K$,
\begin{align*}
	K/N = \overline{K} \subset K/N \\
	|K| = |\overline{K}||N|
\end{align*}

Vamos a discutir el teorema. Recordemos que dado $G$ el centro $Z(G)$ es el conjunto de los elementos que conmutan con todos (ver definición \ref{dfn:centro}). Recordamos además las proposiciones \ref{pro:centronormal} y \ref{pro:subcentronormal} que nos dicen que el centro es normal y que cualquier subgrupo del centro es abeliano y normal. El centro está bien pero tampoco es para tanto: suele ser muy pequeño. WTF.

% TODO restate theorems

Aquí en medio ha desvariado bastante, remontándose hasta el teorema \ref{thm:correspondenciasubgrupos}.

\begin{proof}[Demostración del teorema de Sylow]
	Procedemos por inducción [fuerte] en $|G|$.
	\begin{itemize}
		\item Si $|G| = 1$ no hay mucho que probar porque son grupos muy tontos.
		\item Suponemos que\footnote{[La clase en silencio]. \textit{Orlando: Se pueden callar por favor.} [El silencio se hace más hueco]. \textit{Orlando: No hagan ruiditos. Me cuesta concentrarme} [agita las manos]. [Sigue la demostración.]} el teorema es válido para $|G| < n$. Distinguimos los siguientes casos:
		\begin{enumerate}
			\item $|Z(G)| = 0$
			\item $|Z(G)| \neq 0$. Entonces $Z(G)$ es un grupo abeliano no trivial. Es decir que $Z(G) \isom \Z/n_1\Z \times \dots \times \Z/n_l\Z$. Como $p$ divide a $|Z(G)|$ podemos suponer que $p$ divide a $n_1$. Entonces $\overline{(n_1/p)} \in \Z/n_1\Z$ y por tanto
			\begin{align*}
				(\overline{\left(\frac{n_1}{p}\right)}, \overline{0}, \dots, \overline{0}) \text{ tiene orden } p
			\end{align*}
			Es decir que tenemos un $H < Z(G)$ con $|H| = p$.
			
			Teníamos de antes que $|G/H| |H| = |G|$. Por inducción existe $\overline{K} < G/H$ de orden $p^{s-1}$. Aplicamos $|K| = |\overline{K}||H|$ y como $|H| = p,\ |\overline{K}| = p^{s-1}$ tenemos que $|H| = p^s$.
		\end{enumerate}
	\end{itemize}

	Lo hemos probado para una hipótesis en concreto pero falta algo (no sé el qué). Seguimos con la demostración.
	\begin{align*}
		|G| = |Z(G)| + [G:C(a_{s+1})] + \dots  + [G:C(a_r)]
	\end{align*}
	$|G|$ es no nulo módulo $p$ y $|Z(G)$ es nulo módulo $p$, por lo que necesariamente tiene que ocurrir que alguno de los $[G:C(a_i)]$ sea no nulo módulo $p$. Supongamos que es el primero, es decir, supongamos que $[G:C(a_{s+1})]$ es no nulo módulo $p$. Además tenemos que
	\begin{align*}
		\underbrace{|G|}_{p^sm} = \underbrace{|C(a)|}_{p^sm'}\cdot \underbrace{[G:C(a)]}_{\text{ no divisible por p}}
	\end{align*}
	Como $[G:C(a)] \geq 2,\ |C(a)| = p^sm' < |G|$ por inducción el subgrupo $C(a_{s+1})$ tiene un subgrupo de orden $p^s$.
\end{proof}

% 20181105

\begin{ej}
	Supongamos $|G| = 2^2 \cdot 11 \cdot 13$. Por el teorema de Sylow tenemos que existen subgrupos $P_2, P_{11}, P_{13} < G$ con órdenes $|P_2| = 2^2,\ |P_{11}| = 11,\ |P_{13}| = 13$. Sin embargo no podemos garantizar que exista un $Q$ con orden $|Q| = 2^2 \cdot 13$. Si ocurriera esto sería buenísimo porque existiría un $P < G$ con $P \cap Q = \{e\}$ y por tanto $P\cdot Q = G$ y automáticamente $G \isom P \times_{\phi} Q$. Esto no ocurre porque en general no sabemos si $P_2$ y $P_13$ son normales y por tanto no podemos garantizar que $Q = P_2 \cdot P_13$ sea siquiera un grupo.
	
	Lo interesante del ejemplo anterior es que si tenemos $G$ descompuesto como producto directo de dos grupos y uno de ellos es normal, entonces tenemos automáticamente un producto semidirecto. Sin embargo, si tenemos $G$ descompuesto en 3 grupos, no basta con que uno sea normal, sino que tienen que ser normales 2. Supongamos $G$ se descompone en $P,Q,R$. Necesitamos que $P$ sea normal para que $P\cdot Q$ sea grupo. Y necesitamos que $R$ sea normal para que $(P\cdot Q) \cdot R$ sea también un grupo y podamos dar un producto semidirecto.
\end{ej}



Resultado muy fuerte que hay que saber probar.

\begin{thm}
	\label{thm:interseccionneutroconmutan}
	Sea $G$ un grupo, $H_1, H_2 \normsub G \land H_1 \cap H_2 = \{e\}$. Entonces $\forall h_1 \in H_1,\ h_2 \in H_2$ se tiene que $h_1 h_2 = h_2 h_1$.
\end{thm}

\begin{proof}
	Probaremos que $h_1 h_2 \inv{h_1} \inv{h_2} = e$. Para ello probaremos que $h_1 h_2 \inv{h_1} = h_2$. Sabemos que por ser $H_2 \normsub G$ tenemos que $h_1 H_2 \inv{h_1} = H_2$. Es decir, que $h_1 h_2 \inv{h_1} \in H_2$. Si multiplicamos a la derecha por $\inv{h_2} \in H_2$ nos sigue quedando un elemento de $H_2$: $h_1 h_2 \inv{h_1} \inv{h_2} \in H_2$. Para $H_1$ tenemos lo mismo: $h_2 h_1 \inv{h_2} \inv{h_1} \in H_1$. Por alguna razón estos dos elementos son el mismo y como pertenece a ambos subgrupos entonces pertenece a la intersección y por tanto $h_1 h_2 \inv{h_1} \inv{h_2} = e$.
\end{proof}

\begin{ej}
	Consideramos $D_4$ que es un p-grupo pues $|D_4| = 2^3$. En este caso el centro no es el trivial: $Z(D_4) = \{1, B^2\}$.
\end{ej}

\begin{ej}
	Consideramos $H$ (el grupo de cuaterniones, ejemplo \ref{ej:grupocuaterniones}, y su retículo, figura \ref{fig:ig:reticulocuaterniones}) que también es un p-grupo pues $|H| = 2^3$. El retículo de este grupo es extraño y volvemos a tener que $Z(H) = \{1, B^2\}$.
\end{ej}

\begin{ej}
	Si $G$ es un p-grupo con $|G| = p^s$ entonces $G$ tiene subgrupos de orden $1, p, p^2, \dots, p^s$.
\end{ej}

\begin{proof}
	Procedemos por inducción en $s$. Para $s = 1$ es trivial: el subgrupo es el propio $G$.
	
	Supongamos que $|Z(G)| = p^{s'}$ con $s' \leq s$. Sabemos que $Z(G) \normsub G$ y además todo subgrupo de $Z(G)$ es normal en $G$. $\exists \alpha \in Z(G) \mid o(\alpha) = p$. Tenemos que $\langle \alpha \rangle < Z(G)$ y por tanto $\langle \alpha \rangle \normsub G$. Consideramos ahora $G \to G/\langle \alpha \rangle$. Tenemos que $|G/\langle \alpha \rangle| = p^{s-1}$
\end{proof}

\begin{ej}[de aplicación de los teoremas de Sylow]
	Sea $G$ con $|G| = 3\cdot 5$.
	\begin{itemize}
		\item Tenemos por el primer teorema de Sylow (\ref{thm:sylow1}) que existen $P_3, P_5 < G$ con $|P_3| = 3,\ |P_5| = 5$ (aplicamos el teorema dos veces primero cogiendo $p = 3$ y luego $p = 5$).
		\item Tenemos también que $P_3 \cap P_5 = \{e\}$ ya que los elementos de $P_3$ tienen orden que divide a 3 y los elementos de $P_5$ orden que divide a 5, por tanto, los elementos de la intersección tienen que tener orden que divida a 3 y a 5 por lo que solo puede ser el neutro.
		\item Como $P_3 \cap P_5 = \{e\}$ sabemos por el teorema \ref{thm:cardinalidadproductolibre} que $P_3 P_5$ tiene 15 elementos. Si fuéramos capaces de probar que alguno de ellos es normal tendríamos un producto semidirecto.
		\begin{itemize}
			\item Aplicamos el tercer teorema de Sylow (\ref{thm:sylow3}) para averiguar quién es $n_3$ (el número de 3-subgrupos de Sylow en $G$). Tomamos $|G| = 3^1 \cdot 5$ (cogemos $p = 3,\ m = 5$). Entonces $n_3 \in \{1, 5\}$ pues $n_3$ tiene que dividir a $m = 5$. Además $n_3 \equiv 1 \mod 3 \implies n_3 \in \{1, 4, 7, \dots\}$. Concluimos que $n_3 = 1$.
			\item De aquí concluímos que el único conjugado de $P_3$ es $P_3$ (solo hay un 3-subgrupo de Sylow en 3, es decir, $\{g P \inv{g} \mid g \in G\} = \{P\} \implies gP\inv{g} = P,\ \forall g \in G \implies gP = Pg,\ \forall g$) luego $P_3 \normsub G$.\footnote{Orlando: \textit{Esto es buenísimo!} [Se alegra muchísimo de lo que acaba de probar.]}
			\item Hacemos lo mismo con $n_5$ y obtenemos que $n_5 = 1$ y concluímos que $P_5 \normsub G$.
		\end{itemize}
		\item No solo uno de ellos es normal, sino que los dos son normales. Tenemos un producto semidirecto y concluímos que $G \isom \Z/3\Z \times \Z/5\Z$.
	\end{itemize}
\end{ej}

\begin{ej}
	Hacemos lo mismo con un grupo $G$ que tiene $|G| = 2\cdot 7$.
	\begin{itemize}
		\item Del primer teorema de Sylow (\ref{thm:sylow1}) tenemos que $\exists P_2, P_7 < G$ con órdenes $|P_2| = 2,\ |P_7| = 7$.
		\item Es claro que $P_7$ tiene que ser normal (de dibujarlo) pero aún así supongamos que no sabemos contar y somos creyentes de los teoremas de Sylow, veamos que $P_7$ es normal.
		\begin{itemize}
			\item Obtenemos $n_7$ del tercer teorema:
			\begin{align*}
				\begin{cases}
				n_7 \text{ divide a } 2 \\
				n_7 \equiv 1 \mod 7
				\end{cases} \implies n_7 = 1
			\end{align*}
			\item Análogamente obtenemos que $n_2 = 1$.
		\end{itemize}
		\item Volvemos a tener dos subgrupos normales y tenemos que $|P_2 \cdot P_7| = 2 \cdot 7$ (con un razonamiento análogo al de antes) de lo que obtenemos un producto semidirecto y por tanto $G \isom \Z/2\Z \times \Z/7\Z$.
	\end{itemize}
\end{ej}

\begin{ej}
	Consideramos el grupo $S_4$ que tiene orden $|S_4| = 4! = 4\cdot 3 \cdot 2 = 2^3 \cdot 3$.
	\begin{itemize}
		\item Del primer teorema de Sylow obtenemos que $\exists P_2, P_3 < S_4$ con $|P_2| = 8,\ |P_3| = 3$.
		\item ¿Será $S_4$ un producto semidirecto? ¿Será $P_2$ o $P_3$ un subgrupo normal?
		\begin{itemize}
			\item Veamos quien es $n_3$. Por el tercer teorema de Sylow (\ref{thm:sylow3}) tenemos que $n_3$ divide a $m = 8$ y que $n_3 \equiv 1 \mod p = 3$. Con estas condiciones tenemos que $n_3$ puede ser o bien 1 o bien 4.
			
			Recordemos que $\sigma \in S_4 \land o(\sigma) = 3 \iff \sigma$ es un ciclo de longitud 3. Y recordemos que en $S_4$ había 8 ciclos de longitud 3. Entonces tenemos que $n_3$ no puede ser 1 ya que en tal caso $P_3 \normsub S_4$ y por tanto en $S_4$ habría solo 2 ciclos de orden 3 resulta que hay ocho. Concluimos que $n_3 = 4$.\footnote{Efectivamente, de entre los 8 ciclos de longitud 3 que hay en $S_4$ salen 4 parejas que viven cada una en uno de los conjugados de $P_3$.}
			\item Veamos quien es $n_2 = \{gP_2\inv{g} \mid P\} = \{P_2 = P_2^{(1)}, \dots, P_2^{(n_2)}\}$. Por el tercer teorema de Sylow (\ref{thm:sylow3}) tenemos que $n_2$ divide a $m = 3$ y que $n_2 \equiv 1 \mod p = 2$. Con estas condiciones tenemos que $n_2$ puede ser o bien 1 o bien 3.
			
			Para $n_2 = 1$ tendríamos que $P_2 \normsub S_4$ y por tanto todos los elementos de orden par tendrían que vivir en $P_2$. De orden 2 hay 6 elementos y de orden 4 hay otros 6, es decir, que en $P_2$ que es un grupo de orden 8, viven al menos $6 + 6 = 12$ con lo cual llegamos a una contradicción. Por lo que necesariamente $n_2 = 3$.
		\end{itemize}
		\item Pues no, ninguno de los p-subgrupos de Sylow que encontramos es normal.
		\item No hemos conseguido un producto semidirecto, pero vamos a probar que $P_2 \isom D_4$ (y por extensión todos sus conjugados porque tenemos el isomorfismo de conjugación entre ellos). Para eso, haremos una presentación de $P_2$ análoga a la de $D_4$ (ver ejemplo \ref{ej:famosogrupod4}).
		\begin{itemize}
			\item Tomamos $A = (13), B = (1234)$. ¿Por qué? Por el contexto geométrico de $D_4$ que se puede ver en el ejemplo \ref{ej:famosogrupod4}. Recordemos que la $A$ es la simetría y $B$ es el giro.
			\item Vemos que todo funciona y que la presentación queda igual que la de $D_4$.
		\end{itemize}
	\end{itemize}
\end{ej}


% 20181106 teoría


% 20181107 teoría

Cogemos un grupo de Sylow $|G| = p^sm mcd(m,p) = 1, s \geq 1$. Tenemos para el $F$ del segundo tercer teorema de Sylow que $|F| = |F_1| + |F_2| + \dots + |F_l|$ donde cada $F_j = \{qP_{i_j}\inv{q} \mid q \in Q\}$ y $|F_j| = [Q:N_Q(P_{i_j})]$.

\begin{pro}
	Si $Q$ es un p-subgrupo de Sylow y $P'$ es un p-subgrupo de Sylow entonces el normalizador de $P'$ en $Q$ es
	\begin{align*}
		N_Q(P') = P' \cap Q
	\end{align*}
\end{pro}

De aquí obtenemos que $|F_j| = [Q:N_Q(P_{i_j})] = [Q:Q \cap P_{i_j}]$. Como $Q, P_{i_j}$ son p-subgrupos tienen órdenes que son potencias de $p$ por lo que $|F_j$ es cociente de potencias de $p$ y por tanto es potencia de $p$.

\begin{obs}[para la prueba del tercer teorema de Sylow]
	$n_p \equiv 1 \mod p$
\end{obs}

\begin{proof}
	En particular, tomamos $P = Q$. En este caso, la clase de $P$, $F_1 = \{pP\inv{p} \mid p \in Q = P\} = \{P\}$. $|F_2| = [Q:N_Q(P_{i_2})] = [P:P \cap P_{i_2}] = p_{r_2}$ porque $P$ y $P_{i_2}$ no son iguales.
\end{proof}

\begin{obs}Si $Q$ es un p-subgrupo de Sylow de $G$ entonces $Q \subset gP\inv{g}$ para algún $g \in G$.
\end{obs}

\begin{proof}
	Procedemos por refutación: supongamos que $Q \not \subset F$. Recordemos que 
	\begin{align*}
		|F| = |F_1| + |F_2| + \dots + |F_s| \qquad |F_k| = [Q:Q\cap P_{i_j}]
	\end{align*}
	Si afirmamos que $Q \not \subset Q$ entonces $|F_j|$ tiene que ser un múltiplo de $p$ ya que al hacer la intersección $Q \cap P_{i_j}$ obtenemos un conjunto propio. De este modo, $|F| = \sum |F_j|$ también es un múltiplo de $p$. La contradicción llega con la observación anterior, ya que $|F| \equiv 1 \mod p$.
\end{proof}

Lo interesante de verdad es el corolario que obtenemos de esta observación:

\begin{cor}
	$F$ es el conjunto de todos los subgrupos de Sylow de $G$.
\end{cor}

\begin{obs}
	Por último probaremos que $n_p \divides m$.
\end{obs}

\begin{proof}
	$F = \{g P \inv{g} \mid g \in G\}$ y tenemos que $|F| = [G:N_G(F)] \land |G| = p^s m \land P \subset N(P)$. Además
	\begin{align*}
		\underbrace{|G|}_{p^sm} = \underbrace{|P|}_{p^s}\underbrace{[G:P]}_{m}
	\end{align*}
	Ahora $P \subset N(P)$ y también $|G| = |N(P)[G:N(P)]$.
\end{proof}

\begin{ej}
	Consideramos $|S_5| = 5! = 5\cdot 4!$ tomamos $p = 5, m = 4!, s = 1$.
	\begin{itemize}
		\item Por el primer teorema tenemos que existen subgrupos de orden $p^s = 5$. Esto ya lo sabíamos. 
		\item De hecho hasta sabíamos que había $4! = 24$ ciclos de longitud 5. Como $p = 5$ es un número primo, los subgrupos de orden 5 no tienen elementos en común. Cada subgrupo tendrá 4 elementos y como hay 24 ciclos de orden 5 habrá 6 subgrupos de orden 5.
	\end{itemize}
\end{ej}

% 20181108

\begin{ej}
	\label{ej:clasificacionsylow}
	Sea $G$ un grupo, $H < G, N < G$ subgrupos. Recordemos que si $H\cap N = \{e\}, HN = G \land N \normsub G$ entonces existe un producto semidirecto para el que $G \isom H \times_\phi N$. Si $|G| = p^a q^b$ con $p \neq q$ primos, entonces existen $P_p, P_q < G$ con $|P_p| = a, |P_q| = b$. Además se tiene que $P_p \cap P_q = \{e\}, |P_pP_q| = |P_p||P_q|$ y por tanto $P_pP_q = G$.
	
	Realizamos un estudio sistemático de los grupos dado el orden similar al del teorema \ref{thm:clasificacionfinitos} pero utilizando los teoremas de Sylow
	\begin{itemize}
		\item Si $|G| = 1$ no tiene interés estudiarlo.
		\item Si $|G| = 2, 3$ entonces $G \isom \Z/2\Z$ o $G\isom \Z/3\Z$.
		\item Si $|G| = 4 = 2^2$ entonces $G$ es abeliano. Lo demostramos en la proposición \ref{pro:primocuadradoabeliano} para todo grupo de orden $p^2$ con $p$ primo.
		\item Si $|G| = 5$ entonces $G \isom \Z/5\Z$.
		\item Si $|G| = 6 = 2\cdot 3$ entonces $G \isom \Z/2\Z \times \Z/3\Z$ o $G \isom D_3$. Sabemos por Sylow que existen $P_2, P_3 \normsub G$ con $|P_2| = 2, |P_3| = 3$. Además del tercer teorema de Sylow obtenemos $n_3 = 1$, es decir que en $F_3$ tenemos solo un grupo. Para $n_2$ solo tenemos que $n_2 = 1, 3$. Ahora bien, como $n_3 = 1$ tenemos que $P_3 < G$. Por tanto, existe un producto semidirecto para el que $G \isom P_3 \times_\phi P_2  =?\ \Z/3\Z \times \Z/2\Z$.\footnote{Por convención ponemos el normal primero, para poder aplicar directamente la construcción sin liarnos.}
		
		Veamos que de este producto semidirecto nos salen dos estructuras. En primer lugar vemos quiénes son $N$ y $H$. En este caso el grupo normal es $P_3$ por lo que $N = P_3$ y $H = P_2$. Veamos los automorfismos interiores $Int:H \to Aut(\Z/3\Z) = (\{\overline{1}, \overline{2}\}, \cdot) = \uds{Z/3\Z}$. Como $Aut(\Z/3\Z)$ tiene dos elementos, obtenemos dos estructuras
		\begin{itemize}
			\item Si tomamos que $e_H \mapsto e_{Aut(Z/3\Z)} = \overline{1}$ entonces encontramos que $G \isom \Z/3\Z \times \Z/2\Z$.
			\item Si tomamos que $e_H \mapsto \overline{2}$ ocurre que $G \isom D_3$. Vamos a verlo.
			
			Supongamos que $P_3 = \langle a \rangle, o(a) = 3$. Si para algún $h\in H$ definimos la conjugación $hx\inv{h}$ para $x \in G$ tenemos que como $P_3 \normsub G$ entonces $hP_3\inv{h} = P_3$. Ahora supongamos que $H = P_2 = \langle b \rangle, o(b) = 2$. Entonces para un $b$, con el automorfismo seleccionado $a \mapsto ba\inv{b} = a^2 \implies ab = ba^2$ y llegamos a la presentación de $D_3$ (con las a's y las b's cambiadas.)
		\end{itemize}
		
		\item Si $|G| = 7$ entonces $G \isom \Z/7\Z$.
		\item Si $|G| = 8$ Sylow dice poco. Lo vimos en algún sitio %TODO cita requerida
		\item Si $|G| = 9$ tampoco tenemos mucho que decir
		\item Si $|G| = 10 = 2\cdot 5$. Como de costumbre sabemos que existen $P_2, P_5 < G$ con los ordenes correspondientes. Por el tercer teorema llegamos a que $n_5 = 1$ y por tanto a que $P_5 \normsub G$. Para $P_2$ no tenemos nada, pero solo por ser $P_5$ normal existe un producto semidirecto para el que $G \isom P_5 \times P_2 \isom \Z/5\Z \times_\phi \Z/2\Z$. Como en el caso de $|G| = 6$ obtendremos dos estructuras.
		
		Tomamos $N = P_5, H = P_2$. Tenemos que definir morfismos $Int: \Z/2\Z \to Auto(\Z/5\Z) = (\{\overline{1}, \overline{2}, \overline{3}, \overline{4}\}, \cdot ) = \uds{\Z/5\Z}$. Para ver cuantos morfismos salen veamos el orden de los elementos de $Aut(\Z/5\Z)$: Los elementos $\{\overline{1}, \overline{2}, \overline{3}, \overline{4}\}$ tienen órdenes $1,4,4,2$ respectivamente. En $\Z/2\Z = \{\overline{0}, \overline{1}\}$ tenemos dos posibilidades\footnote{Estamos abusando un poco de la notación de clases, ir con cuidado.} Un automorfismo viene dado por donde enviamos el generador de $\Z/2\Z$ en este caso el $\overline{1}$.
		\begin{itemize}
			\item Si $\overline{1} \mapsto \overline{1}$ obtenemos el homomorfismo trivial y por tanto la estructura dada por la presentación $o(a) = 5, o(b) = 2, ba\inv{b} = a \implies G \isom \Z/2\Z \times \Z/5\Z$ abeliano.
			\item Si $\overline{1} \mapsto \overline{4}$ la estructura que obtenemos es $o(a) = 5, o(b) = 2, ba\inv{b} = a^4 = a^{-1}$. Esta presentación es la del grupo $D_5$.
		\end{itemize}
	\item Si $|G| = 11$ pasa la historia de los primos.
	\item Si $|G| = 12 = 2^2 \cdot 3$. Entonces del tercero de Sylow tenemos $n_3 = 1, 4$ y $n_2 = 1, 3$. Tristeza.\footnote{Orlando: \textit{Sylow nunca dice toda la verdad, se puede hilar más fino.}}
	
	Ahora se le ocurre afirmar que no puede ocurrir que $n_2 = 3 \land n_3 = 4$ simultáneamente.
	
	Supongamos que $n_3 = 4$ entonces habría 4 subgrupos de orden 3 y por tanto habría $2 \cdot 4$ elementos de orden 3 (el neutro tiene orden 1). Ya tenemos 9 elementos bajo control. Para controlar los 12 nos faltan 3 elementos que llamaremos $a,b,c$ y que podrían formar un grupo con el neutro: $\{e, a, b, c\}$. Efectivamente esto dice Sylow, que hay un subgrupo de orden 4 ($a,b,c$ no pueden tener orden 3 porque si no no podrían pertenecer a un grupo de orden 4). Como ya hemos agotado los elementos, no es posible que haya más subgrupos de orden 4, por lo que necesariamente $n_2 = 1$.
	\item Así podemos seguir hasta $|G| = 29$ ya que cualquier orden menor que 30 es producto de como máximo dos primos.
	\end{itemize}
\end{ej}

\begin{ej}
	Sea $G$ abeliano y $|G| = 20 = 2^2\cdot 5$.
	\begin{itemize}
		\item Por el primer teorema de Sylow tenemos que $\exists P_4 < G, |P_4| = 4$.
		\item Por el segundo teorema, tenemos que todo subgrupo de orden 4 está en $F_4 = \{gP_4\inv{g} \mid g \in G\}$. Como $G$ es abeliano, $F_4$ solo tiene un elemento luego $P_4$ es el único subgrupo de orden 4.
		\item Análogamente concluímos que $P_5 < G$ es el único subgrupo de orden 5.
	\end{itemize}
\end{ej}

\begin{ej}
	Estudiamos el grupo $G = \langle a, b \rangle$ con presentación $o(b) = 4,\ o(a) = 3, ba\inv{b} = a^2$.
	
	Pendiente, posible ejercicio de examen.
\end{ej}


\begin{ej}
	Sea $|G| = 30$. Entonces $G$ no es un grupo simple.
	\begin{proof}
		Recordemos que $G$ es simple si sus únicos subgrupos normales son $G$ y $\{e\}$ (ver definición \ref{dfn:simple}).
		
		Tenemos que $|G| = 30 * 2 \cdot 3 \cdot 5$. Por el primer teorema de Sylow tenemos que $\exists P_5$ con $|P_5| = 5$. Además por el tercer teorema tenemos que $n_5 = 1, 6$. Análogamente tenemos $|P_3| = 3$ y $n_3 = 1, 10$.
		
		Supongamos que $n_5 = 6, n_3 = 10$. Sean $S_1, \dots S_6$ los 6 subgrupos de orden 5. Como 5 es primo entonces cada $S_i$ es cíclico de orden 5 y necesariamente $S_i \cap S_j = \{e\}$, porque si $S_i$ y $S_j$ compartieran algún elemento, entonces serían el mismo grupo pero hemos supuesto que había 6 subgrupos de orden 5. Cada $S_i = \{1, a, a^2, a^3, a^4\}$ por ser $5$ primo $\implies S_i$ cíclico, es decir, que tenemos $4 \cdot 6 = 24$ elementos distintos de orden 5 (en cada grupo tenemos el neutro que tiene orden 1 y otros cuatro que deben tener orden 5 por ser $S_i$ cíclico). Sean ahora $H_1, \dots, H_{10}$ los subgrupos de orden 3. Aplicando el mismo argumento, $H_i \cap H_j = \{e\},\ H_i = \{e, b, b^2\} \implies$ hay $2 \cdot 10 = 20$ elementos distintos de orden 3. Con esto llegaríamos a que en $G$ hay al menos $20 + 24 = 44 > 30$ elementos por lo que llegamos a una contradicción. Es decir, que necesariamente tiene que ocurrir que o $n_3 = 1$ o $n_5 = 1$, por lo que existe un subgrupo normal distinto de $G$ o $\{e\} \implies G$ no puede ser simple.
	\end{proof}
\end{ej}

\begin{ej}
	Sea $|G| = 48$. Entonces o bien $G$ tiene un subgrupo de orden 8 o bien $G$ tiene un subgrupo de orden 16.
	
	\begin{proof}
		Tenemos $|G| = 2^4 \cdot 3$. Por el primer teorema de Sylow tenemos que $\exists P_2,\ |P_2| = 2^4$ y por el tercer teorema tenemos que $n_2 = 1, 3$.
		
		ESTO TIENE UNAS LAGUNAS...
		\begin{itemize}
			\item Supongamos que $n_2 = 3$. Entonces $F_3 = \{gP_3 \inv{g} \mid g \in G\} = \{P_{2_1} = P_2, P_{2_2}, P_{2_3}\}$. Probaremos que algún elemento de $F_3$ tiene un subgrupo normal, es decir, $\exists H \normsub P_{2_i}$ para algún $i = 1, 2, 3$.
			\begin{itemize}
				\item Consideramos la intersección $P_{2_2} \cap P_{2_3}$. Tenemos que $|P_{2_2} \cap P_{2_3}| = 1, 2, 4, 8$. Supongamos que $|P_{2_2} \cap P_{2_3}| = 4$. Entonces $|P_{2_2} \cdot P_{2_3}| = \frac{|P_{2_2}||P_{2_3}|}{|P_{2_2} \cap P_{2_3}|} = \frac{16 \cdot 16}{4} = 48 = |G|$. Esto no puede ocurrir, tiene que haber algún elemento de orden $3$ y en $P_{2_i}$ no puede haber ningúno. Por tanto concluímos que $|P_{2_2} \cap P_{2_3}| > 4 \implies |P_{2_2} \cap P_{2_3}| = 8$.\footnote{Ojo aquí, lo que está haciendo es aplicar el teorema \ref{thm:cardinalidadproductolibre} con mucho arte. Podía haber probado con $|P_{2_2} \cap P_{2_3}| = 2$ pero en realidad no le hace falta, ya que $|P_{2_i}| = 16$ es fijo y por tanto la única manera de cambiar $|P_{2_2} \cdot P_{2_3}$ es tocando el denominador. De ahí concluye que $|P_{2_2} \cap P_{2_3}| > 4$.}
				\item Es claro que $P_{2_2} \cap P_{2_3} < P_{2_2}$. Por alguna razón tenemos que $P_{2_2} \cap P_{2_3} \normsub P_{2_2}$ y $P_{2_2} \cap P_{2_3} \normsub P_{2_3}$. Recordemos que $P_{2_2} \cap P_{2_3} \normsub G \iff N(P_{2_2} \cap P_{2_3}) = G$ y que el normalizador $N(H)$ siempre contiene a $H$ y es el menor grupo en el que $H \normsub N(H)$. Entonces $P_{2_2} \normsub N(P_{2_2} \cap P_{2_3}) \implies \forall g \in G, g(P_{2_2} \cap P_{2_3})\inv{g} = P_{2_2} \cap P_{2_3}$. En particular $\forall g \in P_2,\ g \in N(P_{2_2} \cap P_{2_3}) \land \forall g \in P_{2_3},\ g \in N(P_{2_2} \cap P_{2_3})$.
			\end{itemize}
		\end{itemize}
	\end{proof}
\end{ej}

\begin{ej}
	Consideramos $|S_4| = 4! = 2^3 \cdot 3$. Podemos hacer el mismo argumento que antes para los subgrupos de orden 3.
\end{ej}

\begin{pro}
	Sea $G$ un grupo, $H \normsub G, K \normsub G$ y $H \cap K = \{e\}$. Entonces $\forall h, k,\ h \in H, k \in K \implies hk = kh$.
\end{pro}

\begin{thm}
	Sea $G$ un grupo finito, $H \normsub G$ y $K \normsub G$. Entonces son equivalentes
	\begin{enumerate}
		\item $H \cap K = \{e\} \land HK = G$
		\item la función $H \times K \to G,\ (h, k) \mapsto hk$ es un isomorfismo de grupos
	\end{enumerate}
\end{thm}

\begin{proof}$ $\newline
	\begin{itemize}
		\item $1 \implies 2$. Lo primero decir que la función $H\times K \to G$ existe por teoría de conjuntos. Tenemos por el teorema \ref{thm:cardinalidadproductolibre} que $|HK| = |H||K|$. Con esto tenemos que la función es sobreyectiva porque $|H||K| = |G|$. Además es claro que la función es inyectiva. Además como $H\cap K = \{e\} \land H \normsub G \land K \normsub G$ tenemos que la función es un homomorfismo de grupos. Concluimos que la función es un isomorfismo de grupos.
		\item $2 \implies 1$. Sea $H \times e = \{(h, e) \mid h \in H\}$. Es claro que $H < H \times K$: es subgrupo porque es finito y es cerrado. Análogamente sea $e \times K = \{(e, k) \mid k \in K\}$ y $e \times K < H \times K$.
		
		Veamos ahora que $H \times e$, y por extensión, $e \times K$ son subgrupos normales en $H \times K$. Tenemos que probar que $\forall (a,b) \in H \times K,\ (a,b)(H \times e)\inv{(a,b)} = (H \times e)$. Sea $h \in H$, entonces
		\begin{align*}
			(a,b)(h, e)\inv{(a,b)} = (\underbrace{ah\inv{a}}_{\in H}, \underbrace{be\inv{b}}_{=e}) \in H \times e \implies H \times e \normsub H \times K
		\end{align*}
		Análogamente lo tenemos para $e \times K$.
		
		Además, es claro que $(H \times e) \cap (e \times K) = \{(e,e)\}$ que es el neutro de $H \times K$ y por el isomorfismo de la hipótesis $(e, e) \mapsto e \implies (H \times e) \cap (e \times K) \mapsto H \cap K = \{e\}$. 
		
		Por último tenemos que $(H \times e) \cdot (e \times K) = H \times K \isom G$ por hipótesis. Además, podemos obtener cualquier elemento de $HK$ con el mismo isomorfismo: $\forall h \in H, k \in K, (h, e)\cdot(e, k) \mapsto hk \in HK \implies HK = G$.
	\end{itemize}
\end{proof}

\begin{cor}
	Sea $G$ un grupo finito, $H \normsub G$ y $K \normsub G, N \normsub G$. Entonces son equivalentes
	\begin{enumerate}
		\item $H \times K \times N \to G,\ (h,k,n) \mapsto hkn$ es un isomorfismo de grupos.
		\item $H\cap (KN) = K \cap (HN) = N \cap (HK) = \{e\}$ y $HKN = G$.
	\end{enumerate}
\end{cor}

\begin{proof}
	Es análoga a la del teorema anterior.
\end{proof}

\begin{cor}
	Dados $H, K, N$ subgrupos normales de $G$ entonces $\forall g \in G$ existe una única operación para la que $g = hkn$ y dicha operación es el isomorfismo $H\times K \times N \to G$.
\end{cor}

% falta día 15 de noviembre

\begin{thm}
	Sea $G$ un grupo abeliano finito. Entonces $G$ es suma directa de sus p-subgrupos de Sylow.
\end{thm}

\begin{ej}
	Consideramos $G = \Z/12\Z$ que tiene $|G| = 2^2\cdot 3$. Se tiene que
	\begin{align*}
		P_2 = \{\overline{0}, \overline{3}, \overline{6}, \overline{9}\}\qquad P_3 = \{\overline{0}, \overline{4}, \overline{8}\} \qquad \Z/12\Z = P_2 \oplus P_3
	\end{align*}
\end{ej}

Sea ahora $G$ un p-grupo abeliano finito, es decir que $G$ es abeliano y que $|G| = p^r$. Es inmediato que $\forall g \in G,\ o(g) = p^s$ con $s \leq r$. Si tomamos $n \in \Z$ y utilizamos notación aditiva ($ng$ significa $g$ operado consigo mismo $n$ veces) tenemos que
\begin{align*}
	\alpha_n : G &\to G \\
	g &\mapsto ng
\end{align*}
es un homomorfismo de grupos (cuando $G$ es abeliano). Si ahora tomamos $p \in \Z$ con $p$ primo, tenemos que $\alpha_p(g) \mapsto pg$. Por el teorema de Lagrange (\ref{thm:lagrange}) tenemos que si $|G| = n \land p \divides n$ entonces $\exists \alpha \in G \mid o(\alpha) = p$. Como $|G| = p^r$ entonces $\alpha_p:G \to G$ no es inyectiva y por tanto $\emptyset \subsetneq \ker \alpha_p < G$. Vamos a profundizar en el subgrupo $\ker \alpha_p$. Este subgrupo es
\begin{align*}
	\ker \alpha_p = \{g \in G \mid o(g) \divides p\} = \{g \in G \mid o(g) = 1 \lor o(g) = p\}
\end{align*}
ya que $p$ es primo.


\begin{ej}
	Consideramos $G = \Z/p^r\Z$.
	\begin{align*}
		\Z/p^r\Z \to &\Z/p^r\Z \\
		\overline{0} \mapsto & \overline{1} \\
		\vdots & \mapsto \vdots \\
		\overline{p^{r-1}} & \mapsto \overline{0} \\
		\overline{p^r-1} &
	\end{align*}
\end{ej}

\begin{ej}
	Consideramos $G = \Z/p^r\Z \oplus \Z/p^s\Z$. Observemos que $G$ nunca va a ser cíclico porque $mcd(p^r, p^s) > 1$ (nunca serán coprimos). Definamos aquí el producto por $p$.
	\begin{align*}
		\alpha_p : G &\to G \\
		(\overline{a}, \overline{b}) \mapsto (p\overline{a}, p \overline{b}) \\
	\end{align*}
	
	En este caso es necesario que $\ker \alpha_p = \{(\overline{a}, \overline{b}) \mid p\overline{a} = \overline{0} \land p \overline{b} = \overline{0}\} = \langle \overline{p^{r-1}}\rangle \oplus \langle \overline{p^{s-1}} \rangle$ donde $\langle \overline{p^{r-1}}\rangle \isom \Z/p^r\Z$ y $\langle \overline{p^{s-1}}\rangle \isom \Z/p^s\Z$. En este caso (en el que el p-grupo no es cíclico) se observa que hay más de 1 subgrupo de orden $p$.
\end{ej}

¿Será verdad que esta observación caracteriza a los grupos cíclicos?

\begin{thm}[de alguien 1]
	Sea $G$ un p-subgrupo abeliano. Entonces $G$ es cíclico si y solo si tiene un único subgrupo de orden $p$.
\end{thm}

\begin{proof}
	Consideramos $\alpha_p : G \to G$. $\ker \alpha_p$ consiste en $\overline{0}$ y todos los elementos de orden $p$. Sea $N$ el único subgrupo de orden $p$. Por tanto $\ker \alpha_p = N$.
	
	La imagen $\ima \alpha_p$ es un subgrupo de $G$ y sabemos (por los teoremas de isomorfía) que $\ima \alpha_p \isom G / N$. Pongamos que $|G| = p^r$, por hipótesis, $\ker \alpha_o = N \land |N| = p$. En particular, tenemos que $\ima \alpha_p$ es un p-grupo abeliano de orden $|\ima \alpha_p \isom G/N| = \frac{p^r}{p} = p^{r-1}$.  Como $p \divides |\ima \alpha_p|$ tiene que existir un elemento de orden $p$ en $\ima \alpha_p$ y por tanto $N < \ima \alpha_p < G$. Es único porque si hubiera dos subgrupos de orden $p$ en $\ima \alpha_g$, también los habría en $G$ y hemos partido de lo contrario. Aplicando inducción podemos suponer que el criterio es válido para $\ima \alpha_p$ y por tanto podemos suponer que $\ima \alpha_p$ es cíclico.
	
	Sea $\overline{g} \in \ima \alpha_p$ que genera el subgrupo $\ima \alpha_p$. Entonces $\overline{g} \in G/N \isom \ima \alpha_p$. Fijamos un elemento $g \in G$ cuya imagen en $G / N$ sea $\overline{g}$. Recordemos que hay una correspondencia (\ref{thm:correspondenciasubgrupos}) entre los subgrupos de $G$ que contienen a $N$ y los subgrupos de $G/N$. Por tanto tenemos un elemento $g \in G$ que genera un subgrupo y quisiéramos ver que $\langle g \rangle = G$. Si demostramos que $N < \langle g \rangle$ entonces $\langle g \rangle = G$ por la correspondencia mencionada. Esta última afirmación ($N < \langle g \rangle = G'$) es válida porque $G'$ es un p-grupo $\implies G'$ contiene un elemento de orden $p \implies N < G' = \langle g \rangle$. Luego $\langle g \rangle = G$.
\end{proof}