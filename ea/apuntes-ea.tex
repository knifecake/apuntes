\documentclass{book}

\usepackage[utf8]{inputenc}  % para que funcionen las tildes
\usepackage{hyperref}
\usepackage{amsmath}
\usepackage{amssymb}
\usepackage{amsthm}
\usepackage{txfonts}
\usepackage{lmodern}
\usepackage[dvipsnames]{xcolor}
\usepackage{thmtools, thm-restate}
\usepackage{datetime} % para la hora de compilación
\usepackage{graphicx}
\graphicspath{{images/}}
\usepackage[spanish,es-noquoting]{babel} % es-noquoting es para que funcione tikz
\usepackage{mathabx} % para \divides
\usepackage{centernot} % para \centernot\inculdes
\usepackage{wrapfig}
\usepackage{multicol}
\usepackage{subcaption}
\usepackage{xfrac}
\usepackage{standalone}
\usepackage{tikz}
\usetikzlibrary{arrows.meta}
\usetikzlibrary{shapes}
\usetikzlibrary{decorations.text}
\usetikzlibrary{positioning}
\usetikzlibrary{external}
\tikzexternalize[prefix=./tikzbuild/]
\tikzexternalize % activate!

%\usepackage[a6paper,margin=5mm]{geometry}
\usepackage[a4paper, top=1.5cm,bottom=1.5cm,left=1cm,right=1cm]{geometry}

%\setlength{\parindent}{0pt}
\usepackage{parskip}

% ESTILOS DE DEFINICIONES Y TEOREMAS
\declaretheoremstyle[
	bodyfont=\normalfont,
	shaded={
		margin=8pt,
		bgcolor=White,
		rulecolor=Black,
		rulewidth=1pt
}]{mythm}

\declaretheoremstyle[
	bodyfont=\normalfont,
	shaded={
		margin=1em,
		bgcolor={rgb}{0.9,0.9,0.9}
}]{mydfn}

\declaretheoremstyle[
bodyfont=\normalfont,
spacebelow=1em,
spaceabove=1em,
]{myej}

% DEFINICIONES DE ENTORNOS DE TEOREMAS
\declaretheorem[
	name=Teorema,
	refname={teorema,teoremas},
	Refname={Teorema,Teoremas},
	style=mythm
]{thm}

\declaretheorem[
	name=Corolario,
	refname={corolario,corolarios},
	Refname={Corolario,Corolarios},
	style=myej
]{cor}

\declaretheorem[
	name=Proposici\'{o}n,
	refname={proposici\'{o}n,proposiciones},
	Refname={Proposici\'{o}n, Proposiciones},
	sharenumber=thm,
	style=myej
]{pro}

%\newtheorem*{cor}{Corolario}
\newtheorem*{lem}{Lema}

\declaretheorem[
	name=Definici\'{o}n,
	refname={definici\'{o}n,definiciones},
	Refname={Definici\'{o}n,Definiciones},
	style=mydfn
]{dfn}

\declaretheorem[
name=Ejercicio,
refname={ejercicio,ejercicios},
Refname={Ejercicio,Ejercicios},
style=myej,
numbered=no
]{ex}

\declaretheorem[
name=Ejemplo,
refname={ejemplo,ejemplos},
Refname={Ejemplo,Ejemplos},
style=myej
]{ej}

\declaretheorem[
name=Observaci\'{o}n,
refname={observaci\'{o}n,observaciones},
Refname={Observaci\'{o}n,Observaciones},
style=myej
]{obs}

% COMANDOS ÚTILES PARA LA TEORÍA DE GRUPOS
%\newcommand{\normsub}{\mathbin{\triangleleft}}
\newcommand{\normsub}{\lhd}
\newcommand{\uds}[1]{\mathcal{U}(#1)}
\newcommand{\inv}[1]{#1^{-1}}
\newcommand{\ima}{\text{Im }}
\newcommand{\isom}{\simeq}
\newcommand{\autom}[1]{\text{Aut}(#1)}
\newcommand{\gen}[1]{\langle#1\rangle}
\newcommand{\biy}[1]{\text{Biy}(#1)}

%\DeclareMathSymbol{\varprod}{\mathop}{largesymbolsA}{16}

\newcommand{\hr}{\rule{\textwidth}{.4pt}}
\newcommand{\N}{\mathbb{N}}
\newcommand{\Z}{\mathbb{Z}}
\newcommand{\R}{\mathbb{R}}
\newcommand{\Q}{\mathbb{Q}}
\newcommand{\ZnZ}{\mathbb{Z}/n\mathbb{Z}}
\newcommand{\ZmZ}{\mathbb{Z}/m\mathbb{Z}}


\renewcommand\qedsymbol{$\clubsuit$}

\title{Apuntes de Estructuras Algebraicas}
\author{Elias Hernandis}

\begin{document}
\maketitle
Revisión del \today $ $ a las \currenttime.

\tableofcontents

\part{Primer parcial - hoja 1}

\chapter{Grupos}

\section{Grupos}

\begin{dfn}[Grupo]
	Llamamos grupo al par $(G, \ast)$, donde $G$ es un conjunto no vacío y $\ast: G \times G \to G$ es una función que cumple las siguientes propiedades:
	\begin{enumerate}
		\item Clausura. $\forall a, b \in G, a \ast b \in G$
		\item Asociatividad. $\forall a, b, c \in G,\ (a \ast b) \ast c = a \ast (b \ast c)$
		\item Elemento neutro. $\exists e \in G, \forall a \in G \mid a \ast e = e \ast a = a$
		\item Elemento inverso. $\forall a \in G, \exists \inv{a} \in G \mid a \ast \inv{a} = \inv{a} \ast a = e$
	\end{enumerate}
\end{dfn}

En general, la clausura es muy difícil de probar, por lo que recurrimos a dar un grupo como subgrupo de otro o dar una biyección entre un grupo existente y lo que queremos probar que es grupo.

\paragraph{Notación}

\begin{itemize}
	\item Aunque técnicamente el grupo es el par $(G, \ast)$, es común referise al grupo como $G$.
	\item Cuando la operación es la suma, se suele llamar al elemento neutro $e = \mathbf{0}$. Cuando la operación es el producto, se suele llamar al elemento neutro $e = \mathbf{1}$.
	\item Denotamos por $a^k$:
	\begin{itemize}
		\item si $k > 0,\ a^k = \underbrace{a \ast a \ast \dots \ast a}_\text{k veces}$
		\item si $k = 0,\ a^0 = e$
		\item si $k < 0,\ a^k = \underbrace{\inv{a} \ast \inv{a} \ast \dots \ast \inv{a}}_\text{-k veces}$
	\end{itemize}
	\item Se suele omitir la operación. Sobre todo cuando la operación es el producto. Por ejemplo, en $(G, \cdot)$, $a \cdot b = ab$.
\end{itemize}

\begin{thm}[Propiedad cancelativa]
	Sea $G$ un grupo, $a, b, c \in G$.
	\begin{align}
		a \ast b = a \ast c \implies b = c \\
		c \ast a = b \ast a \implies a = b
	\end{align}
\end{thm}

\begin{proof}
	Por la existencia del elemento inverso podemos multiplicar por $\inv{a}$ a la izquierda en la primera expresión y obtenemos $\inv{a} a b = \inv{a} a c \implies e b = e c \implies b = c$. Lo mismo ocurre por la derecha en la segunda expresión.
\end{proof}

\begin{pro}[Unicidad del elemento neutro]
	En un grupo $G$ hay exactamente un elemento neutro $e$.
\end{pro}

\begin{proof}
	Supongamos existen $e_1, e_2 \in G$ elementos neutros. Por ser $e_1$ elemento neutro se tiene que $e_1 \ast e_2 = e_2$ y por ser elemento neutro $e_2$ se tiene que $e_1 \ast e_2 = e_1$. Por tanto $e_1 = e_2$.
\end{proof}

\begin{pro}[Unicidad del inverso de un elemento]
	Sea $G$ un grupo, $g \in G$, entonces $\exists! \inv{g} \mid g \ast \inv{g} = e$. 
\end{pro}

\begin{proof}
	Supongamos $a$ tiene inversos $b_1$ y $b_2$. Entonces $a \ast b_1 = a \ast b_2 = e$. Por la propiedad cancelativa $b_1 = b_2$.
\end{proof}

\begin{dfn}[Orden de un elemento]
	Sea $(G, \ast)$ un grupo. Decimos que $a \in G$ tiene orden finito si $\exists k \in \mathbb{N}$ tal que $a^k = e$.
	Si existen tales valores de $k$, llamamos orden del elemento $a$ al mínimo de ellos:
	\begin{align}
		o(a) = \min \{k \in \mathbb{N} \mid a^k = e \}
	\end{align}
\end{dfn}

\begin{dfn}[Orden o cardinalidad de un grupo]
	Sea $G = \{a_1, a_2, \dots \}$ un grupo junto con alguna operación. Si $|G| < \infty$ decimos que el orden de $G$, $|G| = |\{a_1, a_2, \dots, a_n\}| = n$.
\end{dfn}

\begin{dfn}[Grupo abeliano]
	Sea $(G, \ast)$ un grupo. Diremos que $G$ es abeliano $\iff \forall a,b \in G,\ a \ast b = b \ast a$.
\end{dfn}

\begin{thm}
	\label{thm:abelianosdeorden2}
	Sea $G$ un grupo tal que $\forall g \in G,\ g \ast g = e$. Entonces $G$ es abeliano.
\end{thm}

\begin{cor}
	$\forall a \in G,\ o(a) = 2 \implies G$ es abeliano.
\end{cor}

\begin{proof}
	Sean $a,b \in G$. Tenemos que probar que $a\ast b = b \ast a$. Consideramos el elemento $(a \ast b) \in G$ por clausura. Por hipótesis tenemos que $(a \ast b) \ast (a \ast b) = e \implies (a \ast b) = \inv{(a \ast b)} = \inv{b} \ast \inv{a} = b \ast a$.
\end{proof}

\subsection{Ejemplos de grupos}



Por último, vemos una manera de generar nuevos grupos a partir de grupos existentes.

\begin{dfn}[Producto directo de grupos]
	Sean $(G_1, \ast), (G_2, \bullet)$ grupos. Llamamos producto directo de los grupos $G_1$ y $G_2$ al grupo $(G_1\times G_2, \sim)$. Donde $\sim : (G_1 \times G_2) \times (G_1 \times G_2) \to G_1 \times G_2,\ (g_1, g_2) \sim (g_1', g_2') = (g_1\ast g_1', g_2 \bullet g_2')$.
\end{dfn}

\section{Subgrupos}

\begin{dfn}[Subgrupo]
	Sea $(G, \ast)$ un grupo, $S \in G, S \neq \emptyset$. Diremos que $(S, \ast)$ es un subgrupo de $(G, \ast)$ y lo denotaremos por $S < G$ si verifica las siguientes condiciones:
	\begin{enumerate}
		\item Clausura. $\forall a, b,\ a,b \in S \implies a \ast b \in S$
		\item Elemento neutro. $e \in S$
		\item Elemento inverso. $\forall s \in S, \inv{s} \in S$ 
	\end{enumerate}
	(La propiedad asociativa siempre se hereda.)
\end{dfn}

En caso de que el grupo del que elegimos el subgrupo sea finito, la clausura no es tan complicada de probar.

\begin{pro}
	Si $\{S_i\}_{i \in \mathbb{N}}$ es una familia de subgrupos de $G$, entonces $\bigcap S_i$ también es un subgrupo de $G$.
\end{pro}

% TODO: demostrar


\begin{dfn}[Subgrupo generado varios elementos]
	\footnote{Este teorema reemplaza al de \textit{grupo generado por dos elementos} dado en clase.}Sea $(G, \ast)$ un grupo, $S \subset G,\ S \neq \emptyset$. El subgrupo generado por $S$ es
	\begin{align}
	\langle S \rangle = \{s_1^{\alpha_1} \ast s_2^{\alpha_2} \ast \dots \ast s_n^{\alpha_n} \mid s_1, s_2, \dots, s_n \in S,\ \alpha_1, \alpha_2, \dots, \alpha_n \in \Z \}
	\end{align}
\end{dfn}

\begin{pro}
	El subgrupo generado por $S$, $\langle S \rangle$ es el más pequeño que contiene a $S$.
\end{pro}

El siguiente teorema no lo ha dado drácula\footnote{De verdad que quería poner el nombre.} pero no me acuerdo pero viene en \cite{dor96} y simplifica bastante la bida.

\begin{thm}
	\label{thm:subgrupoxinverso}
	Sea $G$ un grupo y $H$ un subconjunto de $G$. Entonces $H < G \iff \forall x,y \in H, x\inv{y} \in H$.
\end{thm}

\begin{proof}
	De \cite{dor96}.
	\begin{itemize}
		\item ($\implies$). Supongamos que $H < G$. Entonces $x,y \in H \implies xy \in H \land y \in H \implies \inv{y} \in H$ y por tanto $x\inv{y} \in H$.
		\item ($\impliedby$). Supongamos que $x,y \in H \implies x\inv{y} \in H$. Veamos que se cumplen las 3 condiciones para que sea subgrupo:
		\begin{itemize}
			\item Elemento neutro. Tomamos $y = x$ y tenemos que $x\inv{x} = e \in H$.
			\item Elemento inverso. Tomamos ahora $x = e,\ y = x$ y tenemos que $e\inv{x} = \inv{x} \in H$.
			\item Clausura. Tenemos que si $x,y \in H$ por la propiedad anterior $\inv{y} \in H$ y por tanto $xy = x\inv{(\inv{y})} \in H$.
		\end{itemize}
	\end{itemize}
\end{proof}

% TODO: probar que es subgrupo y que es el más pequeño

Normalmente, utilizaremos la definición restringida a un elemento:

\begin{dfn}[Subgrupo generado por un elemento]
	\label{dfn:subgrupogenerado}
	Sea $G$ un grupo, $g \in G$. Llamamos subgrupo generado por $g$ a
	\begin{align}
		\langle g \rangle = \{g^k \mid k \in \mathbb{Z}\}
	\end{align}
\end{dfn}

\begin{pro}
	El subgrupo generado por $g \in G$ en efecto es un subgrupo.
\end{pro}

\begin{proof}$ $\newline
	\begin{enumerate}
		\item Es cerrado por $\ast$ puesto que $\forall a^k, a^{k'} \in S, a^k \ast a^{k'} = a^{k + k'} \in S$.
		\item $a^0 = e \in A$
		\item $\forall a^{k}, a^{-k} \in A$
	\end{enumerate}
\end{proof}

\begin{pro}
	Si $o(g) = n$, entonces $\langle g \rangle$ tiene $n$ elementos (el orden de $\langle g \rangle$ es $n$).
\end{pro}

\begin{proof}
	Primero comprobamos que no hay más de $n$ elementos distintos. Consideramos $k \in \Z,\ k = cn + r$ para algunos $c, r \in \Z,\ 0 \leq r < n$ por el algoritmo de la división. Entonces $a^k = a^{cn + r} = a^{cn} a^{r} = a^{r}$ pues $o(a) = n$.
	
	Ahora probaremos que no hay menos de $n$ elementos distintos, es decir, que $\langle g \rangle = \{1, g, g^2, \dots, g^{n-1}\}$ Supongamos existen $0 \leq i < j < n$ tales que $a^i = a^j$. Entonces por cancelación $a^{j - i} = e = a^0 \implies j = i$ lo que da una contradicción.
\end{proof}

\begin{thm}
	Sea $G$ un grupo, $g \in G$. El menor subgrupo de $G$ que contiene a $g$ es $\langle g \rangle$.
\end{thm}

\begin{proof}
	Tenemos que probar que para cualquier $H$ subgrupo de $G$, $g \in H \implies g^k,\ \forall k \in \Z$.
\end{proof}

\begin{dfn}[Grupo cíclico]
	Sea $(G, \ast)$ un grupo. Diremos que $G$ es cíclico si $\exists g \in G \mid \langle g \rangle = G$.
\end{dfn}

\begin{thm}
	\label{thm:ciclicoimplicaabeliano}
	Si $G$ es cíclico entonces $G$ es abeliano.
\end{thm}

\begin{proof}
	Tenemos que probar que $\forall a,b \in G,\ ab = ba$. Sabemos que $a = g^i, b = g^j$ para algunos $i, j \in \Z \implies ab = a^i a^j = a^{i+j} = a^{j+1} = a^j a^i = ba$.
\end{proof}


\begin{thm}
	\label{thm:coprimosgeneradosiguales}
	Sea $g \in G$ tal que $o(g) = n \in \N \geq 1$ y sea $r \in \N$. Si $r$ y $n$ son coprimos, entonces $\langle g \rangle = \langle g^r \rangle$.
\end{thm}

\begin{cor}
	Si $r$ y $n = o(g)$ son coprimos entonces $o(g) = o(g^r)$.
\end{cor}

\begin{proof}
	Recordamos que $p$ y $q$ son coprimos $\iff\ \exists \alpha, \beta \in \Z \mid \alpha p + \beta n = 1$. Recordamos que $\langle g \rangle = \{1, g, g^2, \dots, g^{n-1}\}$ donde $n = o(g)$. Tenemos que probar la doble inclusión. Fijémonos en que $g^r \in \langle g \rangle \implies \langle g^r\rangle \subset \langle g \rangle$ pues $\langle g \rangle$ contiene a todos los elementos de la forma $g^k,\ k \in \Z$ (ver definición \ref{dfn:subgrupogenerado}). Ahora probaremos que $\langle g \rangle \subset \langle g^r \rangle$. Como $r$ y $n$ son coprimos, $g = g^{\alpha r + \beta n} = (g^r)^\alpha (g^n)^\beta = (g^r)^\alpha \in \langle g^r \rangle \implies \langle g \rangle \subset \langle g^r \rangle$. Concluimos que $\langle g \rangle = \langle g^r \rangle$.
\end{proof}

\begin{ej}
	En $\Z/4\Z = \{0, 1, 2, 3\}$ con la suma tomamos $g = 1$ y por tanto $n = o(g) = 4$, y tomamos $r = 3$ y por tanto $mcd(n, r) = 1$. Efectivamente se verifica que $o(1^3) = o(1+1+1) = o(3) = 4 = o(1)$ o lo que es lo mismo, $\langle 1 \rangle = \langle 3 \rangle$.
\end{ej}

\begin{pro}
	\label{thm:ordenescoprimos}
	Sea $g \in G$ tal que $o(g) = n$ y sea $r \in \N$ con $r \divides n$ ($r$ divide a $n$). Entonces $o(g^r) = \frac{n}{r}$.
\end{pro}

\begin{proof}
	Sea $n'$ tal que $n = rn'$. Probaremos que $r\divides n \implies o(g^r) = n'$.
	\begin{align*}
		\langle g^r \rangle = \{g^r, g^{2r}, g^{3r}, \dots, g^{n'r} = g^n\} \subset \{g, g^2, g^3, \dots, g^n\} = \langle g \rangle
	\end{align*}
	$\langle g^r \rangle$ tiene $n'$ elementos distintos porque para cualquier $i = 0,\dots, n'$, $o(g^{ir}) <= o(g) = n$ por lo que no se repite ninguno. Además cualquier $g^{ir}$ está bien definido porque al dividir $r$ a $n$, $ir \in \N$.
\end{proof}

\begin{thm}[Hoja 1, ejercicio 9]
	Sea $o(g) = n \in \N$ y sea $N \in \Z$. Entonces $o(g^N) = \frac{o(g)}{mcd(N, o(g))}$.
\end{thm}

\begin{proof}
	Afirmamos que $n$ y $N/d$, con $d = mcd(N,n)$ son coprimos. Expresamos $g^N = (g^{N/d})^d$. Por el [corolario del] teorema $\ref{thm:coprimosgeneradosiguales}$ tenemos que $o(g^{N/d}) = o(g) = n$. Por la proposición $\ref{thm:ordenescoprimos}$ tenemos que $o((g^{N/d})^d) = \frac{o(g^{N/d})}{d} = \frac{n}{d}$.
\end{proof}

\begin{thm}[Hoja 1, ejercicio 7]
	\label{thm:subconjuntocerrado}
	Sea $(G, \ast)$ un grupo y $S \subset G,\ S \neq \emptyset$ un subconjunto finito de $G$. Si $S$ es cerrado por la operación $\ast$ entonces $S$ es un subgrupo de $G$.
\end{thm}

\begin{proof}
	Se verifican las 3 propiedades
	\begin{enumerate}
		\item Clausura. Por hipótesis.
		\item Elemento neutro. Sea $s \in S$. Si $s = e$ ya hemos terminado. Si $s \neq e$, sabemos que $\{s^1, s^2, \dots\} \subset S$. Pero $S$ es finito $\implies \exists\ 0 < i < j$ tales que $s^i = s^j \implies s^{j - i} = e$. Como $j > i \implies j - i > 0$, hemos obtenido $e$ de operar $s$ consigo mismo, luego $e \in S$.
		\item Elemento inverso. Tomamos $r = j - i$ de la propiedad anterior. Tenemos $s^r = e \implies s \ast s^{r-1} = e \implies s^{r-1} = s^{-1}$.
	\end{enumerate}
\end{proof}

%TODO cambiar la definicion de unidades por la del conjunto de los elementos de orden $m$
%\begin{dfn}[Conjunto de unidades]
%	Sea $A$ un anillo donde el elemento identidad respecto del producto es $1$. Entonces
%	\begin{align}
%		\uds{A} = \{a \in A \mid \exists b,\ a\cdot b = b\cdot a = 1\}
%	\end{align}
%\end{dfn}

%\begin{ej}
%	Las unidades son interesantes porque (a veces?) generan grupos.
	
%	En $\Z/4\Z$ las unidades son $\uds{Z/4\Z} = \{\overline{1}, \overline{3}\}$. Este ejemplo es particularmente interesante porque $(\uds{\Z/4\Z}, \cdot)$ es un grupo. No es un subgrupo porque no hereda la operación de $(\Z/4\Z , +)$.
%\end{ej}

%\begin{pro}
%	$\overline{a} \in \ZnZ \iff mcd(a,n) = 1$ en $\Z$.
%\end{pro}
%
%\begin{proof}$ $\newline
%	\begin{itemize}
%		\item ($\implies$) $\overline{a} \in \uds{\ZnZ} \iff \exists \overline{b} \in \ZnZ \mid \overline{a}\overline{b} = \overline{ab} = 1 \iff ab = rn + 1 \iff 1 = ab - rn \iff \alpha a + \beta n = 1 \iff mcd(a,n) = 1$
%		\item ($\impliedby$) $mcd(a,n) = 1 \iff \alpha a + \beta n = 1 \implies \overline{\alpha a + \beta n} = \overline{1} \iff \overline{\alpha}\overline{a} + \overline{\beta}\overline{n} = \overline{1} \iff \overline{\alpha}\overline{a} = 1 \implies \exists \overline{b} \in \Z \mid \overline{a}\overline{b} = 1$ (el $\overline{b}$ es $\overline{\alpha}$).
%	\end{itemize}
%\end{proof}


\subsection{El teorema de Lagrange}

\begin{dfn}[Clase lateral]
	Sea $(G, \ast)$ un grupo, $H < G,\ g \in G$. Definimos
	\begin{itemize}
		\item $g \ast H = gH = \{g \ast h \mid h \in H\}$ es una clase lateral izquierda de $H$
		\item $H \ast g = Hg = \{h \ast g \mid h \in H\}$ es una clase lateral derecha de $H$
	\end{itemize}
\end{dfn}

\begin{thm}
	\label{thm:ordencajaslaterales}
	Si $H < G$ tiene orden $n < \infty$ entonces $|gH| = |Hg| = |H| = n$.
\end{thm}

\begin{proof}
	Consideramos la aplicación $f: H \to gH,\ f(h) \to g\ast h$ para un $g \in G$ dado. Es inyectiva: $f(h_1) = f(h_2) \implies h_1 = h_2$ puesto que $xh_1 = xh_2 \implies h_1 = h_2$ por la propiedad cancelativa. Es sobreyectiva porque $\forall h \in H,\ g \ast h = f(h)$. Por tanto $f$ es biyectiva y los órdenes son iguales.
\end{proof}

\begin{pro}
	Sea $H < G,\ g \in G$. Las clases laterales $gH$ y $Hg$ cumplen las siguientes propiedades (las cumplen las dos pero damos solo las de la izquierda):
	\begin{enumerate}
		\item $g \in H \iff g\ast H = H$
		\item $g \in g \ast H \implies G = \bigcup_{g \in G} g \ast H$
		\item $g' \in g \ast H \implies g' \ast H = g \ast H$
		\item $g_1 \ast H \cap g_2 \ast H \neq \emptyset \implies g_1 \ast H = g_2 \ast H$
	\end{enumerate}
\end{pro}

\begin{proof}
	(solo de la última propiedad)
	Sabemos que existe $\alpha \in g_1 \ast H \cap g_2 \ast H$ de la forma $\alpha = g_1 \ast h_1 = g_2 \ast h_2,\ h_1, h_2 \in H$. Ahora bien, $g_1 \ast h_1 = g_2 \ast h_2 \iff \inv{g_2} \ast g_1 \ast h_1 = h_2 \iff \inv{g_2}g_1 \in H \implies g_2(\inv{g_2}g_1)H = g_2(\inv{g_2}g_1H) = g_2 H$.
\end{proof}

De las propiedades anteriores se obtiene que $\{g_i \ast H\}_{g_i \in G}$ es una partición de $G$. Además, por el teorema \ref{thm:ordencajaslaterales}, como $|g \ast H| = |H|$ la partición divide $G$ en cajas iguales (ver cuadro \ref{table:cajasiguales}). Pongamos que $G$ es finito y que hay $r$ cajas, entonces $|G| = r|g_i \ast H| = r|H| \implies |H| \mid |G|$. A continuación veremos otra forma de dar esta relación de equivalencia.


Para algún $H < G$, la partición que hemos dado anteriormente es la definida por la relación de equivalencia $g_1 R g_2 \iff g_1 \ast H = g_2 \ast H$. Otra manera de definirla es $g_1 R g_2 \iff \inv{g_2}g_1 \in H$. Se verifica que esta nueva definición es una relación de equivalencia.

\begin{figure}[h]
	\centering
	\renewcommand{\arraystretch}{1.5}
	\begin{tabular}{|c|c|c|}
		\hline
		$g_1 \ast H$ & $g_2 \ast H$ & $\dots$ \\\hline
		$\dots$ & $H$ & $\dots$ \\\hline
		$\dots$ & $g_{r-1} \ast H$ & $g_r \ast H$\\\hline
	\end{tabular}
	\caption{Partición de $G$ en $r$ cajas iguales}
	\label{table:cajasiguales}
\end{figure}

%TODO: probar que es una relación de equiv


\begin{thm}[de Lagrange]
	\label{thm:lagrange}
	Sea $G$ un grupo finito y $H < G$. Entonces $|H| \divides |G| $ (el orden de $H$ divide al orden de $G$).
\end{thm}

\begin{cor}
	Sea $G$ un grupo y $g \in G$. Entonces $o(g) \divides |G|$ (el orden de un elemento divide al orden del grupo).
\end{cor}

\begin{cor}
	Si $G$ es un grupo de orden $p$, con $p$ primo, entonces $G$ es cíclico.
\end{cor}

\begin{proof}
	Sea $g \in G,\ g \neq e$. Por el teorema de Lagrange $|\langle g \rangle| \divides |G| = p$. Como $p$ es primo sus únicos divisores son $1$ y $p$ y como $|\langle g \rangle| > 1$ se ha de tener $|\langle g \rangle| = p$. Por tanto $\langle g \rangle = G$ y $G$ es cíclico. 
\end{proof}

\subsection{Subgrupos normales y grupo cociente}


\begin{dfn}[Subgrupo normal]
	Sea $H < G$. Diremos que $H$ es un subgrupo normal de $G$ y lo denotaremos por $H \normsub G \iff \forall g \in G,\ g\ast H = H \ast g$.  
\end{dfn}

\begin{pro}
	Si $G$ es abeliano entonces todos sus subgrupos son normales.
\end{pro}


\begin{dfn}[Conjunto cociente en grupos]
	Sea $H < G$. Definimos
	\begin{align}
	G/H = \{gH \mid g \in G\} = \{\overline{x} \mid \overline{x} = \{g \in G \mid \inv{g}x \in H\}\}
	\end{align}
\end{dfn}

\begin{pro}
	Sea $H \normsub G$. $(G/H, \ast)$ con la operación $\ast: G/H \to G/H, (xH)(yH) \mapsto (xy)H$ es un grupo.
\end{pro}

\begin{proof}
	La operación $\ast$ está bien definida. $\forall \overline{x}, \overline{y} \in G/H,\ \overline{x} \ast \overline{y} = xHyH = xyHH = xyH = \overline{x \ast y}$.
	
	El elemento neutro es $\overline{e}$ pues $\forall \overline{x} \in G/H,\ \overline{e} \ast \overline{x} = eHxH = exH = xH = \overline{x}$.
	
	El elemento inverso está bien definido: $\inv{\overline{x}} = \overline{\inv{x}}$ pues $\forall \overline{x} \in G/H,\ \overline{x}\inv{\overline{x}} = xH \inv{x}H = x\inv{x}H = eH = \overline{e}$.
\end{proof}

\begin{dfn}[Índice]
	Sea $H < G$. Definimos el \textbf{índice de $H$ en $G$}, y lo representamos mediante $[G:H]$, como el cardinal del conjunto cociente $G/H$. \cite{dor96}
\end{dfn}

\begin{thm}
	\label{thm:indice2normal}
	De \cite{dor96}\footnote{No lo hemos dado explícitamente pero se utiliza para algunos ejemplos.}
	Sea $H < G$ con $[G : H] = 2$ (con índice de $H$ en $G$ igual a 2). Entonces $H$ es normal.
\end{thm}

% !TeX root = ../apuntes-ea.tex


\chapter{Homomorfismos de grupos}

\section{Homomorfismos de grupos}

\begin{dfn}[Homomorfismo de grupos]
	Sean $(G_1, \cdot), (G_2, \ast)$ grupos. Decimos que $f: G_1 \to G_2$ es un homomorfismo de grupos si $\forall a,b \in G_1,\ f(a\cdot b) = f(a) \ast f(b)$.

	\begin{itemize}
		\item si $f$ es inyectiva, $f$ es un monomorfismo
		\item si $f$ es sobreyectiva, $f$ es un epimorfismo
		\item si $f$ es biyectiva, $f$ es un isomorfismo
		\item si $G_2 = G_1$ y $f$ es un isomorfismo, entonces $f$ se llama automorfismo
	\end{itemize}
	Si existe un isomorfismo entre dos grupos, decimos que son isomorfos y lo denotamos por $G_1 \isom G_2$.
\end{dfn}

\begin{figure}[h]
	\centering
	\begin{tikzpicture}[scale=0.7]
	\node (a) at (0,1) {$a$};
	\node (b) at (0,0) {$b$};
	\node (ab) at (0,-1) {$a\ast b$};
	
	\node (fa) at (4,1) {$f(a)$};
	\node (fb) at (4,0) {$f(b)$};
	\node (fab) at (4,-1) {$f(a)\ast f(b)$};
	\draw (0,0) ellipse (.9 and 2);
	\draw (4,0) ellipse (1.8 and 2);
	
	\draw (0, -2) node[anchor=north] {$G_1$};
	
	\draw (4, -2) node[anchor=north] {$G_2$};
	
	\draw[-{Latex[length=2mm]}] (a) -- (fa);
	\draw[-{Latex[length=2mm]}] (b) -- (fb);
	\draw[-{Latex[length=2mm]}] (ab) -- (fab);
	\end{tikzpicture}
	\caption{Homomorfismo de grupos}
	\label{fig:homomorfismo}
\end{figure}


\begin{dfn}[Núcleo de un homomorfismo]
	Sea $f:G_1 \to G_2$ un homomorfismo. Definimos el núcleo $\ker f = \{x \in G_1 \mid f(x) = e_2 \in G_2\}$ (los que van a parar al neutro).
\end{dfn}

\begin{dfn}[Imagen de un homomorfismo]
	Sea $f:G_1 \to G_2$ un homomorfismo. Definimos la imagen $\ima f = \{y \in G_2 \mid \exists x \in G_1, f(x) = y\}$.
\end{dfn}

\begin{pro}Sea $f: G_1 \to G_2$ un homomorfismo. $\ker f < G_1$.
\end{pro}

\begin{proof} Probamos las 3 propiedades de los subgrupos
	\begin{enumerate}
		\item $a,b \in \ker f \implies a \cdot b \in \ker f$. $f(a \cdot b) = f(a) \ast f(b) = e_2 \ast e_2 = e_2$.
		\item $a \in \ker f \implies a^{-1} \in \ker f$. $f(a) = e_2,\ f(a^{-1}) = e_2 \implies (f(a))^{-1} = e_2$.
		\item $e_1 \in \ker f$.
	\end{enumerate}
\end{proof}

\begin{thm}
	Sea $f: G_1 \to G_2$ un homomorfismo. $\ima f < G_2$.
\end{thm}

\begin{proof} Es análoga a la del $\ker f$.\end{proof}

\begin{thm}
	Sea $f : G_1 \to G_2$ un homomorfismo. $\ker f \normsub G_1$
\end{thm}

\begin{proof}
	Tenemos que probar que $\forall a \in G_1, a (\ker f) a^{-1} \subset \ker f$.
	
	Sea $h \in \ker f$. $f(a h a^{-1}) = f(a)\underbrace{f(h)}_{e_2}f(a^{-1}) = f(a)f(a^{-1}) = e_2\subset \ker f$
\end{proof}

\begin{pro}
	Sea $f:G_1 \to G_2$ un homomorfismo de grupos. $f$ es inyectiva si y solo si $\ker f = \{e\}$.
\end{pro}

\begin{proof}$ $ \newline
	\begin{itemize}
		\item $(\impliedby$) Suponemos que $f$ es inyectiva. Sabemos que en un homomorfismo $f(e_1) = e_2$ y además $\ker f = {e_1}$ por hipótesis.
		\item $(\implies)$ Tenemos que probar que dados $a,b \in G_1,\ f(a) = f(b) \implies a = b$. Decir que $f(a) = f(b)$ es lo mismo que decir $e_2 = f(a)^{-1}f(b) = f(a^{-1}) f(b) = f(a^{-1}b) \implies a^{-1}b \in \ker f = \{e_1\} \implies a = b$.
	\end{itemize}
\end{proof}

\begin{pro}
	Sean $G_1, G_2, G_3$ grupos y sean $f:G_1 \to G_2,\ g:G_2 \to G_3$ homomorfismos de grupos. Entonces $g \circ f$ es a su vez un homomorfismo de grupos.
\end{pro}

\begin{thm}
	Sea $f:G_1 \to G_2$ un homomorfismo de grupos. Entonces $o(f(g))$ divide a $o(g)$.
\end{thm}

\begin{thm}
	Sea $f:G_1 \to G_2$ un isomorfismo de grupos. Entonces $o(g) = o(f(g))$.
\end{thm}

\begin{proof}
	Consideramos $f$ y $f^{-1}$ para los que se verifica el teorema anterior. $o(g) \mid o(f(g)) \land o(f(g)) \mid o(f^{-1}(f(g))) = o(g) \implies o(g) = o(f(g))$. 
\end{proof}

%20180925

\section{Retículo de subgrupos}

\begin{dfn}[Retículo de subgrupos]
	Dado un grupo $G$, el retículo de subgrupos es un grafo con todos los subgrupos de $G$. Denotamos la relación de inclusión con un vértice entre dos grupos. Es costumbre poner el mayor grupo arriba y denotar la inclusión por las diferencias en altura.
\end{dfn}

Lo importante de esta sección:
\begin{itemize}
	\item Todo subgrupo de un grupo cíclico es cíclico.
	\item Dado un epimorfismo entre dos grupos existe una correspondencia biyectiva entre los subgrupos del primero y los del segundo.
	\item En $\ZnZ$ existe un subgrupo por cada divisor de $n$ y esos son todos los subgrupos que hay.
\end{itemize}

\begin{ej}[Retículo de subgrupos $\Z$]
	$\Z$ tiene infinitos subgrupos, todos los $k\Z$. En muchas ocasiones nos va a interesar solo dibujar unos pocos, para relacionarlos con subgrupos de otros grupos distintos de $\Z$. A continuación se muestra el retículo de subgrupos de $\Z$ construido a partir de $6\Z$.
	
	\begin{figure}[h]
		\centering
		\begin{tikzpicture}
		\node (z) at (0,1) {$\Z$};
		\node (2z) at (-1,0) {$2\Z$};
		\node (3z) at (1,0) {$3\Z$};
		\node (6z) at (0,-1) {$6\Z$};
		
		\draw (z) -- (2z);
		\draw (z) -- (3z);
		\draw (2z) -- (6z);
		\draw (3z) -- (6z);
		\end{tikzpicture}
		\caption{Una parte del retículo de subgrupos de $\Z$, en concreto la de los $n\Z$ con $n \divides 6$.}
	\end{figure}

	Los grupos que contienen a $6\Z$ son los de la forma $k\Z$ donde $k$ divide a $6$, ya que entre los múltiplos de los divisores de $6$ también se encuentran los múltiplos de $6$.
\end{ej}

\begin{pro}
	Sea $n = \min_{r \in \N,\\r > 0} \{ r \in H,\ H < \Z\}$. Entonces $nH = \Z$.
\end{pro}
\begin{proof}
	Probamos la doble inclusión. Por hipótesis $n \in H$ y por tanto $\langle n \rangle = n\Z \subset H$. Sea $\alpha \in H$. Por el algoritmo de la división, podemos expresar $\alpha = an + s$ con $0 \leq s < n \implies s = 0 \implies H \subset n\Z$. Luego $H = n\Z$.
\end{proof}

El siguiente teorema no lo ha dado Orlando explícitamente pero básicamente lo que dice es lo que dijo en las 3 clases sobre correspondencia entre subgrupos pero un poco más ordenado.

\begin{thm}[de correspondencia entre subgrupos mediante homomorfismos]
	Sea $f:G_1 \to G_2$ un homomorfismo de grupos. Se tiene \cite{dor96}:
	\begin{enumerate}
		\item Si $H_1 < G_1$ entonces $f(H_1) < G_2$
		\item Si $H_2 < G_2$ entonces $f^{-1}(H_2) = \{h_1 \in G_1 \mid f(h_1) \in H_2\} < G_2$
		\item Si $H_2 \normsub G_2$ entonces $f^{-1}(H_2) \normsub G_1$
		\item Si $H_1 \normsub G_1$ y $f$ es además sobreyectiva (es un epimorfismo) entonces $f(H_1) \normsub G_2$
	\end{enumerate}
\end{thm}

\begin{proof}$ $\newline
	\begin{enumerate}
		\item Demostramos que se cumplen las 3 propiedades de los grupos. Sabemos que $e_1 \in H_1 \implies e_2 \in f(H_1) = H_2$. Además, sabemos que $\forall x \in H_1,\ \inv{x} \in H_1$ y por ser $f$ un homomorfismo tenemos que $\forall f(x) \in H_2,\ \inv{f(x)} = f(\inv{x}) \in H_2$. Por último, tenemos que $\forall x,y \in H,\ xy \in H_1 \implies \forall f(x),f(y) \in H_2,\ f(x)f(y) = f(xy) \in H_2$.
		\item Es análoga a la de la primera afirmación.
		\item Tenemos que probar que para un $g_1 \in G_1,\ \forall h_1 \in f^{-1}(H_2) = H_1,\ g_1 h_1 = h_1 g_1$. Sabemos que $\forall h_1,\ \exists h_2 \in H_2 \mid \inv{f}(h_2) = h_1$. Entonces $g_1h_1 = h_1 g_1 \iff \inv{f}(g_2)\inv{f}(h_2) = \inv{f}(h_2)\inv{f}(g_2) \iff \inv{f}(g_2h_2) = \inv{f}(h_2g_2)$ que es cierto por hipótesis de que $H_2$ es normal.
		\item Tenemos que probar que para $g_2 \in G_2$ dado, $\forall h_2 \in H_2 = f(H_1),\ g_2h_2 = h_2g_2$. Comenzamos por asegurar que $\exists g_1 \in G_1 \mid f(g_1) = g_2$ por ser $f$ sobreyectiva. Por tanto $g_2h_2 = h_2 g_2 \iff f(g_1)f(h_1) = f(h_1)f(g_1) \iff f(g_1h_1) = f(h_1g_1)$ que es cierto por hipótesis.
	\end{enumerate}
\end{proof}

Queremos establecer una relación entre los retículos de subgrupos de dos grupos que son el dominio y la imágen de un epimorfismo $f: G_1 \to G_2$. Los subgrupos de $G_2$ siempre contendrán al elemento neutro $e_2$ por lo que podemos establecer una relación natural entre los subgrupos de $G_1$ que contienen a $\ker f$ con los subgrupos de $G_2$.

\begin{thm}\label{thm:correspondenciasubgrupos}\footnote{Este teorema es un desastre. Las hipótesis no las ha dado y las conclusiones tampoco. Es lo que más o menos he creido que quería decir. Es posible que se corresponda con la proposición 4.4.6 del \cite{dor96} pero en dicha proposición no se exige que $f$ sea sobre.}
	Sea $f: G_1 \to G_2$ un epimorfismo. Existe una biyección entre el retículo de subgrupos de $G_2$ y subgrupos de $G_1$ que contienen al $\ker f$. Se cumple que $H_2 < G_2 \iff \inv{f}(H_2) \supset \ker f$.
	
	En particular, el número de subgrupos de $G_2$ es igual al número de subgrupos de $G_1$ que contienen al núcleo.
	\begin{align*}
		|\{H_2 \mid H_2 < G_2\}| = |\{H_1 < G_1 \mid \ker f \in H_1\}|
	\end{align*}
\end{thm}

\begin{proof}
	Sabemos que por ser $f$ homomorfismo, $H_1 < G_1 \implies f(H_1) < G_2$.
	
	Veamos que la relación entre los subconjuntos de $G_1$ y de $G_2$ se mantiene al aplicar el epimorfismo. Sea $H_2 \subset G_2$. Como $f$ es sobre $f(\inv{f}(H_2)) = H_2$. Ahora sea $H_2' \mid H_2 \subset H_2' \subset G_2$. Ocurre lo de antes y además $\inv{f}(H_2) \subset \inv{f}(H_2') \subset G_1$.
	
	Ahora lo extendemos de subconjuntos a subgrupos. Asociamos a cada $H_2 < G_2$ el subgrupo $\inv{f}(H_2) < G$. Es un subgrupo porque al ser $f$ epimorfismo mantiene la operación. En particular, $e_2 \in H_2 \implies \ker f = \inv{f}(e_2) \subset \inv{f}(H_2)$.
	
	Por último afirmamos que si $\ker f \subset H_1 < G_1$, entonces $H_1 = \inv{f}(f(H_1))$. Para probar esto probamos la doble inclusión. $H_1 \in \inv{f}(f(H_1))$ es evidente pues $h \in H_1 \implies f(h) \in f(H_1)$. Ahora probamos $\ker f \subset H_1 \implies H \subset \inv{f}(f(H_1))$.
	\begin{align*}
		\alpha \in \inv{f}(f(H_1)) \iff& f(\alpha) \in \inv{f}(f(H_1)) \\
		\iff& \exists h_1 \in f(H_1) \mid f(\alpha) \in f(H_1) \\
		\iff& \exists h_1 \in H \mid f(\alpha)\inv{(f(h_1))} = e_2 \\
		\iff& \exists h_1 \in H_1 \mid f(\alpha \inv{h_1}) = e_2 \\
		\iff& \exists h_1 \in H_1 \mid \alpha \inv{h_1} \in \ker f \\
		&\alpha \inv{h_1} h_1 \implies \alpha \in H_1
	\end{align*}
\end{proof}

\begin{figure}[h]
	\centering
	\begin{tikzpicture}
		\draw (-2,0) ellipse (1.3 and 2);
		\draw (2, 0) ellipse (1 and 1.5);
		\draw (-2,-2.5) node {$G_1$};
		\draw (2, -2) node {$G_2$};
		
		\node (ker) at (-2, 0) {$\ker f$};
		\node (e2) at (2,0) {$e_2$};
		
		\draw (ker) ellipse(.5 and .5);
		\draw (e2) ellipse (.3 and .3);
		\draw (-2,.5) -- (2,.3);
		\draw (-2,-.5) -- (2,-.3);
		
		% subgrupos de G1
		\draw (ker) ellipse (.8 and 1.2);
		\draw (-2,-1.2) node[anchor=north] {$H_1$};
		
		% subgrupos de G2
		\draw (e2) ellipse (.7 and .9);
		\draw (2, -.9) node[anchor=north] {$H_2$};
	\end{tikzpicture}
\end{figure}

\section{Teoremas de la isomorfía (versión de clase)}

\begin{pro}[O ejemplo]
	Sea $f: \ZnZ \to \ZnZ$. $f$ es un isomorfismo $\iff f(\overline{1}) = \overline{a} \in \uds{\ZnZ}$
\end{pro}

\begin{ej}
	Sea $g \in G$ fijado. Definimos $\phi_g : G \to G$
	\begin{align*}
		G \to^{\phi_g} & G \to^{\phi^{-1}_g} & G \\
		x \mapsto &gx\inv{g} & \\
		&z \mapsto &\inv{g}x\inv{(\inv{g})}
	\end{align*}
	Y $\phi_g \cdot \phi_g^{-1} = Id$.
\end{ej}

\begin{proof}
	Para que $f$ sea isomorfismo tiene que ser sobre luego $o(\overline{a}) = n \implies \overline{a} \in \uds{\ZnZ}$.
\end{proof}

\begin{thm}
	\label{teorema:preisomorfia1}
	Sea $f: G_1 \to G_2$ un homomorfismo de grupos, $H \normsub G_1$ con $H \subset \ker f$. Sea $\pi: G_1 \to G_1/H$ el homomorfismo que genera las clases de equivalencia (ver figura \ref{fig:tmisomorfia1}). Entonces se cumple lo siguiente
	\begin{enumerate}
		\item existe un homomorfismo de grupos $\overline{f}:G_1/H \to G_2$ tal que $\overline{f} \circ \pi = f$
		\item $\ker \overline{f} = \ker f / H$
	\end{enumerate}
\end{thm}

\begin{figure}[h]
\centering
\begin{tikzpicture}
\node (g1) at (0,0) {$a \in G_1$};
\node (g2) at (4,0) {$f(a) \in G_2$};
\node (gh) at (0, -3) {$\overline{a} \in G_1/H$};

\draw[-{Latex[length=2mm]}] (g1) -- (g2) node[pos=.5, above] {$f$};
\draw[-{Latex[length=2mm]}] (g1) -- (gh) node[pos=.5, left]{$\pi$};
\draw[-{Latex[length=2mm]}] (gh) -- (g2) node[pos=.5, below] {$\overline{f}$};
\end{tikzpicture}
\caption{Homomorfismos que intervienen en el teorema \ref{teorema:preisomorfia1}}
%TODO
\label{fig:otracosa}
\end{figure}

\begin{proof} $ $\newline
	\begin{enumerate}
		\item Probaremos que si construimos $\overline{f}$ con $\overline{f}(\overline{a}) = f(a)$ entonces $\overline{f}$ está bien definida. Tenemos que ver que $\overline{a} = \overline{a'} \implies f(a) = f(a')$. Partimos de $\overline{a} = \overline{a'} \implies a \inv{(a')} \in H \implies f(a \inv{(a')}) = e_2 \implies f(a) \inv{f(a')} = e_2 \implies f(a) = f(a')$.
		
		\item Observemos que $\overline{f}(\overline{a}\overline{b}) = \overline{f}(\overline{ab}) = f(ab) = f(a)f(b)=\overline{f}(\overline{a})\overline{f}(\overline{b})$. Ahora probamos las dos inclusiones a la vez $\overline{a} \in \ker \overline{f} \iff \overline{f}(\overline{a}) = e_2 \iff f(a) = e_2 \iff \overline{a} \in \ker f$.
	\end{enumerate}
\end{proof}

\begin{thm}[Primer de la isomorfía]
	Sea $f:G_1 \to G_2$ un epimorfismo. Existe un isomorfismo $\overline{f}: G_1 / \ker f \to G_2$.
\end{thm}

\begin{figure}[h]
	%TODO hacer
	\centering
	\begin{tikzpicture}
	\node (g1) at (0,0) {$G_1$};
	\node (g2) at (4,0) {$G_2$};
	\node (gh) at (0, -3) {$G_1/\ker f$};
	
	\draw[-{Latex[length=2mm]}] (g1) -- (g2) node[pos=.5, above] {$f$};
	\draw[-{Latex[length=2mm]}] (g1) -- (gh) node[pos=.5, left]{$\pi$};
	\draw[-{Latex[length=2mm]}] (gh) -- (g2) node[pos=.5, below] {$\overline{f}$};
	\end{tikzpicture}
	\caption{Primer teorema de la isomorfía.}
	\label{fig:tmisomorfia3}
\end{figure}

\begin{proof}
	$f = \pi \circ \overline{f}$ y $f$ es sobre, luego $\overline{f}$ también es sobreyectiva.
	
	% TODO probar que f barra es inyectiva \implies es isomorfismo
\end{proof}

\begin{thm}[Segundo teorema de la isomorfía]
	
	Sean $H \normsub G,\ K \normsub G$ y $H \subset K$ Entonces
	\begin{align}
		(G/H)/(K/H) = G/K
	\end{align}
\end{thm}

\begin{figure}[h]
	%TODO hacer
	\centering
	\begin{tikzpicture}
	\node (g1) at (0,0) {$G$};
	\node (g2) at (4,0) {$G/K$};
	\node (gh) at (0, -3) {$G/H$};
	
	\draw[-{Latex[length=2mm]}] (g1) -- (g2) node[pos=.5, above] {$h$};
	\draw[-{Latex[length=2mm]}] (g1) -- (gh) node[pos=.5, left]{$\pi$};
	\draw[-{Latex[length=2mm]}] (gh) -- (g2) node[pos=.5, below] {$\overline{h}$};
	\end{tikzpicture}
	\caption{Segundo teorema de la isomorfía.}
	\label{fig:tmisomorfia2}
\end{figure}

\begin{proof}
	$\overline{h}$ es sobreyectiva y $\ker \overline{h} = K/H$
\end{proof}

% 20180927

\begin{thm}
	Sea $f:G_1 \to G_2$ un epimorfismo. Si $N \normsub G_1$, entonces $f(N) \normsub G_2$. Como $f$ es epimorfismo cualquier $g \in G_2,\ g_2 = f(g_1)$ para algún $g_1 \in G_1$. Como $N \normsub G_1$, tenemos que $gN\inv{g} \in N$. Que $f(N) \normsub G_2$ quiere decir que $\forall f(g) \in G_2, f(g)f(N)f(\inv{g}) \subset f(N)$. Ahora bien $f(g)f(N)\inv{f(g)}$. Y esto sigue pero lo ha dicho y no lo ha escrito y no me ha dado tiempo.
\end{thm}

\begin{lem}
	Sea $h:G_1 \to G_2$ homomorfismo de grupos. Sean $N \normsub G_1$ y $N \subset \ker h$. 
	\begin{enumerate}
		\item Entonces existe un homomorfismo de grupos $\overline{f}:G_1/N \to G_2$ que cumple $\overline{f} \circ \pi = f$
		\item $\ker \overline{f} = \ker f / N$.
	\end{enumerate}
\end{lem}

\begin{cor}
	Si $N = \ker f$ entonces $\ker \overline{f} = \{0\}$ y $\overline{f}$ es un monomorfismo.
\end{cor}

\begin{cor} % TODO esto es el primer teorema otra vez
	Si $f$ es además un epimorfismo, entonces $\overline{f}$ es una biyección.
\end{cor}

% TODO poner aquí la imágen del primer teoremaº



\begin{proof}
	Consideramos $f:H \to HK$ que es un homomorfismo porque $H < HK$ (porque $h = he_k,\ \forall h \in H$ y satisface la definción de producto). Y ahora consideramos un epimorfismo $h:HK \to HK/K$ que existe porque $K \normsub HK$. Sea $\pi = f \circ g$. Afirmamos que $\ker \pi = H \cap K$. Faltan cosas.
	
	\begin{align*}
		H/(H\cap K) \isom HK/K
	\end{align*}
\end{proof}

\begin{cor}
	Si $H,K < G$ con $K \normsub G$ entonces existe un epimorfismo $\pi:H\to HK/K$ y $\ker \pi = H \cap K$.
\end{cor}

\begin{thm}\footnote{Esta vez si que dijo teorema.}
	Sea $f:G_1 \to G_2$ un homomorfismo de grupos. Entonces $\ima f \isom G_1 / \ker f$.
\end{thm}

Este teorema viene a decir que dado un homomorfismo $f:G_1 \to G_2$, si lo restringimos a $f:G_1 \to \ima f$ obtenemos un epimorfismo.

%_---------------

\begin{pro}
	Sea $G$ un grupo con orden $n$. Sea $H < G$ con índice de $H = p \mid mcd(p,n) = 1$. Entonces $H$ es un subgrupo normal.
\end{pro}

\section{Teoremas de la isomorfía (versión con pies y cabeza)}

\begin{thm}(Primer teorema de la isomorfía)
	Sea $f:G_1 \to G_2$ un epimorfismo y sea $\pi:G_1 \to G_1/\ker f$. Entonces existe un isomorfismo $\overline{f} : G_1 / \ker f \to G_2$ tal que $f = \pi \circ \overline{f}$.
\end{thm}

\begin{figure}[h]
	%TODO hacer
	\centering
	\begin{tikzpicture}
	\node (g1) at (0,0) {$G_1$};
	\node (g2) at (4,0) {$G_2$};
	\node (gh) at (0, -3) {$G_1/\ker f$};
	
	\draw[-{Latex[length=2mm]}] (g1) -- (g2) node[pos=.5, above] {$f$};
	\draw[-{Latex[length=2mm]}] (g1) -- (gh) node[pos=.5, left]{$\pi$};
	\draw[-{Latex[length=2mm]}] (gh) -- (g2) node[pos=.5, below] {$\overline{f}$};
	\end{tikzpicture}
	\caption{Primer teorema de la isomorfía.}
	\label{fig:tmisomorfia1}
\end{figure}

\begin{thm}(Segundo teorema de la isomorfía)
	Sea $G$ un grupo, $H \normsub G,\ K \normsub G$ y $H < K$. Entonces $K/H$ es un subgrupo normal de $G/H$ y
	\begin{align}
		\sfrac{G/H}{K/H} \isom G/K
	\end{align}
\end{thm}

\begin{thm}[Tercer teorema de la isomorfía]
	Sea $G$ un grupo, $H < G,\ K \normsub G$. Entonces $HK < G$, $K \normsub HK$ y $H\cap K \normsub H$. Además,
	\begin{align}
		\sfrac{HK}{K} \isom \sfrac{H}{(H \cap K)}
	\end{align}
\end{thm}


% !TeX root = ../apuntes-ea.tex

\chapter{Consideraciones adicionales}

Este capítulo incluye más teoría que integra varios conceptos de los capítulos anteriores.

\section{Producto libre de grupos}

\begin{dfn}[Producto libre de grupos]
	\label{dfn:productolibre}
	Sean $S,T$ subconjuntos del grupo $G$. Definimos $ST = \{s\ast t \mid s \in S \land t \in T\}$.
\end{dfn}

Es importante remarcar el \textbf{el producto libre de [sub]grupos no siempre es un grupo. En general solo es un conjunto.} Ver el teorema \ref{thm:condicionproductolibre}

Observemos que la función $f: S \times T \to ST,\ (s,t) \mapsto st$ no es un homomorfismo de grupos. Esto es porque al operar dos elementos de $S \times T$ no se comporta bien. Sean $s,s'\in S, t,t'\in T$
\begin{align*}
(s,t) \mapsto st \\
(s',t') \mapsto s't' \\
\end{align*}
esperamos que 
\begin{align*}
f((s,t)(s',t')) = f(st, s't') \mapsto f(s,t)f(s',t') = sts't'
\end{align*}
pero en realidad ocurre que
\begin{align*}
f((s,t),(s',t')) \mapsto ss'tt' \neq f(s,t)f(s',t')
\end{align*}

No obstante, aunque la función que lleva $H_1 \times H_2 \to H_1 H_2$ no sea un homomorfismo, sí podemos saber cuantos elementos tiene $H_1H_2$.

\begin{thm}[Cardinalidad del producto libre]
	\label{thm:cardinalidadproductolibre}
	Sean $H_1, H_2 < G$ con $G$ finito. Entonces
	\begin{align}
	|H_1H_2| = \frac{|H_1||H_2|}{|H_1 \cap H_2|}
	\end{align}
\end{thm}

\begin{proof}
	Utilizaremos la función $f:H_1 \times H_2 \to H_1 H_2$ que es sobreyectiva por definición de $H_1 H_2$. Para una función sobreyectiva $f: A \to B,\ |A| = \sum_{b \in B} |f^{-1}(b)|$.
	
	%TODO argumentar lo del alpha
	
	Sean las fibras los conjuntos $f^{-1}(h_1h_2)$ de los pares de elementos que van a parar al mismo $h_1h_2 \in H_1 H_2$. La condición necesaria y suficiente para que $(h_1', h_2')$ esté en la misma fibra que $(h_1, h_2)$ es que $h_1' = h_1 \alpha \land h_2' = h_2 \alpha,\ \alpha \in H_1 \cap H_2$. Entonces $|f^{-1}(h_1, h_2)| = | (h_1 \alpha, h_2\alpha),\ \alpha \in H_1\cap H_2| = |H_1 \cap H_2| \implies |H_1||H_2| = |H_1 H_2| |H_1 \cap H_2|$ 
\end{proof}

\begin{thm}
	\label{thm:condicionproductolibre}
	Sean $H_1, H_2$ subgrupos de $G$, con $G$ finito. Si $H_2 \normsub G$ entonces $H_1 H_2 < G$ (si uno de los subgrupos es normal, entonces el producto es subgrupo).
\end{thm}

\begin{proof}
	Observamos que podemos escribir $H_1H_2 = \bigcap_{h \in H_1} h \ast H_2$. Como $H_2 \normsub G,\ h\ast H_2 \cdot h' H_2 = h h' H_2\ \forall h \in H_1$. Si nos fijamos $H_1 H_2$ es cerrado por la operación pues $h h' H_2 \in H_1H_2$ y como $G$ es finito y por tanto $H_1, H_2$ también, $H_1H_2$ es un subgrupo.	
\end{proof}

\begin{thm}
	Si $H_1 \normsub G \land H_2 \normsub G \implies H_1 H_2 \normsub G$ (si los dos subgrupos son normales, enotnces el producto también es normal).
\end{thm}

\begin{proof}
	$H_1,H_2 < G$ luego $\forall g \in G,\ gH_1H_2g^{-1} = gH_1g^{-1}gHg^{-1}  = H_1 H_2 $.
\end{proof}

\section{Grupos cíclicos}

\begin{thm}
	Todo subgrupo de $\ZnZ$ es cíclico.
\end{thm}

\begin{proof}
	La propiedad de cíclico se hereda de $\Z$ y se prueba igual utilizando el algoritmo de la división. %TODO probarlo
\end{proof}

\begin{thm}
	Consideramos $\ZnZ$ Para cada divisor $d$ de $n$, existe un único subgrupo cíclico de orden $d$.
\end{thm}

\begin{proof}
	% TODO añadir teoremas de prácticas previos a Lagrange
	$d \divides n \implies n = dn' \implies n'\Z < n\Z$ Además, por el teorema de prácticas, $|n'\Z| = d$ y por tanto $|f(n'\Z)| = d$ donde $f:n\Z \to \ZnZ$ es la relación de equivalencia habitual.
\end{proof}

\begin{thm}
	Sean $\overline{k}, \overline{k'} \in \ZnZ$. Entonces $o(\overline{k}) = o(\overline{k'}) = d \implies \langle \overline{k} \rangle = \langle \overline{k'} \rangle$ 
\end{thm}

\begin{thm}
	Sean $n, m \in \N$. El grupo producto directo $\ZnZ \times \ZmZ$ es cíclico $\iff mcd(n,m) = 1$.
\end{thm}

\begin{proof}
	Para que $\ZnZ \times \ZmZ$ sea cíclico debe haber un elemento $a \in \ZnZ \times \ZmZ \mid o(a) = m\cdot n$. Si $m$ y $n$ no son coprimos entonces el orden de $a$ no puede ser $m\cdot n$. %TODO pensar y explicar
\end{proof}

\begin{thm}
	\label{thm:noprobado1}
	Si $G$ es abeliano y $|G| < \infty$ entonces $G$ es un producto de grupos cíclicos finitos.
\end{thm}

\begin{proof}
	Dice que no lo vamos a probar, pero veremos algunos resultados más adelante (en la sección sobre clasificación de grupos finitos \ref{gruposfinitosnotables}).
\end{proof}

\part{Segundo parcial - hojas 2, 3 y 4}

% !TeX root = ../apuntes-ea.tex

\chapter{El teorema de Cauchy}

\section{Consideraciones previas}

\subsection{Centro de un grupo y sus propiedades}

\begin{dfn}[Centro de un grupo]
	Sea $G$ un grupo finito. Definimos el centro de $G$, $Z(G) = \{a \in G \mid \forall g \in G,\ ag = ga\}$.
\end{dfn}

El centro es útil en grupos finitos no abelianos.

\begin{pro}
	Sean $a, b \in Z(G)$. Entonces $ab \in Z(G)$.
\end{pro}

\begin{proof}
	Tenemos que $ag = ga$ y que $bg = gb$. Ahora tenemos que probar que $g(ab) = (ab)g$. Es trivial manipulando $(ab)g = agb = gab$.
\end{proof}

\begin{pro}
	Sea $G$ un grupo. $Z(G)$ es un subgrupo y además es un subgrupo normal.
\end{pro}

\begin{proof}
	$\forall g \in G,\ Z(G)g = \{ag \mid a \in G \land \forall b \in G,\ ab = ba\} = \{ga \mid a \in G \land \forall b \in G,\ ab = ba\} = gZ(G)$.
\end{proof}

\begin{pro}
	Si $H < Z(G)$ entonces $H$ es abeliano y normal.
\end{pro}

\begin{pro}
	Sea $g \in G,\ \phi_g: G \to G$ el isomorfismo definido por $\phi_g(x) = gx\inv{g}$. Entonces
	\begin{align*}
	x \in Z(G) &\iff \forall g \in G, gx = xg \iff gx\inv{g} = x \\
	x \in Z(G) &\iff \forall g \in G,\ \phi_g(x) = x 
	\end{align*}
\end{pro}

\begin{pro}
	$G$ es abeliano $\iff G = Z(G)$
\end{pro}

Sea $a \in G \land o(a) = n$. Si $a$ es el único elemento de orden $n$ entonces $n = 2 \land a \in Z(G)$. Probamos primero que $n=2$. Si $a$ es el único elemento de orden $n$ entonces tiene que ocurrir que $a$ y $a^{n-1}$ tienen el mismo orden por lo que $1 = n-1 \implies n = 2$.

\begin{pro}
	\label{pro:triplecentro}
	Si $G/Z(G)$ es cíclico de orden $n$ entonces $n = 1$. Otra manera de formularlo: Si $G/Z(G)$ es cíclico, entonces $G = Z(G)$. Otra manera más de formularlo: si $G/Z(G)$ es cíclico entonces $G$ es abeliano.
\end{pro}

\begin{proof}
	Supongamos que $G/Z(G) \isom \ZnZ$. Vamos a probar que $n$ tiene que ser 1. Supongmos que $G/Z(G) = \{\overline{\alpha_i}, i = 1, \dots, n\}$ donde $\overline{\alpha_i} = \alpha^i Z(G)$. Fijamos $g \in G$ con $g = \alpha^j h,\ h \in Z(G),\ 0 \leq j < n$ y fijamos $f' \in G$ con $g' = {\alpha^j}' h',\ h' \in Z(G),\ 0 \leq j' < n$. Entonces $gg' = \alpha^j h{\alpha^j}' h' = \alpha^{j+j'}hh' = {\alpha^j}' h' \alpha^j h = gg'$ (podemos conmutar las $h$ con cualquier elemento porque $h \in Z(G)$, por el contrario, los $\alpha$ no necesitamos conmutarlos, solo agruparlos cuando están juntos). Es decir, que $\forall g, g' \in G$ tenemos que $gg' = g'g$ por lo que $G$ es abeliano.
\end{proof}


\begin{ej}[Hoja 1, ej 33]
	Sea $G$ un grupo. Suponed que existe un único $a \in G$ de orden 2. Demostrad que $a \in Z(G)$.
\end{ej}

\begin{proof}
	Recordamos que $a \in Z(G) \iff ga = ag,\ \forall g \in G$. Definimos el isomorfismo de conjugación $\phi_g (x) = gx\inv{g}$ para algún $g$. Como $\phi_g$ es isomorfismo lleva elementos de orden $n$ en elementos de orden $n$. Entonces $\phi_g(a) = a$ ya que $a$ es el único elemento de orden 2. Por tanto $ga\inv{g} = a \implies ga = ag \implies a \in Z(G)$.
\end{proof}

\subsection{Centralizador de un elemento}

\begin{dfn}
	[Grupo de automorfismos]
	Sea $G$ un grupo. Llamamos grupo de automorfismos al grupo
	\begin{align}
	\autom{G} = \{f \mid f: G \to G \text{ isomorfismo}\}
	\end{align}
\end{dfn}

\begin{pro}
	La función $\gamma: G \to \autom{G}$ definida con $\gamma(g) \mapsto \gamma_g$, donde $\gamma_g : G \to G, \gamma_g(x) = gx\inv{g}$, es un homomorfismo.
\end{pro}

\begin{proof}
	Verifica la definición: para $g,g' \in G$
\end{proof}

\begin{dfn}[Elementos conjugados]
	\label{dfn:elementosconjugados}
	Sean $a,b \in G$. Decimos que $a$ y $b$ son conjugados $\iff \exists g \in G \mid \gamma_g(a) = b$.
\end{dfn}

\textbf{Nota:} La relación de conjugación solo merece la pena en grupos no abelianos, porque en un grupo abeliano, cualquier par de elementos es conjugado.

\begin{ej}
	En $S_3$ afirmamos lo siguiente:
	\begin{itemize}
		\item que $1$ solo tiene como conjugado a sí mismo,
		\item que $\{(12),(13),(23)\}$ son conjugados entre sí,
		\item y que $\{(123),(132)\}$ también son conjugados entre sí.
	\end{itemize}
	Es decir, que la conjugación nos genera una partición con 3 cajas disjuntas.
\end{ej}

\begin{pro}
	La relación de conjugación es una relación de equivalencia $aRb \iff a$ y $b$ son conjugados.
\end{pro}

\begin{proof}
	Comprobamos que $R$ es una relación de equivalencia:
	\begin{enumerate}
		\item Reflexiva: $\forall a \in R, aRa$: tomamos $g = e$ y automáticamente tenemos que $ea\inv{e} = a$.
		\item Simétrica: $\forall a,b \in R,\ aRb \implies bRa$: $\exists g, g a \inv{g} = b$. Tomamos $\gamma_{\inv{g}}$ y tenemos que $\gamma_{\inv{g}}(b) = a \implies bRa$.
		\item Transitiva: $\forall a,b,c \in G,\ aRb \land bRc \implies aRc$. Por hipótesis tenemos que $\exists g \in G \mid \gamma_g(a) = b \land \exists g' \in G \mid \gamma_{g'}(b) = c$. Por tanto $\gamma_{gg'}(a) = (\gamma_{g'} \gamma_g)(a) = \gamma_{g'}(b) = c$.
	\end{enumerate}
\end{proof}



En esta relación de equivalencia, las clases de equivalencia son de la forma $\overline{a} = \{ga\inv{g} \mid g \in G\}$ (conjuntos de los elementos que son conjugados de $a$). Queremos saber cuántos elementos hay en cada clase de equivalencia.

Fijamos $a \in G$ y definimos

\begin{dfn}[Centralizador de un elemento]
	\label{dfn:centralizador}
	Sea $a \in G$. Llamamos centralizador de $a$ al conjunto
	\begin{align}
	C(a) = \{g \in G \mid \gamma_g(a) = g a \inv{g} = a\}
	\end{align}
	Se tiene que $\forall a \in G,\ e \in C(a)$, es decir que $C(a)$ no es vacío.
\end{dfn}

\begin{pro}
	$C(a)$ es un subgrupo de $G$
\end{pro}

\begin{proof}
	Por el teorema \ref{thm:subconjuntocerrado} solo necesitamos probar la clausura, es decir, tenemos que probar que $\forall g,g' \in G,\ g \in C(a) \land g' \in C(a) \implies gg' \in C(a)$. Sale solo $(gg')a\inv{gg'} = gg'a\inv{(g')}\inv{g} = ga\inv{g} = a \in C(a)$.
\end{proof}

\begin{pro}
	\label{pro:cardinalcajas}
	$|\{ga\inv{g} \mid g \in G\}| = [G:C(a)] = r$ (el número de elementos de una clase de equivalencia es el índice de un representante)
\end{pro}

\begin{proof}
	Fijamos $a \in G$ y definimos $H = C(a) = \{g \in G \mid ga\inv{g} = a\}$.
\end{proof}



\section{Teorema de Cauchy}

\begin{thm}[de Cauchy]
	Sea $G$ un grupo finito con $|G| = n$. Si $p$ es primo y $p\divides n$ entonces $G$ contiene un elemento de orden $p$.
\end{thm}

\begin{proof}
	Procedemos por casos:
	\begin{itemize}
		\item Si $G$ es abeliano. Descomponemos $|G| = n = p_1^{\alpha_1}p_2^{\alpha_2}\dots p_s^{\alpha_s}$. Por el teorema \ref{thm:noprobado1}, $G \isom \Z/p_1^{\beta_1}\Z \times \Z/p_2^{\beta_2}\Z \times \dots \times \Z/p_s^{\beta_r}\Z$ donde cada $\alpha_i$ es la suma de algunos $\beta_r \qed$.
		
		\item Si $G$ no es abeliano. Particionamos $G$ con la relación de equivalencia dada anteriormente (definición \ref{dfn:elementosconjugados}), $aRb \iff \exists g \in G \mid ga\inv{g} = b$. Recordemos que cada clase de equivalencia es de la forma $\overline{c} = \{gc\inv{g} \mid g \in G\}$. Observamos que si partimos de $e$, el elemento neutro, $eRb \implies \exists g \mid ge\inv{g} = b$ pero $\forall g \in G,\ ge\inv{g} = e$ por lo que $\overline{e}$ tiene un único elemento.
		
		Tomemos ahora una clase de equivalencia, la que contenga a $a \in G$. La clase es $\overline{a} = \{ga\inv{g} \mid g \in G\}$. Es claro que $a \in \overline{a}$ por la propiedad reflexiva de $R$, luego por lo menos en $\overline{a}$ tiene un elemento.
		
		\begin{align*}
		\overline{a} = \{ga\inv{g} \mid g \in G\} = \{a\} &\iff ga\inv{g} = a,\ \forall g \in G \\
		&\iff ga = ag,\ \forall g \in G
		\end{align*}
		\begin{align*}
		|\overline{a}| = 1 &\iff \overline{a} = 1 \\
		&\iff a \in Z(G)
		\end{align*}
		
		Supongamos que la partición está dada por subconjuntos $\overline{a_1}, \overline{a_2}, \dots, \overline{a_s}$. Por ser una partición, cualquier elemento vive en una sola caja, luego para saber cuantos elementos tiene $G$ nos vale con sumar los elementos de cada caja:
		\begin{align*}
		|G| = \sum_{i = 1}^{s} |\overline{a_i}| = \sum_{i = 1}^n |\{ga_i\inv{g} \mid g \in G\}|
		\end{align*}
		Ahora bien, por la proposición \ref{pro:cardinalcajas} tenemos que $|\overline{a_i}| = [G:C(a_i)]$. Por tanto decir que $|\overline{a_i}| = 1 \implies [G:C(a_i)] = 1 \implies G = C(a_i)$.
		
		Ahora vamos a dividir el sumatorio en dos: por un lado las cajas de un solo elemento y luego las cajas de varios elementos:
		\begin{align}
		\label{eq:thmcauchy}
		|G| = |Z(G)| + \sum_{i = r + 1}^{s} [G : C(a_i)] \text{ donde } |Z(G)| = r \text{ y } [G : C(a_i)] \geq 2, \forall i = r+1,\dots, s
		\end{align}
		
		Ahora para probar el teorema de Cauchy procedemos por inducción en $n = |G| = [G:C(a_i)]\cdot |C(a_i)|$.
		
		\begin{enumerate}
			\item Caso $n = 1$. $G = \{e\}$ que es obvio.
			\item Caso $n \implies n+1$. Pueden pasar dos cosas:
			\begin{itemize}
				\item o bien $p \divides |C(a_i)|$ para algún $i = r+1, \dots, s$ entonces, por hipótesis inductiva, $C(a_i)$ contiene algún elemento de orden $p$. Ahora bien, $C(a_i) < G \implies G \implies $ el elemento también está en $G$. Podemos proceder por inducción y todo es genial \qedsymbol
				
				\item o bien $p \not\divides |C(a_i)|,\ \forall i = r+1,\dots,s$. No podemos proceder por inducción. En este caso $[G:C(a_i)]\cdot |C(a_i)| = |G| \implies p \divides [G: C(a_i)],\ i = r+1,\dots, s$. No
				
				Como $|G| = |Z(G)| + \sum_{i = r + 1}^{s} [G : C(a_i)]$ y por hipótesis $p \divides |G| \land p \divides [G : C(a_i)], \forall i = r+1,\dots,s \implies p \divides |Z(G)| \implies |Z(G)|$ es múltiplo de $p$. Como $Z(G)$ es abeliano, $\exists \alpha \in Z(G) \mid o(\alpha) = p$. Luego se reduce al caso abeliano y ya estaría \qedhere
			\end{itemize}
		\end{enumerate}
	\end{itemize}
\end{proof}

\begin{ej}
	Sea $G$ tal que $|G| = pq$. Entonces por le teorema de Cauchy $\exists a,b \in G \mid o(a) = p \land o(b) = q$. Como $p$ y $q$ son primos los ordenes de $\langle a \rangle$ y $\langle b \rangle$ son coprimos y por tanto $\langle a \rangle \cap \langle b \rangle = \{e\}$. Por el teorema del orden de conjunto\footnote{No sabemos si alguno es normal, luego no tenemos garantías de que el producto sea un grupo} producto libre (\ref{thm:cardinalidadproductolibre}), $|\langle a \rangle \langle b \rangle| = pq$. Lo que si que sabemos es que $G = \{a^ib^j \mid 0 \leq i < p -1 \land 0 \leq j < q - 1\} = \langle a, b \rangle$.
\end{ej}

\begin{ej}
	Sea $G$ tal que $|G| = 2q$. Análogamente al caso anterior llegamos a que $o(a) = 2$. Como $\langle b \rangle$ tiene índice 2 entonces $\langle b \rangle \normsub G$. Esto nos permite saber como operar con las palabras $a^ib^j$ una vez tenemos un isomorfismo que lleva $a b \inv{a} = b^j$ (tiene que ir a algún $b^j$ porque por ser isomorfismo tiene que llevar elementos de orden $q$ en elementos de orden $q$: los $b \in \langle b \rangle$)
\end{ej}

Dada la relación de equivalencia de conjugación (definición \ref{dfn:elementosconjugados}), definimos $C$ como el conjunto de los representantes de las clases de equivalencia. Entonces podemos decir
\begin{align*}
G = \bigcup_{c_i \in C} \{a \in G \mid a R c_i\}
\end{align*}
Observemos que $d \in Z(G) \iff \{a \in G \mid a R d\} = \{gd\inv{g} \mid g \in G\} = \{d\}$. Y por tanto podemos escribir
\begin{align*}
C = Z(G) \cup (C\setminus Z(G))
\end{align*}
que aunque pareza obvio quiere decir que $C$ se puede expresar como la unión disjunta de las cajas que tienen solo un elemento que se corresponden con elementos que están en el centro y las cajas que tienen más de uno. Y por lo visto en la demostración del teorema de Cauchy tenemos que
\begin{align*}
|G| = \sum_{c_i \in C} | \overline{c_i} | = |Z(G)| + \sum_{i = r + 1}^{s} [G : C(a_i)] \text{ donde } [G : C(a_i)] \geq 2
\end{align*}

\section{P-grupos}

\begin{dfn}[P-grupo]
	Sea $p$ primo. Decimos que $G$ es un p-grupo si $|G| = p^r$.
\end{dfn}

Nos interesan sobre todo los p-grupos no abelianos

\begin{thm}
	Si $G$ es un p-grupo entonces $Z(G)$ es no trivial (no es el vacío).
\end{thm}

\begin{proof}
	Podemos escribir sin distinguir entre cajas de uno o varios elementos
	\begin{align*}
	|G| = |C(c_i)||[G:C(c_i)]|
	\end{align*}
	es decir que tenemos una factorización de $|G| = p^r$ luego $|C(c_i)|$ y $|[G:C(c_i)]|$ son ambos potencias de $p$. Y aplicando esto a la expresión \ref{eq:thmcauchy} tenemos que
	\begin{align*}
	\underbrace{|G|}_{\text{múltiplo de p}} = |Z(G)| + \sum_{i = r + 1}^{s} \underbrace{[G : C(a_i)]}_{\text{múltiplo de p}} \text{ donde } [G : C(a_i)] \geq 2
	\end{align*}
	por lo que $|Z(g)|$ tiene que ser múltiplo $p$ por lo que $Z(G)$ no puede ser el trivial.
\end{proof}

\begin{ej}
	Tenemos que $Z(D_4) = \{1,B^2\}$ y $Z(H) = \{1, B^2\}$ donde $H$ es el grupo de cuateriones (ejemplo \ref{ej:grupocuaterniones}) y $D_4$ es el famoso grupo (ejemplo \ref{ej:famosogrupod4}). 
\end{ej}

\begin{thm}
	Si $p$ es primo y $|G| = p^2$ entonces $G$ es abeliano.
\end{thm}

\begin{proof}
	Por el la demostración del teorema anterior tenemos que o bien $|Z(G)| = p$ o bien $|Z(G)| = p^2$. Afirmamos que $|Z(G)| \neq p$ ya que si fuera así $|G/Z(G)| = p \implies G/Z(G)$ cíclico pero hemos probado (proposición \ref{pro:triplecentro}) que $G/Z(G)$ no puede ser cíclico. Por lo tanto $|Z(G)| = p^2 \implies Z(G) = G \implies G$ es abeliano.
\end{proof}

% ---------------- después del parcial 1
% 20181009

\hr

Sea $\sim$ una relación de equivalencia definida por $a\sim b \iff \exists g \in G \mid ga\inv{g} = b$ para $a,b \in G$. Esta relación da una partición de $G$ en clases de la forma $cl(a) = \{ga\inv{g} \mid g \in G\}$. En el caso abeliano esta relación es la de igualdad, por lo que no nos merece la pena liar este pifostio para saber que $a\sim b \iff a = b$. 

Es muy importante saber cómo contamos los elementos de una clase, es decir, de cuantas formas podemos \textit{mover} el elemento $a$ con $g \in G$. Para ello definimos el centralizador (definición \ref{dfn:centralizador}) como $C(a) = \{h \in G \mid ha\inv{h} = a\} < G$. Queremos probar que $|cl(a)| = [G:C(a)] = r$.

Lo probamos tomando clases laterales a la izquierda (por ejemplo) y partiendo $G$ en $r$ cajas. Las cajas son de la forma $\alpha_iC(a),\ i = 1, \dots, r$. Esta partición no tiene que ver con la partición anterior. Observemos que para cualquier $g \in \alpha_i C(a), g = \alpha_i h$, tenemos que $g a \inv{g} = \alpha_i h a \inv{h} \inv{\alpha_i} = \alpha_i a \inv{\alpha_i}$ es decir que los $g \in C(a)$ no se mueven fuera de la caja. Es decir, que si $\alpha_i \neq \alpha_j$ para $i\neq j$ entonces hay $r$ maneras de mover a $g$ y por tanto $|cl(a)| = r$.

Probaremos que en efecto los $\alpha_i$ son distintos.

Sean $g_1, g_2 \in G$. $g_1a\inv{g_1} = g_2a\inv{g_2} \iff (\inv{g_2}g_1)a(\inv{g_1}g_2) = a \iff (\inv{g_2}g_1)a\inv{(\inv{g_2}g_1)} \iff C(a) \inv{g_2}g_1 \in C(a) \iff g_1 \in g_2C(a)$.

Si $G/\sim$ tiene $N$ elementos, tomamos $\{c_1, \dots, c_N\}$ como el conjunto de los representantes, donde $c_i$ es un representante de cada conjunto de la partición. Entonces pordemos expresar
\begin{align*}
G = \bigcup_{c_i \in C} = cl(c_i)
\end{align*}
donde $|cl(c_i)| = [G:C(c_i)]$. Por tanto decir que $|cl(c_i)| = 1$ es equivalente ($\iff$) a decir que $G = C(c_i) = \{\forall g \in G,\ gc\inv{g} = c\} \iff c \in Z(G)$.

Afirmábamos que
\begin{align*}
|G| = \sum_{c_i \in C} |cl(c_i)| = |Z(G)| + \sum_{c_i \in C\setminus Z(G)} [G:C(c_i)]
\end{align*}
descomponiendo la suma en las clases con solo un elemento y las clases con más de dos elementos.

\hr

\begin{ej}
	Consideramos $D_3$ (ver ejemplo \ref{ej:grupod3}). Nos fijamos en que $B \not\in Z(D_3)$ es decir que en $cl(B)$ hay más de un elemento. En particular por lo visto anteriormente $|cl(B)| = [G:C(B)]$. Ahora bien $C(B) = \{1, B, B^2\}$ luego $|cl(B)| = [G:C(B)] = 2$. La pregunta es ¿quién es el compañero de $B$ en su clase? Es fácil, recordamos que $\phi_g (x) = gx\inv{g}$ (el isomorfismo conjugación) es un isomorfismo y que $\{1, B, B^2\}$ es normal, por lo que $o(B) = o(\phi_g(B)) = 2$. Entonces $\phi_g(B) \neq 1$ porque no coinciden los órdenes, de manera que $\phi_g(B) = B^2$ por necesidad. Luego el otro elemento es el $B^2$.
	
	¿Qué pasa con el elemento $A$? Pues ocurre que $A \in C(A)$ y $\{1, A\} \in C(A)$. me faltan cosaaaasss
	
	Para conlcuir queda que la relación $\sim$ parte $D_3$ en 3 cajas, a saber:
	\begin{align*}
	D_3 = \{\underbrace{1}, \underbrace{B, B^2}, \underbrace{A, AB, AB^2}\}
	\end{align*}
\end{ej}


\begin{ej}
	\label{ej:clasesd4}
	El caso del famoso grupo $D_4$ (ver ejemplo \ref{ej:famosogrupod4})es mucho más interesante porque $Z(D_4)$ no es trivial. Elegimos por ejemplo el elemento $B^2$. Probar que $\phi_g(B^2) = gB^2\inv{g} = B^2,\ \forall g \in D_4$ es complicado. Pero fijémonos en que $\phi_B(B^2) = BB^2\inv{B} = B^2$ y que $\phi_A(B^2) = AB^2\inv{A} = B^2$. Entonces cualquier palabra en $A$ y en $B$ no mueve a $B^2$, por ejemplo $AB(B^2)\inv{B}\inv{A} = B^2$. Nos convencemos de que $B^2 \in Z(D_4)$. Con esto ya tenemos que $|Z(D_4)| \geq 2$ (puesto que de momento ya sabemos que $1, B^2 \in Z(G)$. Podría ser entonces $|Z(D_4)| = 4, 8$ (probamos los divisores de $|D_4|$). Como $D_4$ no es abeliano, es claro que $|Z(D_4) \neq 8$. Tampoco puede ser $|Z(D_4) \neq 4$ porque si tuviera 4, el cociente $D_4/Z(G)$ tendría orden $2$ y por tanto sería cíclico. Pero ya hemos probado que $G/Z(G)$ no puede ser cíclico (ver proposición \ref{pro:triplecentro}). Luego ya sabemos que $Z(D_4) = \{1, B^2\}$.
	
	Vamos a seguir sacando cajas. Veamos $cl(B)$. Claramente $B \in C(B)$ y por alguna razón que me falta $C(B) = \{1, B, B^2, B^3\}$. Por la fórmula tenemos que $|cl(B)| = [D_4:C(B)] = 2$. Tenemos una vez más que utilizar el isomorfismo de conjugación. Sabemos que $cl(B) = \{ga\inv{g} \mid g \in G\}$. Pero al ser $\phi_g$ isomorfismo y $\langle B \rangle$ normal, tenemos que $\phi_g : \langle b \rangle \to \langle b \rangle$ también es isomorfismo y por tanto lleva elementos de orden $n$ en elementos de orden $n$. Por tanto $\phi_g(B) = gB\inv{g}$ solo puede ser $B^3$ (a parte de $B$). Luego ya tenemos que $cl(B) = \{B, B^3\}$.
	
	¿Qué pasa con $A$? Pues es claro que $C(A) \supset \{1, A, B^2, AB^2\}$ ya que $B^2 \in Z(G)$ por lo que está en todos los $C(c_i)$.
\end{ej}


\hr

% 20181011

Vez pasada tomábamos $a \in G$ y teníamos $cl(a) = \{g a \inv{g} \mid g \in G\} = \{a=a_1, a_2, \dots, a_r\}$ y $C(a) = \{g \in G \mid ha\inv{h} = a \}$. Concluíamos que $|cl(a)| = [G:C(a)]$.

Vamos a generalizar al caso $S \subset G,\ S \neq \emptyset$. Consideramos la familia de subconjuntos siguiente:
\begin{align*}
\{gS\inv{g} \mid g \in G\} = \{S = S_1, S_2, \dots, S_r\}
\end{align*}
que tiene $r$ subconjuntos distintos.

Recordemos que la conjugación dada $\phi_g(x) = gx\inv{g}$ (el isomorfismo conjugación) es un isomorfismo\footnote{A veces tomate frito llama a este isomorfismo $\gamma_g$}, y por tanto una biyección entre subconjuntos $S_i \subset G$. Por tanto $|S| = \phi_g(S)$.

\begin{dfn}[Normalizador de un subgrupo]
	\label{dfn:normalizador}
	Fijado $S \subset G$, definimos el normalizador de $S$:
	\begin{align}
	N(S) = \{h \in G \mid hS\inv{g} = S\}
	\end{align} 
\end{dfn}

Se parece mucho a la definición de centralizador de un elemento (\ref{dfn:centralizador}). En el caso en que $S = \{a\}$ tenemos que $N(S) = \{h \in G \mid ha\inv{h} = a\} = C(a)$.

Ojo, decir que $hS\inv{h} = S$ no significa que $\forall b_i \in S,\ hb_i\inv{h} = b_i$, sino que $hb_i\inv{h} \in S$ (no mandamos cada elemento a él mismo, sino que todos quedan dentro del subconjunto). Es decir que \textit{$N(S)$ es el conjunto de la totalidad de elementos para los que $\phi_g$ manda el subconjunto $S$ en sí mismo.}

\begin{pro}
	Dado $S \subset G,\ N(S)$ es un subgrupo.
\end{pro}

\begin{proof}$ $\newline
	Como $G$ es finito, $N(S)$ es subgrupo $\iff S \neq \emptyset \land N(S)$ es cerrado por la operación.
	\begin{itemize}
		\item Es claro que $e \in N(S)$ pues $eS\inv{e} = S$, luego $N(S) \neq \emptyset$.
		\item Tenemos que probar la clausura. Si $h_1S\inv{h_1} = S \land h_2S\inv{h_2} = S$ tenemos que $\underbrace{(h_2S\inv{h_2})}_{\in S}\inv{h_1} = S \implies h_1h_2 \in N(S)$.
	\end{itemize}
\end{proof}

\begin{pro}
	\label{pro:propiedad2Ns}
	$\{gS\inv{g} \mid g \in G\} = \{S = S_1, S_2, \dots, S_r\}$ son $r$ subconjuntos distintos. Es decir que $r = [G: N(S)]$.
\end{pro}

\begin{proof}
	A la izquierda del lector.\footnote{Left to the reader.}
\end{proof}

Supongamos ahora que en vez de ser $S \subset G$, tomamos $S < G$. Recordemos que dado $g\in G$, $\phi_g$ es un isomorfismo por tanto manda elementos de un subgrupo en otro subgrupo (si el subgrupo es normal, manda elementos de un subgrupo en sí mismo).

\begin{pro}
	$H \subset N(H)$
\end{pro}

\begin{proof}
	Si tomamos $h \in G$, tenemos que $hH\inv{g} = H$ y también $\inv{h}H\inv{(\inv{h})} = H$ (todo elemento de $H$ tambéin tiene a su inverso en $H$).
\end{proof}

\begin{thm}
	Sea $G$ grupo, $H < G$. Entonces $H \normsub N(H)$ y $N(H)$ es el mayor subgrupo de $G$ con esta propiedad, es decir, $H \normsub H' \implies H' < N(H)$.
\end{thm}

\begin{proof}$ $\newline
	\begin{itemize}
		\item Para probar que $N\normsub N(H)$ tiene sentido olivdarse del grupo $G$. Tenemos que $h \in N(H) \iff hH\inv{h} = H, \forall h \in G$. En particular, tenemos que $hH\inv{h} = H,\ \forall h \in N(H) \implies H$ es normal en $N(H)$.
		
		\item Para porbar que $N(H)$ es el mayor subgrupo con esta propiedad demostraremos que si $H < H'$ y $H \normsub H'$ entonces $H' \subseteq N(H)$. La demostración es casi una tautología. Tenemos que $\forall h' \in H',\ h'H\inv{h'} = H \implies \forall h' \in H',\ h' \in N(H) \implies H' \subset N(H)$.
	\end{itemize}
\end{proof}

\begin{cor}
	$H \normsub G \iff N(H) = G$
\end{cor}

\begin{proof}
	Sabemos que $H\normsub H = \{gH\inv{g} \mid g \in G\}$ y dicho conjunto tiene $[G:N(H)] = 1$ elementos, luego $N(H) = G$. En otras palabras, el normalizador de un subgrupo $H < G$ normal es todo el grupo $G$.
\end{proof}

\begin{pro}
	$Z(G) < N(H)$
\end{pro}

\begin{proof}
	Por definición de $Z(G)$ tenemos que los elementos $g \in Z(G)$ fijan no solo los elementos dentro de subconjuntos, sino que los fijan uno a uno. Por lo que es claro que $Z(G) < N(H)$. 
\end{proof}

\begin{ej}
	Vamos a empezar por $G = S_3$. En $S_3$ tenemos los subgrupos $\langle (12) \rangle, \langle (13) \rangle, \langle (23) \rangle$ de orden 2 y el subgrupo $\langle (123) \rangle = \{(1), (123), (132)\}$ de orden 3. %TODO pintar retículo
	\begin{itemize}
		\item En el caso de este último $g\langle (123) \rangle \inv{g} = \langle (123) \rangle$ porque es el único subgrupo de orden 3. Por tanto $\langle (123) \rangle \normsub S_3$ y entonces $N(\langle (123) \rangle) = S_3$.
		\item Sin encambio en el caso de los subgrupos de orden $2$ es posible que $g\langle (12) \rangle \neq \langle (12) \rangle$, porque hay más de un subgrupo de orden 2. Observemos por ejemplo que $(13)(12)\inv{(13)} = (32) = (23)$, luego $\langle (12) \rangle$ no es normal en $S_3$, ya que hemos encontrado $g = (13) \in G$ que lo mueve. Pero ¿quién es el normalizador $N(\langle (12) \rangle)$? Pues ya sabemos que es un subgrupo propio, porque no puede dar todo $S_3$. Evidentemente $\langle (12) \rangle \subset N(\langle (12) \rangle)$. Luego tiene que ser que $N(\langle (12) \rangle) = \langle (12) \rangle$\footnote{No tiene gracia que $\langle (12) \rangle$ sea normal en sí mismo, lo que tiene gracia es que $\langle (12) \rangle$ es el mayor grupo donde $\langle (12) \rangle$ es normal.} 
	\end{itemize}
\end{ej}

%20181011

\begin{ej}
	Seguimos por el famoso grupo $D_4$ (presentación en el ejemplo \ref{ej:famosogrupod4}). Vimos anteriormente (ejemplo \ref{ej:clasesd4}) que $Z(D_4) = \{1, B^2\}$. Tenemos su retículo en \ref{fig:reticuloD4}. Queremos ver de entre los subgrupos de $D_4$, cuáles son los que conmutan.
	\begin{itemize}
		\item Empecemos por $\langle b \rangle = \{1, b, b^3, b^3\}$. Observamos que $\langle b \rangle$ es normal puesto que tiene índice 2, es decir que $\{g\langle B \rangle \inv{g} \mid g \in G\} = \{\langle B \rangle\}$ y tiene sentido que $[G:N(\langle B \rangle)] = 1$. Es decir que como $\langle B \rangle$ es normal tenemos que $N(\langle B \rangle) = D_4$.
		\item Seguimos por $H = \{1, A, B^2, AB^2\}$. Ocurre lo mismo, luego $N(H) = D_4$.
		\item Con el caso de $\langle B^2 \rangle$ tenemos también que $N(\langle B^2 \rangle) = D_4$ por ser normal.
		\item Agotados los subgrupos normales, nos quedan los más difíciles. Consideramos ahora $\langle A \rangle$. Una vez más nos preguntamos quién es el normalizador de $\langle A \rangle$.
		\begin{enumerate}
			\item Es claro que $\langle A \rangle$ conjugará con otros subgrupos de orden 2.
			\item También es claro que $\langle A \rangle \subset N(\langle A \rangle)$ y que $\langle B^2 \rangle \subset N(\langle A \rangle)$. Luego $N(\langle A \rangle)$ tiene al menos 2 elementos.
			\item También sabemos que $N(\langle A \rangle) \subsetneq G$ puesto que $\langle A \rangle$ no es normal, por lo que no puede tener 8 elementos. Por esto y porque $N(\langle A \rangle) < G$, concluimos que $|N(\langle A \rangle)| = 4$.
			\item ¿Cuáles mueven al $\langle A \rangle$? Sabemos que no puede haber más de dos, pues el normalizador tiene 4 elementos. Pues mirando la presentación nos damos cuenta de que $BA = A\inv{B} \iff BA\inv{B} = AB^2$. Luego nos damos cuenta de que $A$ se mueve a $AB^2$.
			\item Análogamente nos damos cuenta de que $AB$ se mueve a $AB^3$.
			\item Ya tenemos los dos elementos que se mueven.
		\end{enumerate}
	\end{itemize}
\end{ej}

\begin{ej}
	Vamos ahora con el grupo de cuaterniones $H$ descrito en el ejemplo \ref{ej:grupocuaterniones}.
	
	\begin{enumerate}
		\item Nos dibujamos el retículo.
		\item Primeramente nos damos cuenta de que $\langle A \rangle \cap \langle b \rangle \supsetneq \{e\}$ porque $H$ tiene 8 elementos y por la fórmula del producto libre \ref{thm:cardinalidadproductolibre} y porque todo producto directo de subgrupos está contenido en el grupo aunque no sea subgrupo.
		\item Ocurre lo mismo con los demás subgrupos de orden 4 ($\langle A \rangle,\ \langle AB \rangle$). Tiene que tener intersección no vacía. En concreto la intersección es el subgrupo generado $\langle A^2 = B^2 = (AB)^2 \rangle$.
		\item En $H$ todos los subgrupos son normales, por lo que no tienen "órbitas" de modo que es muy aburrido.
	\end{enumerate}
\end{ej}

\begin{ej}
	Consideramos ahora $D_5$ que funciona como el $D_4$:
	\begin{align*}
	D_5 = \{1, B, B^2, B^3, B^4, A, AB, AB^2, AB^3, AB^4\} \\
	o(B) = 5
	\end{align*}
	\begin{itemize}
		\item Primera observación. Como $o(B) = 5$ que es primo, tenemos que $o(B^k) = 5,\ k = 1, \dots, 4$. Luego cualquier subgrupo generado por $\langle B^k \rangle = \langle B \rangle$. Aquí falta algo.
		\item Observemos que los subgrupos propios pueden ser de 2 o 5 elementos.
		\item No puede haber subgrupos generados por dos elementos de $D_5$ (por qué?)
		\item Los únicos subgrupos son $\langle B \rangle$ y los generados por $A, AB, AB^2, AB^3, AB^4$.
		\item Afirmamos que $\{gA\inv{g} \mid g \in G\} = \{\langle A \rangle, \langle AB \rangle, \langle AB^2 \rangle, \langle AB^3 \rangle, \langle AB^4 \rangle \}$. Vamos a probarlo.
		
		\begin{enumerate}
			\item Primero nos damos cuenta de que $\{1, A\} \in N(\langle A \rangle)$.
			\item Además tenemos que no puede haber otro grupo por encima de $\langle A \rangle$ y $D_5$ por lo que tenemos que $N(A) = \langle A \rangle$.
			\item Por tanto en la órbita de $A$ tenemos $[D_5:\langle A \rangle] = 5$ grupos.
		\end{enumerate}
		
	\end{itemize}
\end{ej}

\hr

Sea $X$ conjunto. Consideramos
\begin{align*}
Biy(X) = \{f \mid f: X \to X \text{ biyección}\}
\end{align*}
En el caso en que $|X| = n$, por ejemplo $X = \{1, 2, 3, \dots, n\}$ tenemos que $Biy(X) = S_n$. Como $f:X \to X$ si $f$ es inyectiva entonces automáticamente es sobre y por tanto biyectiva.

En general, tiene sentido pensar en $Biy(X)$ aunque $|X| = \infty$. Además, en dicho conjunto viven la biyección identidad y la biyección inversa para cada biyección. Por tanto, tiene sentido pensar en $(Biy(X), \circ)$ como un grupo (la composición de biyecciones da una biyección).

Nos concentramos en el caso en el que $|X| = n$ que nos da $Biy(X) = S_n$. Ya hemos visto que $S_2 = \{1, \sigma\} \implies |S_2| = 2$ y para $S_3$ tenemos $|S_3| = 3!$ y en general $|S_n| = n!$.

Fijamos un conjunto $X$ y un homomorfismo de grupos $\alpha: X \to Biy(X)$. A partir de estos datos definimos una relación de equivalencia que nos da una partición de $X$, es decir, vamos a partir $X$ en conjuntos disjuntos.

\begin{ej}
	Supongamos\footnote{Por qué cojones cambia ahora la letrita?} $G = X,\ |G| = n$ y consideramos $\rho: G \to \autom{G} \subset Biy(X)$. Definimos la relación en $X = G$
	\begin{align*}
	aRb \iff \exists g \in G \mid \phi_g(a) = b,\ \phi_g(x) = gx\inv{g}
	\end{align*}
	que es la relación de conjugación dada por el isomorfismo de conjugación de toda la vida.
	
	Ahora, en lugar de pensar en $G = X$ pensamos en $X = \{H < G\}$ (los subgrupos de $G$). Para cualquier isomorfismo de grupos $\beta: G \to G$, tenemos que si $H < G$ entonces $\beta(H) < G$.
	
	Lo que hemos hecho aquí es un caso particular de lo que viene ahora.
\end{ej}

\begin{pro}
	Sea $\alpha: G \to Biy(X),\ g \mapsto \alpha(g)$ un homomorfismo de grupos. Definimos la relación de equivalencia
	\begin{align}
	aRb \iff \exists g \in G \mid \alpha(g)(a) = b
	\end{align}
	Afirmamos que la relación es de equivalencia y que nos divide $G$ en subconjuntos disjuntos (nos particiona $G$).
\end{pro}

\begin{proof}Probamos las 3 propiedades de las relaciones de equivalencia.
	\begin{enumerate}
		\item Reflexiva: $\forall x \in X, a R a$. Por ser $\alpha$ homomorfismo tenemos que $\alpha(e_G) = id_X$ y por tanto $\alpha(e_G)(a) = a$.
		\item Simétrica: $aRb \implies bRa$. Partimos de que $\exists g \in G \mid \alpha(g)(a) = b$. Tomamos $\inv{g} \in G$ y por ser $\alpha$ homomorfismo de grupos tenemos que $\alpha(\inv{g})(b) = \inv{(\alpha(g))}(b) = a$.
		\item Transitiva: $aRb \land bRc \implies aRc$. Partimos de que $\exists g, g' \in G \mid \alpha(g)(a) = b \land \alpha(g')(b) = c$. Tomamos $g'g \in C$ y tenemos que $\alpha(g'g)(a) = \alpha(g')(\alpha(g)(a)) = \alpha(g')(b) = c$ por composición de biyecciones.
	\end{enumerate}
\end{proof}

¿Cómo son las clases que da la partición?

Pues tenemos que $cl(a) = \{\alpha(g)(a) \mid g \in G\}$ para un $a \in G$. Definimos $H_a = \{g \in G \mid \alpha(g)(a) = a\}$. Tenemos por lo visto anteriormente que $H_a < G \land |cl(a)| = [G:H_a]$. Entonces tenemos lo siguiente:
\begin{itemize}
	\item En el caso en que $X = G$ tenemos que $H_a = C(a)$ donde $C(a)$ es el centralizador de $a$ (definición \ref{dfn:centralizador}).
	\item En el caso en que $X = \{H < G\}$ tenemos que $H_a = N(a)$ donde $N(a)$ es el normalizador de $a$ (definición \ref{dfn:normalizador}).
\end{itemize}
Veremos que se pueden dar más casos.

\begin{ej}
	Fijamos $\sigma \in S_n$ y $G = \langle \sigma \rangle$ subgrupo genereado por $\sigma$ en $S_n$. Entonces $G = \langle \sigma \rangle \to S_n = Biy(X)$ algo pasó. Si $X = \{1, 2, \dots, n\}$ definimos $\sigma(1) = 2,\ \sigma(2) = 1,\ \sigma(i) = i+1, i = 3,\dots, n-2,\sigma(n-1) = 3$. La clase $cl(i) = \{\sigma^k(i) \mid k \in \Z\}$ en particular contiene a la identidad ya que $\sigma^{n!} = id$ y $n! \in \Z$. Nos quedan dos clases
	% TODO: dibujito de sigma
	\begin{align*}
	cl(1) &= \{1, 2\} \\
	cl(3) &= \{3, 4, 5, \dots, n - 1\}
	\end{align*}
\end{ej}

Vemos que si fijamos $\sigma$ se define una partición en $\{1, \dots, n\}$ de subconjuntos disjuntos
\begin{align*}
F_1 \cup F_2 \cup \dots \cup F_n
\end{align*}

Si $r = |F_i| > 1$, $F_i = \{i_0, i_1, \dots, i_r\}$ tal que $\sigma(i_0) = i_1, \sigma(i_1) = i_2, \dots, \sigma(i_r) = i_0$.

\begin{dfn}[Ciclo]
	\label{dfn:ciclo}
	Diremos que $\sigma$ es un ciclo de longitud $r$ si en la partición definida
	\begin{align*}
	F_1 \cup F_2 \cup \dots \cup F_n
	\end{align*}
	todas las cajas $F_j,\ j < r$ tienen un único elemento y $F_r$ tiene $r$ elementos.
\end{dfn}

\begin{pro}
	Toda biyección $\sigma \in S_7$ se puede descomponer como composición de ciclos.
\end{pro}

\begin{ej}
	Consideramos\footnote{Utilizamos la notación de biyecciones de \cite{dor96}.}
	\begin{align*}
	\sigma = \left(\begin{array}{ccccccc}
	1 & 2 & 3 & 4 & 5 & 6 & 7 \\
	2 & 1 & 4 & 5 & 6 & 3 & 7
	\end{array}\right)
	\end{align*}
	que nos divide $X = \{1, 2, 3, 4, 5, 6, 7\}$ en tres subconjuntos disjuntos $\{1, 2\},\ \{3, 4, 5, 6\},\ \{7\}$. Por tanto podemos decir
	\begin{align*}
	\sigma = (12)(3456)(7) = (12)(3456) = (3456)(12)
	\end{align*}
	(podemos conmutar porque al ser ciclos disjuntos lo que toque uno no lo toca el otro).
\end{ej}

Proximamente vermos que a partir de la descomposición en ciclos disjuntos es fácil obtener el orden de $\sigma$.

\hr

Falta la semana fatídica de ANAMAT\newline
%TODO: pos eso
\hr

\begin{itemize}
	\item Recordemos que fijado $\sigma \in S_5$ podemos dar una descomposición en ciclos $\sigma = (123)(45)$ que es única aunque los ciclos se escriban diferente (por ejemplo $(123) = (231)$).
	
	\item Fijado $\tau \in S_5$, $\tau \sigma \inv{\tau} = (\tau(1)\tau(2)\tau(3))(\tau(4)\tau(5))$ la descomposición se mantiene
	
	\item Si dos permutaciones $\sigma, \sigma'$ tienen descomposiciones del mismo tipo (un 3-ciclo y un 2-ciclo como antes) entonces existe un $\tau$ que hace pasar de una a otra.
\end{itemize}

\begin{ej}[Posibles descomposiciones en cíclos de $S_4$]
	\begin{itemize}
		\item Para $(1234)$
		\begin{align*}
		cl((1234)) = \{\tau(1234)\inv{\tau} \mid \tau \in S_4\}
		\end{align*}
		\item A la hora de definir $\tau$ tenemos varias posibilidades. En este caso, si empezamos por el $1$, para fijar el segundo elemento solo tenemos 3 posibilidades, para el tercero 2 y para el último una. Por tanto
		\begin{align*}
		|cl((1234))| = 4
		\end{align*}
		
		\item Recordemos que el centralizador
		\begin{align*}
		C_{S_4}((1234)) = \{\sigma \in S_4 \mid \sigma (1234) \inv{\sigma} = (1234)\} < S_4
		\end{align*}
		
		\item Como $S_4$ tiene $|S_4| = 4! = 24$ y tenemos que $|cl((1234))| = [S_4 : C_{S_4}((1234))] = 6$ necesariamente $|C_{S_4}((1234))| = 4$.
		
		\item Nos proponemos calcular el grupo $C((1234))$. Un candidato para $\sigma \in C((1234))$ es $\sigma = (1234)$. En efecto $(1234)(1234)(1234) \in C((1234))$. Siempre ocurre que un elemento conmuta consigo mismo. Además, $\langle (1234) \rangle < C((1234))$ pero como $|\langle (1234) \rangle| = 4 = |C((1234))$ tiene que ocurrir que $\langle (1234) \rangle = C((1234))$. Es decir que de tipo 4 solo tenemos $(1234)$.
		
		\item ¿Qué tipos tenemos? Pues tantos como maneras de descomponer 4 en suma de números positivos, a saber
		\begin{itemize}
			\item (1234) de tipo 4
			\item (123) de tipo 3+1
			\item (12)(34) de tipo 2+2
			\item (12) de tipo 2+1+1
			\item $Id$ de tipo 1+1+1+1 (que es la única que tiene descomposición en 4 unos)
		\end{itemize}
		
		\item En general no es difícil calcular cuantos hay, por lo que a menudo utilizamos este argumento para calcular el grupo centralizador.
		
		\item Lo importante es que estamos descomponiendo $S_4$ de la siguiente manera:
		\begin{align*}
		S_4 &= cl((1234)) \cap cl((1223)) \cap cl((12)(34)) \cap cl((12)) \cap cl(Id) \\
		|S_4| &= |cl((1234))| \cap |cl((1223))| \cap |cl((12)(34))| \cap |cl((12))| \cap |cl(Id)|
		\end{align*}
		\item Ahora analizamos la clase $cl((123))$ de los ciclos de tipo 3+1. Lo primero es saber cuantos hay. Pues tenemos que elegir 3 elementos de entre 4 y luego ordenar los dos que nos quedan por tanto
		\begin{align*}
		|cl((123))| = \binom{4}{3} \times 2 = 8
		\end{align*}
		Por otro lado lo que sabemos es que $(123) \in C((123))$ (porque todos conmutan consigo mismos) y como antes $|C((123))| = 3$ (de la fórmula $|cl((123))| = [S_4:C((123))]$), luego $C((123)) = \langle (123) \rangle$.
		
		\item Igual es un poco más interesante la clase de tipo 2+2. \textbf{Pregunta de examen:} halla generadores del subgrupo centralizador del elemento (12)(34).
		\begin{itemize}
			\item Sabemos que el conjugado de un elemento de tipo 2 tiene que ser otro de tipo 2, por tanto tenemos que ver qué elementos distintos de tipo 2 tenemos. Pues fijamos el 1 por ejemplo y vemos qué parejas podemos hacer. Nos salen 3, a saber, 1 con 2, 1 con 3 y 1 con 4 de lo que concluímos que $|cl((12)(34))| = 3$.
			\item De la misma fórmula que antes sacamos que $|C((12)(34))| = 8$. De orden 8 sabemos que hay solo unos pocos grupos (ver la clasificación en \ref{gruposfinitosnotables}). Veamos con cuál de ellos es isomorfo.
			\item Como siempre sabemos que $(12)(34) \in C((12)(34))$. Tenemos que encontrar los demás $\tau$ que conmutan $\tau \sigma \inv{\tau} = \tau (12)(34) \inv{\tau} = (\tau(1)\tau(2))(\tau(3)\tau(4))$. Probamos con $\tau = (1324)$\footnote{La idea de probar con este viene de decir: pues a ver qué pasa si cambio el 1 con el 3 y el 2 con el 4, que nos daría la permutación (1324). En cualquier caso esto es prueba y error, y parar de probar cuando tengamos un grupo generado de orden 8.}.
			\begin{align*}
			(1324)&(12)(34)\inv{(1324)} \\
			&(34)(21)
			\end{align*}
			Que es el mismo, luego hemos probado que $\tau$ conmuta y por tanto $\tau \in C((12)(34))$. Lástima que no valga porque nos damos cuenta de que $\tau ^2 = (12)(34)$. Vaya. Drácula ha hecho chiste con esto y todo $(X,d)$.\footnote{Aquí se ve claramente que la elección del $\tau$ es casi al azar. Hemos elegido uno que prometía pero hemos tenido la mala suerte de que su cuadrado nos daba un elemento que suponíamos estaba en el grupo ($\tau^2 = (12)(34)$. Podríamos haber descartado este $\tau = (1324)$ pero hemos preferido descartar el elemento (12)(34) que sabíamos que estaba en el grupo. La razón de la sustitución de este último por el (12) es un misterio hasta la fecha.}
			
			Lo que hacemos es quitar el $(12)(34)$ y cambiarlo por el $(12)$. Para evitar $\tau^2 \neq (12)$. En resumen, ya tenemos $(12) \in C((12)(34))$ y $\tau = (1324) \in C((12)(34))$. Si vemos sus grupos generados:
			\begin{align*}
			\langle (1324)\rangle = \{(1324), (12)(23), (4321), Id\} \\
			\langle (12) \rangle = \{(12), Id\}
			\end{align*}
			La intersección de ambos subgrupos es solo la identidad y por la fórmula del producto libre averiguamos que $|\langle (1324)\rangle \langle (12) \rangle| = 8$ por lo $C((12)(34)) = \langle (1324), (12) \rangle$.
			
			Tiene toda la pinta de ser $D_4$ porque está generado por dos elementos, no es abeliano y los órdenes de los generadores son $o((1324)) = 4,\ o((12)) = 2$. Solo nos quedaría probar que se sigue cumpliendo la ecuación de la presentación de $D_4$:
			\begin{align*}
			BA = AB^3 \iff (1324)(12) = (12)(1324)^3
			\end{align*}
			Lo comprobamos y al final sale.
		\end{itemize}
		
		\item Ahora hacemos lo mismo con $C((12))$. Siguiendo un razonamiento similar, llegamos a que $C((12))$ es isomorfo con el grupo de Klein y por extensión con $\Z/2\Z \times \Z/2\Z$.
	\end{itemize}
\end{ej}


Falta la semana fatídica de Estadística

% 20181029
Vez pasada considerabamos $G_1 \times G_2$ y fijado un homomorfismo de grupos $\phi: G_1 \to Aut(G_2)$ hacíamos lo siguiente. En $G_1 \times_{\phi} G_2$ viven los elementos $(a,b) \times_{\phi} (c,d)$ donde la operación cambiaba en la primera coordenada $(a \phi_b(c), bd)$. Probamos la última clase que $G_1 \times_{\phi} G_2$ era un grupo (probar la asociatividad no es trivial).

% 20181030

Observación:

\begin{align*}
	\gamma: G \xrightarrow{Int} Aut(G)
\end{align*}
$\gamma$ es un homomorfismo de grupos que lleva cada elemento $g \in G$ al automorfismo conjugación $\gamma_g(x) = gx\inv{g}$. Observamos que si $N \normsub G,\ \forall g \in G, \gamma_g(N) = gN\inv{g} = N$.

\begin{pro}
	$N$ es normal en $G$ ($N \normsub G$) sí y solo sí al restringir $\phi_g$ a $N$ la imagen es $N$:
	\begin{align*}
		G \xrightarrow{\gamma_g} G \\
		N \xrightarrow{\gamma_g \vert_N} N
	\end{align*}
	Es decir, que si $N$ es normal, $\gamma_g\vert_N$ induce un isomorfismo $\gamma_g\vert_N : N \to N$.
\end{pro}

\begin{proof}
	Cristalina de la definición de subgrupo normal.
\end{proof}

En general, al restringir $\gamma_g$ a un subgrupo de $G$ no tenemos esta propiedad.

Además, si $N \normsub G$ tiene sentido restringir $\gamma: G \xrightarrow{Int} Aut(G)$ a $Aut(N)$ y la restricción da un homomorfismo.

\section{Producto semidirecto}

Sea $G$ un grupo. Sea $N \normsub G$, $H < G$, $N \cap H = \{e\}$ y $NH = G$ (recordemos que por ser $N$ normal, $NH$ es grupo). Entonces $G \isom N \times H$.

Veamos quién es ese isomorfismo $\gamma : G \to N \times H$. Recordemos que considerando dos grupos $G_1, G_2$ y su producto directo $G_1 \times G_2$ existe un $\alpha : G_2 \to Aut(G_1)$. Veremos quien es este $\alpha$ para $H$ y $N$, es decir, quién es $\alpha: H \to Aut(N)$.

Construye $\alpha$ a partir de 4 isomorfismos.

\begin{proof}$ $\newline
	\begin{itemize}
		\item Comenzamos por definir una función $j: N\times H \to G,\ (n, h) \mapsto nh$. Es función está bien definida por teoría de conjuntos pero no es un homomorfismo de grupos\footnote{Ojo con por qué no es homomorfismo. Si tomamos $(n,h),(n', h') \in N \times H$ tenemos que $j((n,h)(n',h')) = nn'hh'$. Podríamos pensar que como $N$ es normal, podemos conmutarlo y obtener $nn'hh' = nhn'h' = j((n,h))j((n',h'))$. \textbf{Pero esto está mal.} Lo que significa ser normal es que para $h \in H$, se tiene que $nh = hn''$ para algún $n'' \in N$.}\footnote{Si los grupos son abelianos entonces sí es claro que es un homomorfismo. Lo que vamos a hacer es ver que dando una estructura especial, sí que es un homomorfismo de grupos incluso para grupos no abelianos}.
		\item Recordemos que por el teorema \ref{thm:cardinalidadproductolibre} tenemos que $|G| = |N||H| = |N \times H|$ por ser $N \cap H = \{e\}$.
		\item Volviendo a lo de la estructura especial. Dar una estructura especial es dar una operación para $N \times H$.
		\begin{itemize}
			\item Sea $A$ un conjunto. Es claro que si tenemos una biyección $\phi : A \to G$ podemos dotar a $A$ de alguna estructura para que sea un grupo.
			\item Para dotar a $A$ de estructura tenemos que definir la operación. Forzamos que para cada $a, a' \in A$ para los que se tiene $\phi(g) = a, \phi(g') = a'$ la operación sea $a a'  = \phi(gg')$.
			\item En este caso nuestro $A$ es $N \times H$. En lugar de utilizar la operación habitual del producto directo definimos otra operación. Para llegar a ella nos fijamos en $(n,h)(n',h') \mapsto nhn'h' = nhn'\inv{h}hh' = n(hn'\inv{h})hh' = nn'hh'$ (intercalamos el neutro, que es legal).
			\item Redefinimos la operación en $N \times H$ para que cuadre con este resultado. Llamaremos al nuevo grupo con la nueva operación $N \times_\phi H$: para $(n,h), (n',h')$ definimos $(n,h)\cdot (n',h') = (n(hn'\inv{h}), hh')$.
			\item Comprobamos que en este caso $j$ es un homomorfismo de grupos:
			\begin{align*}
				j : N \times_\phi H &\to G \\
				(n,h) &\mapsto nh \\
				(n',h') &\mapsto n'h' \\
				(n,h)\cdot(n',h') &\mapsto n(hn'\inv{h})hh' = nn'hh'
			\end{align*}
		\end{itemize} 
	\end{itemize}
\end{proof}

Es muy interesante por que es muy natural llegar a situaciones de esta manera. ¡Y les voy a dar una!\footnote{Sugerencia: leelo con voz de tomatito.}

\begin{ej}
	Sea $|G| = p \cdot q$ y supongamos $p < q$ primos. Por el teorema de Lagrange (\ref{thm:lagrange}) tenemos que existe un subgrupo $H_p < G$ con $|H_p| = p$ y análogamente $\exists H_q \mid |H_q| = q$. A primera vista podríamos pensar que puede haber varios grupos de orden $q$. Pues no.
\end{ej}

\begin{proof}
	Supongamos hay dos grupos $H, H'$ de orden $q$ distintos. La intersección tiene que dar un subgrupo y si los dos grupos tienen un número primo de elementos entonces la intersección solo puede ser el neutro, $H \cap H' = \{e\}$. Entonces por el teorema \ref{thm:cardinalidadproductolibre} tenemos que $|HH'| = q^2 > p\cdot q$ lo que es imposible. Luego sabemos que a lo sumo hay un grupo de orden $q$.
\end{proof}

(Sigue el ejemplo) Supongamos que ese único grupo de orden $q$ se llama $N$. Entonces $\phi_g(N) = N$ ya que un isomorfismo tiene que mandar un subgrupo de $q$ elementos en otro subgrupo de $q$ elementos y $N$ es el único. Por tanto $N \normsub G$. Aplicando el teorema de antes, tenemos que $G \isom N \times H$.

\begin{ej}
	Veamos un ejemplo con más pinta de problema. Demostrar que todo grupo de orden $77$ es cíclico.
\end{ej}

\begin{proof}
	Comenzamos por observar que $77 = 7 \cdot 11$. Por el teorema de Lagrange (\ref{thm:lagrange}) tenemos que existen $H, N < G \mid |H| = 7,\ |N| = 11$ y por lo visto en el ejemplo anterior, $N \normsub H$. Como antes llegamos a que $H \cap N = \{e\}$ y a que $|H N| = pq$. Para aplicar el teorema anterior vemos qué estructura tiene que tener $N \times H$.
	
	Mierda no me da tiempo.
	
	% TODO
\end{proof}


% !TeX root = ../apuntes-ea.tex

\chapter{Biyecciones}

\section{El por qué de la notación cíclica}

\begin{dfn}
	Sea $X$ un conjunto. Definimos
	\begin{align*}
		\biy{X} = \{f: X \to X \mid f \text{ es biyección}\}
	\end{align*}
\end{dfn}

Como coinciden dominio y codominio ($f:X \to X$) si $f$ es inyectiva entonces automáticamente es sobre y por tanto biyectiva.

En general, tiene sentido pensar en $Biy(X)$ aunque $|X| = \infty$. Además, en dicho conjunto viven la biyección identidad y la biyección inversa para cada biyección. Por tanto, tiene sentido pensar en $(Biy(X), \circ)$ como un grupo (la composición de biyecciones da una biyección). Lo escribimos en forma de teorema.

\begin{thm}
	Sea $X$ un conjunto. El par $(\biy{X}, \circ)$ es un grupo.
\end{thm}

Nos concentraremos en el caso en el que $|X| = n < \infty$ que nos da $Biy(X) = S_n$. Ver \autoref{dfn:sn} para una explicación detallada del grupo $S_n$.

Fijamos un conjunto $X$ y un homomorfismo de grupos $\alpha: X \to Biy(X)$. A partir de estos datos definimos una relación de equivalencia que nos da una partición de $X$, es decir, vamos a partir $X$ en conjuntos disjuntos. Veamos un ejemplo particular.

\begin{ej}
	Supongamos $G = X,\ |G| = n$ y consideramos $\rho: G \to \autom{G} \subset Biy(X)$. Definimos la relación en $X = G$
	\begin{align*}
	aRb \iff \exists g \in G \mid \phi_g(a) = b,\ \phi_g(x) = gx\inv{g}
	\end{align*}
	que es la relación de conjugación dada por el isomorfismo de conjugación de toda la vida.
	
	Ahora, en lugar de pensar en $G = X$ pensamos en $X = \{H < G\}$ (los subgrupos de $G$). Para cualquier isomorfismo de grupos $\beta: G \to G$, tenemos que si $H < G$ entonces $\beta(H) < G$.
	
	Lo que hemos hecho aquí es un caso particular de lo que viene ahora.
\end{ej}

Ahora pasamos al caso general.

\begin{pro}
	Sea $\alpha: G \to Biy(X),\ g \mapsto \alpha(g)$ un homomorfismo de grupos\footnote{Ojo: aquí las imágenes de los elementos $g \in G$ son biyecciones $f:G \to G$, por eso tendrá sentido la notación $\alpha(g)(a)$ que significa aplicar la función que nos devuelve $\alpha$ al elemento $a \in G$.}. Definimos la relación de equivalencia $R$ en el conjunto $X$
	\begin{align}
	aRb \iff \exists g \in G \mid \alpha(g)(a) = b
	\end{align}
	Afirmamos que la relación es de equivalencia y que nos divide $X$ en subconjuntos disjuntos (nos particiona $X$).
\end{pro}

\begin{proof}Probamos las 3 propiedades de las relaciones de equivalencia.
	\begin{enumerate}
		\item Reflexiva: $\forall x \in X, a R a$. Por ser $\alpha$ homomorfismo tenemos que $\alpha(e_G) = id_X$ y por tanto $\alpha(e_G)(a) = a$.
		\item Simétrica: $aRb \implies bRa$. Partimos de que $\exists g \in G \mid \alpha(g)(a) = b$. Tomamos $\inv{g} \in G$ y por ser $\alpha$ homomorfismo de grupos tenemos que $\alpha(\inv{g})(b) = \inv{(\alpha(g))}(b) = a$.
		\item Transitiva: $aRb \land bRc \implies aRc$. Partimos de que $\exists g, g' \in G \mid \alpha(g)(a) = b \land \alpha(g')(b) = c$. Tomamos $g'g \in C$ y tenemos que $\alpha(g'g)(a) = \alpha(g')(\alpha(g)(a)) = \alpha(g')(b) = c$ por composición de biyecciones.
	\end{enumerate}
\end{proof}

¿Cómo son las clases que da la partición?

Pues tenemos que para $a \in X$, la clase $cl(a) = \{\alpha(g)(a) \mid g \in G\}$. Definimos $H_a = \{g \in G \mid \alpha(g)(a) = a\}$. Tenemos por lo visto anteriormente que $H_a < G \land |cl(a)| = [G:H_a]$. Entonces tenemos lo siguiente:
\begin{itemize}
	\item En el caso en que $X = G$, es decir, que el conjunto $X$ tiene dentro \textit{elementos} de $G$, tenemos que $H_a = C(a)$ donde $C(a)$ es el centralizador de $a$ (\autoref{dfn:centralizador}).
	\item En el caso en que $X = \{H < G\}$, es decir, que el conjunto $X$ tiene dentro \textit{subgrupos} de $G$, tenemos que $H_a = N(a)$ donde $N(a)$ es el normalizador de $a$ (\autoref{dfn:normalizador}).
\end{itemize}

Vista la definición abstracta, lo que nos interesa de esto es aplicarlo a los grupos $S_n$ de los que hablábamos antes. En particular, ahora daremos una definición formal de ciclo para la notación que introdujimos en la \autoref{sec:notacionciclica}.

\begin{wrapfigure}{l}{0.3\textwidth}
	\begin{tikzpicture}
	\begin{scope}[scale=0.5]
	\node (alpha) at (-2, 0) {$\alpha :=$};
	\node (1) at (0,4) {$1$};
	\node (2) at (0,3) {$2$};
	\node (3) at (0,2) {$3$};
	\node (4) at (0,1) {$4$};
	\node (dots) at (0,-1) {$\vdots$};
	\node (nmenos1) at (0,-2) {$n-1$};
	\node (n) at (0,-3) {$n$};
	
	\node (1p) at (6,4) {$1$};
	\node (2p) at (6,3) {$2$};
	\node (3p) at (6,2) {$3$};
	\node (4p) at (6,1) {$4$};
	\node (5p) at (6,0) {$5$};
	\node (dotsp) at (6,-1) {$\vdots$};
	\node (nmenos1p) at (6,-2) {$n-1$};
	\node (np) at (6,-3) {$n$};
	
	\node (dotsc) at (3, -1) {$\vdots$};
	
	
	\draw (1) -- (2p);
	\draw (2) -- (1p);
	\draw (3) -- (4p);
	\draw (4) -- (5p);
	\draw (nmenos1) -- (np);
	\draw (n) -- (3p);
	\end{scope}
	\end{tikzpicture}
	\caption{La permutación $\alpha$ de $S_n$}
	\label{fig:permalphaejpart}
\end{wrapfigure}

Fijamos $\sigma \in S_n$ y definimos $G = \gen{\sigma}$ el subgrupo generado por $\sigma$ en $S_n$. Definimos ahora el homomorfismo
\begin{align*}
G = \gen{\sigma} & \to S_n = \biy{X},\qquad X = \{1, 2, 3, \dots, n\}
\end{align*}
Las clases $cl(i)$ para $i \in \{1, 2, \dots, n\}$ son de la forma\footnote{Las clases serían de la forma $\alpha(g)(i)$ pero es que en este caso todos los $\alpha(g)$ son elementos de $G = \gen{\sigma}$ y por tanto son de la forma $\sigma^k$.}
\begin{align*}
cl(i) = \{\sigma^k(i) \mid k \in \Z\}
\end{align*}


\begin{ej}
	Consideramos la permutación $\alpha \in S_n$ dada por (ver \autoref{fig:permalphaejpart})
	\begin{align*}
		\alpha = \begin{array}{ccccccc}
		1 & 2 & 3 & 4 & \dots & n-1 & n \\
		2 & 1 & 4 & 5 & \dots & n & 3
		\end{array}
	\end{align*}
	que en la notación cíclica podríamos escribir como $\alpha = (345\dots n)(12)$.
	
	En este caso la clase $cl(1) = \{1, 2\} = cl(2)$ está formada por los elementos que podemos obtener de aplicar $\alpha$ al elemento $1$. Ya se intuye la utilidad de la notación cíclica: la permutación $\alpha$ nunca mezcla elementos de la caja $\{1,2\}$ con elementos de la caja $\{3, 4, 5, \dots, n\}$. Así, también tendremos que $cl(3) = cl(4) = \dots = cl(n) = \{3, 4, 5, \dots, n\}$. Los elementos que hay en estas dos clases coinciden con los elementos que hay en cada uno de los ciclos en los que hemos descompuesto $\alpha$.
\end{ej}

Vemos que si fijamos $\sigma$ se define una partición en $\{1, \dots, n\}$ de subconjuntos disjuntos
\begin{align*}
F_1 \cup F_2 \cup \dots \cup F_n
\end{align*}

Si $r = |F_i| > 1$, $F_i = \{i_0, i_1, \dots, i_r\}$ tal que $\sigma(i_0) = i_1, \sigma(i_1) = i_2, \dots, \sigma(i_r) = i_0$.

\begin{dfn}[Ciclo]
	\label{dfn:ciclo}
	Diremos que $\sigma$ es un ciclo de longitud $r$ si en la partición definida
	\begin{align*}
	F_1 \cup F_2 \cup \dots \cup F_n
	\end{align*}
	todas las cajas $F_j,\ j < r$ tienen un único elemento y $F_r$ tiene $r$ elementos.
\end{dfn}

La definición quiere decir que, en el fondo, un ciclo es un tipo de permutación que al aplicarla sucesivamente sobre el conjunto $X$ lo particiona en varias cajas pero de manera que todas tienen un elemento excepto una, que tiene todos los elementos que se mueven entre ellos por la acción del ciclo. Un ejemplo en el conjunto $X = \{1, 2, 3, \dots, n\}$ sería

\begin{center}
	\begin{tabular}{|c|c|c|c|}
		\hline
		1 & 5 & \dots & \vdots \\\cline{2-4}
		2 & 6 & $\ddots$ & \vdots \\\cline{2-4}
		3 & $\ddots$ & $\ddots$ & \vdots  \\\cline{2-4}
		4 & \dots & \dots & n\\\hline
	\end{tabular}
\end{center}

Observemos que por la notación que hemos elegido, los ciclos tienen la estructura $(\sigma^0(a)\ \sigma^1(a)\ \sigma^2(a) \dots \sigma^s(a))$ donde $\sigma$ es un elemento de $S_n$ y $a$ un elemento de $X$. Dado que si $\sigma^k = Id$ entonces $\sigma^{k + i} = \sigma^i$, si \textit{rotamos} los números que definen el ciclo no estamos haciendo nada. Esto es, el ciclo $(1234) = (2341) = (3412) = (4123)$.

\section{De permutaciones a composiciones de ciclos}

\begin{pro}
	Toda biyección $\alpha \in S_n$ se puede expresar como composición de ciclos disjuntos dos a dos:
	\begin{align*}
		\alpha = \sigma_1 \circ \sigma_2 \circ \dots \circ \sigma_s
	\end{align*}
\end{pro}

\begin{pro}
	La composición de dos ciclos disjuntos conmuta, es decir, si $\sigma_1$ y $\sigma_2$ son ciclos disjuntos (que no comparten ningún elemento entre los paréntesis) entonces $\sigma_1 \circ \sigma_2 = \sigma_2 \circ \sigma_1$
\end{pro}

\begin{cor}
	Toda descomposición de una permutación $\alpha \in S_n$ en ciclos disjuntos $\alpha = \sigma_s \circ \sigma_{s-1} \circ \dots \circ \sigma_2 \circ \sigma_1$ se puede reordenar sin cambiar el resultado.
\end{cor}

\begin{ej}
	Antes de seguir veamos un ejemplo más de cómo una biyección de $S_n$ particiona el conjunto $X = \{1, 2, \dots, n\}$.
	
	Consideramos $\alpha \in S_n$ definida con
	\begin{align*}
		\alpha = \left(\begin{array}{cccccccccc}
		1 & 2 & 3 & 4 & 5 & 6 & 7 & 8 & 9 & 10 \\
		2 & 3 & 1 & 5 & 6 & 4 & 7 & 9 & 8 & 10
		\end{array}\right)
	\end{align*}
	
	La partición que nos da $\alpha$ de $X = \{1, 2, 3, 4, 5, 6, 7, 8, 9 10\}$ es la siguiente:
	\begin{center}
		\begin{tabular}{|c|c|c|c|}
			\hline
			1 & 4 & 7 & \\ \cline{3-3}
			2 & 5 & 8 & 10 \\
			3 & 6 & 9 & \\
			\hline
		\end{tabular}
	
		Partición de $X$ dada por $\alpha = (123)(456)(89)$
	\end{center}
	Esto lo obtenemos de buscar las clases de cada elemento. Empezamos por el que queramos, por ejemplo, el $1$:
	\begin{align*}
		cl(1) = \{\alpha^k(1) \mid k \in \Z\} = \{\alpha^0(1) = 1, \alpha^1(1) = 2, \alpha^2(1) = 3, \alpha^3(1) = 1, \alpha^4(1) = 2, \dots \}
	\end{align*}
	Eliminando duplicidades obtenemos que $cl(1) = \{1,2,3\}$. Análogamente obtenemos $cl(4) = \{4,5,6\},\ cl(7) = \{7\},\ cl(8) = \{8,9\},\ cl(10) = \{10\}$. Lo que hemos hecho es seguir el algoritmo descrito en la \autoref{sec:notacionciclica}, esta vez entendiendo el significado. Obtenemos que $\alpha = (123)(456)(89)$ o cualquier reordenación de los ciclos anteriores, ya que al ser disjuntos, cambiar el orden en el que los rotamos no afecta al resultado.
\end{ej}

\begin{wrapfigure}{l}{0.3\textwidth}
	\centering
	\begin{tikzpicture}
	\node (1) at (0,1) {${1}$};
	\node (2) at (0.866,0.5) {$2$};
	\node (3) at (0.866,-0.5) {$3$};
	\node (4) at (0, -1) {$4$};
	\node (5) at (-0.866, -0.5) {$5$};
	\node (6) at (-0.866,0.5) {$6$};
	
	\draw[->] 	(1) edge (2)
	(2) edge (3)
	(3) edge (4)
	(4) edge (5)
	(5) edge (6)
	(6) edge (1);
	\end{tikzpicture}
	\caption{El ciclo $\sigma = (123456)$}
	\label{fig:ciclo16}
	
	
	\begin{tikzpicture}
	\node (1) at (0,1) {${1}$};
	\node (2) at (0.866,0.5) {$2$};
	\node (3) at (0.866,-0.5) {$3$};
	\node (4) at (0, -1) {$4$};
	\node (5) at (-0.866, -0.5) {$5$};
	\node (6) at (-0.866,0.5) {$6$};
	
	\draw[->] 	(1) edge (3)
	(3) edge (5)
	(5) edge (1)
	(2) edge (4)
	(4) edge (6)
	(6) edge (2);
	\end{tikzpicture}
	\caption{El ciclo $\sigma^2 = (123456)^2$}
	\label{fig:ciclo16cuadrado}
	
	\begin{tikzpicture}
	\node (1) at (0,1) {${1}$};
	\node (2) at (0.866,0.5) {$2$};
	\node (3) at (0.866,-0.5) {$3$};
	\node (4) at (0, -1) {$4$};
	\node (5) at (-0.866, -0.5) {$5$};
	\node (6) at (-0.866,0.5) {$6$};
	
	\draw[->] 	(1) edge[bend left] (4)
	(4) edge[bend left] (1)
	(2) edge[bend left] (5)
	(5) edge[bend left] (2)
	(3) edge[bend left] (6)
	(6) edge[bend left] (3);
	\end{tikzpicture}
	\caption{El ciclo $\sigma^3 = (123456)^3$}
	\label{fig:ciclo16cubo}
\end{wrapfigure}

Veamos ahora cómo se relacionan los órdenes de los ciclos con su longitud.


\begin{ej}
	Consideramos $\sigma = (123456) \in \S_n$. Observamos que $\sigma^6 = Id$ es decir que $\sigma$ tiene orden 6.
	
	De esta manera si nos preguntan por $\sigma^{122} = (123456)^{122} = (123456)^{6\cdot20} \circ (123456)^2 = (123456)^2$ no nos asustamos.
	
	Si nos hubieran dado $\sigma$ con la notación habitual, aparte de que hubiera ocupado mucho, no podríamos haber resuelto esta operación tan rápido.
\end{ej}


\begin{ej}
	Nos preguntamos ahora por las potencias de $\sigma = (123456)$ menores que $6 = o(\sigma)$.
	\begin{itemize}
		\item $\sigma^2$ equivaldría a aplicar $\sigma$ dos veces a cada número $\{1, \dots, 6\}$ (los demás números no nos interesan porque sabemos que $\sigma$ no los mueve). Ayudándonos del dibujo obtenemos que $\sigma^2 = (135)(246)$.
		
		Se verifica que $\sigma^2$ tiene $o(\sigma^2) = 3$ y además si recordamos el \autoref{thm:ordendepotencia} comprobamos que se verifica $o(\sigma^2) = \frac{o(\sigma)}{mcd(o(\sigma), 2)} = \frac{6}{2} = 3$.
		
		\item En cuanto a $\sigma^3$ observamos que al aplicar $\sigma$ 3 veces nos quedan 3 ciclos y que se vuelve a verificar que $o(\sigma^3) =\frac{o(\sigma)}{mcd(o(\sigma), 3)} = \frac{6}{3} = 2$
	\end{itemize}
\end{ej}

Esto nos lleva a enunciar el siguiente teorema

\begin{thm}
	\label{thm:ordenpotenciasciclos}
	Sea $\sigma = (i_1\ i_2\ i_3 \dots i_n)$ un ciclo de longitud $n$. Sea $m \in \Z$ y $d = mcd(n,m)$. Entonces $\sigma^m$ es un producto de $d$ ciclos de longitud $\frac{n}{d}$ y estos son disjuntos dos a dos.
\end{thm}

Poder averiguar los órdenes de ciclos es una herramienta muy potente. Por ejemplo, podemos hacer lo siguiente.

\begin{ex}[H3.8]
	Demuestra que el subgrupo $G < S_4$ generado por los elementos $\sigma = (1432)$ y $\tau = (24)$ es isomorfo a $D_4$.
\end{ex}

\begin{proof}
	Sabemos que $o(\sigma) = 4$ y que $o(\tau) = 2$. Trabajando un poco vemos que
	\begin{align*}
		\gen{\sigma} &= \{\sigma = (1432), \sigma^2 = (13)(24), \sigma^3 = (4321), \sigma^4 = Id\} \\
		\gen{\tau} &= \{\tau = (24), \tau^2 = Id\}
	\end{align*}
	Faltaría ver que $\sigma \tau = \tau \sigma^3$ es decir que $(1432)(24) = (24)(4321)$ (spoiler: es verdad) y ya podríamos identificar $\sigma$ con $B$ y $\tau$ con $A$ para obtener la presentación del famoso grupo $D_4$:
	\begin{align*}
		D_4 \isom G = \gen{\sigma, \tau \mid o(\sigma) = 4 \land o(\tau) = 2 \land \sigma \tau = \tau \sigma^3}
	\end{align*}
\end{proof}

\begin{thm}
	Sea $\alpha$ una permutación expresada como composición de ciclos disjuntos $\alpha = \sigma_1 \circ \sigma_2 \circ \dots \circ \sigma_n$. Entonces el orden de $\alpha$ es el mínimo común múltiplo de los órdenes de cada $\sigma_i$:
	\begin{align*}
		\alpha = \sigma_1 \circ \sigma_2 \circ \dots \circ \sigma_n \text{ disjuntos } \implies o(\alpha) = mcm(\sigma_1, \dots, \sigma_n)
	\end{align*}
\end{thm}

% TODO demostrar esto: dorronsoro página 120

\begin{proof}
	Ver \cite{dor96} página 120.
\end{proof}

\section{Trasposiciones}

\begin{dfn}[Trasposición]
	Una trasposición es un ciclo de orden 2. Cualquier trasposición tiene orden 2.
\end{dfn}

Las trasposiciones tienen la forma $(a\ b)$ pero observemos que también se pueden escribir como $(b\ a)$ ya que lo que estamos haciendo es \textit{rotar} (o empezar en otro lugar del ciclo).

\begin{pro}
	La inversa de cualquier trasposición es ella misma.
\end{pro}

\begin{thm}
	El grupo $S_n$ está generado por las transposiciones $\sigma \in S_n$.
\end{thm}

Ya sabemos que cualquier permutación se puede expresar como producto de ciclos [disjuntos]. Para probar este teorema probaremos la siguiente proposición:

\begin{pro}
	Cualquier ciclo se puede expresar como composición de trasposiciones.
\end{pro}

La prueba es constructiva y describe la manera de expresar un ciclo como composición de trasposiciones.

\begin{proof}
	Sabemos que un ciclo $\sigma$ se escribe como $\sigma = (\sigma^0(a) = a\ \sigma^1(a)\ \sigma^2(a)\ \dots \ \sigma^s(a))$. Pues vasta con observar que la composición
	\begin{align*}
		\sigma = (a\ \sigma^s(a))(a\ \sigma^{s-1}(a))\dots(a\ \sigma^2(a))(a\ \sigma(a))
	\end{align*}
	tiene el mismo efecto.
\end{proof}

\begin{ej}
	La permutación $\sigma = (1234)$ se puede expresar como $\sigma = (14)(13)(12)$.
\end{ej}

\subsection{Paridad de las trasposiciones}

\begin{thm}
	\label{thm:paridadpermutaciones}
	Si $\sigma \in S_n$ se puede descomponer como un número par de trasposiciones entonces toda expresión en $\sigma$ expresada como una composición de un número par de trasposiciones.
	
	Análogamente para las permutaciones que se pueden expresar como una composición de un número impar de trasposiciones.
\end{thm}

\begin{proof}
	Definimos una función
	\begin{align*}
		S_n &\to GL_n(\N)\\
		\sigma &\mapsto \left(\begin{array}{ccc}
		e_\sigma(1) & \dots & e_\sigma(n) \\
		\vdots & \vdots & \vdots
		\end{array}\right)
	\end{align*}
	Esta función es un homomorfismo de grupos.
	
	Entonces si expresamos $\sigma$ como composición de trasposiciones $\sigma = (i_1^{(1)}\ i_2^{(1)})(i_1^{(2)}\ i_2^{(2)}) \dots (i_1^{(r)}\ i_2^{(r)})$ y aplicamos la función que hemos definido nos queda
	\begin{align*}
		A = \left(\begin{array}{ccc}
		e_\sigma(1) & \dots & e_\sigma(n) \\
		\vdots & \vdots & \vdots
		\end{array}\right) = \underbrace{\left(\begin{array}{ccc}
			i_1^{(1)} & \dots & i_2^{(1)} \\
			\vdots & \vdots & \vdots
			\end{array}\right)}_{\det = -1} \dots \underbrace{\left(\begin{array}{ccc}
			i_1^{(r)} & \dots & i_2^{(r)} \\
			\vdots & \vdots & \vdots
			\end{array}\right)}_{\det = -1}
	\end{align*}
	y entonces
	\begin{align*}
		\det A = (-1)^r = \begin{cases}
		1 &\text{ si r es par} \\
		-1 &\text{ si r es impar}
		\end{cases}
	\end{align*}
\end{proof}

Visto que la paridad de una permutación va a ser invariante por la expresión como composición de trasposiciones que elijamos vamos a darle nombre ya que parece importante

\begin{dfn}[Paridad de una permutación]
	Sea $\sigma \in S_n$.
	\begin{itemize}
		\item Diremos que $\sigma$ es par si se puede descomponer como una composición de un número par de trasposiciones.
		
		\item Diremos que $\sigma$ es impar si se puede descomponer como una composición de un número impar de trasposiciones.
	\end{itemize}
\end{dfn}

En otros textos, esto se define con la \textit{signatura}

\begin{dfn}[Signatura de una permutación]
	Sea $\sigma \in S_n$ una permutación que podemos descomponer como una composición de $r$ trasposiciones: $\sigma = \tau_1 \circ \tau_2 \circ \dots \circ \tau_r$. Llamamos signatura de $\sigma$ al número $(-1)^r$ y lo denotamos por $\text{sig}(\sigma) = (-1)^r$.
\end{dfn}

Es muy interesante la manera en la que hemos demostrado el \autoref{thm:paridadpermutaciones}. El homomorfismo que hemos construido de $S_n$ a $GL_n(\N)$ se puede extender para llegar al determinante:

\begin{align*}
	\varphi : S_n &\to GL_n(\R) &\to (\{-1, 1\}, \cdot) \\
	\sigma &\mapsto A = \left(\begin{array}{ccc}
	e_\sigma(1) & \dots & e_\sigma(n) \\
	\vdots & \vdots & \vdots
	\end{array}\right) &\mapsto \det(A)
\end{align*}

Si consideramos el homomorfismo desde $S_n$ hasta $(\{-1, 1\}, \cdot)$ nos damos cuenta de que hemos definido un homomorfismo de grupos que además es sobreyectivo.

El núcleo de dicho isomorfismo $\ker \varphi = \{\sigma \in S_n \mid \varphi(\sigma) = 1\}$ es un subgrupo por el teorema de correspondencia entre familias de subgrupos bajo un epimorfismo (ver \autoref{thm:correspondenciasubgruposdor96}). Además este subgrupo es normal y de índice 2. Tan importante es que le damos nombre.

\begin{dfn}[Grupo alternado]
	Sea $\varphi: S_n \to ({-1, 1}, \cdot)$ el homomorfismo de grupos definido arriba. Definimos el grupo alternado $A_n$ como
	\begin{align*}
		A_n = \ker \varphi = \{\sigma \in S_n \mid \sigma \text{ es par}\}
	\end{align*}
\end{dfn}

Recogemos los resultados que hemos dejado caer antes de la definición:

\begin{pro}
	$A_n \normsub S_n$ y además $[S_n : A_n] = 2$
\end{pro}

\begin{cor}
	Todo grupo $S_n$ tiene un subgrupo normal de orden $2$.
\end{cor}

% !TeX root = ../apuntes-ea.tex

\chapter{Lo nuevo - Parte 2}

Apuntes de Santorum desde la definición de grupo simple.
Las dos definiciones siguientes no están explicitas en los apuntes de Santorum y puede que no sean de mi parte pero las necesito para la legibilidad de la misma.
\begin{dfn}[Grupo de biyecciones]
	\label{dfn:grupobiy}
	Sea $X$ un conjunto, definimos $Biy(X)=\left\{f\mid f : X \longrightarrow X  \right\}$ como el conjunto de biyecciones de $X$ en $X$. Si $|X| \not \eq \infty$, entonces $Biy(X) = S_n$ siendo $S_n$ el conjunto de simetrías o el conjunto de permutaciones de n elementos. $(Biy(X), \circ)$ es un grupo.
\end{dfn}

\begin{dfn}[Grupo alternante]
	\label{dfn:grupopermpar}
	Sea $(S_n,\circ)$ el grupo de permutaciones de $n$ elementos. Llamamos grupo alternante $A_n \subseteq S_n$ al subgrupo de $S_n$ formado por las permutaciones que resultan de componer un número par de transposiciones.
\end{dfn}

\begin{dfn}[Grupo simple]
	Sea $G$ un grupo, decimos que $G$ es un grupo simple si los únicos grupos normales son $G$ y el grupo neutro $\{e\}.$
\end{dfn}

A continuación demostraremos que el grupo alternante $A_n$, es simple para $n\geq 5$. La demostración de este resultado requiere distintas proposiciones.
\begin{pro}
	Sea $G$ un grupo. Si $G$ es finito y abeliano $\implies\ G$ es simple.
\end{pro}
%TODO: demostracion de la proposición.
\begin{pro}
	Sea $A_n$ un grupo alternante, $A_n$ es generado por 3-ciclos para $n\geq 3$.
\end{pro}
\begin{proof}
	Sea $\sigma \in A_n$, entonces $\sigma = (i_1^1\ i_2^1)(i_1^2\ i_2^2)\ldots(i_1^{2n}\ i_2^{2n})$ una composición de un número par de composiciones. Vamos a ver que para cualquier par de transposiciones $(i\ j)(k\ l)$ podemos expresarla como un $3-ciclo$.
	\begin{align*}
		(i\ j)(k\ l) &= (i\ k\ j)(i\ k\ l)\ &si\ los\ elementos\ son\ diferentes.\\
		(i\ j)(i\ l) &= (i\ l\ j)\ &si\ tienen\ un\ elemento\ en\ comun.
	\end{align*}
	Por tanto, como $\forall \sigma \in A_n$ puede ser expresado como un $3-ciclo$ o una composición de estos, $A_n$ está generado por los ciclos de longitud 3.
\end{proof}
\begin{pro}
	\label{pro:alternante3ciclos}
	Sea $A_n$ el grupo alternante de un conjunto de $n$ elementos, $A_n$ es generado por 3-ciclos de la forma $(s\ t\ i)$ con $s,t\in \{1\ldots n\}$ fijos e $i\in\{1\ldots n\}\setminus\{s,t\}$
\end{pro}
\begin{proof}
	Cada 3-ciclo es el producto de 3-ciclos del tipo $(s\ t\ i)$ con $s,t$ fijos e $i$ variable, pues:
	\begin{align*}
		(s\ a\ t) &= (s\ t\ a)^2\\
		(s\ a\ b) &= (s\ t\ b)(s\ t\ a)^2\\
		(t\ a\ b) &= (s\ t\ b)^2(s\ t\ a)\\
		(a\ b\ c) &= (s\ t\ a)^2(s\ t\ c)(s\ t\ b)^2(s\ t\ a)
	\end{align*}
	Entonces, como $A_n$ está generado por 3-ciclos, $A_n$ está generado por ciclos de la forma $(s\ t\ i)$
\end{proof}
\begin{thm}[Igualdad entre subgrupos y grupos alternantes]
	\label{thm:subequalsalternate}
	Si un subgrupo normal $H$ de $A_n$ contiene un 3-ciclo $\implies H = A_n$
\end{thm}
\begin{proof}
	Supongamos que $H$ es no trivial y contiene un 3-ciclo de la forma $(s\ t\ a)$. Usando la normalidad de $H$ vemos que:
	\[
		[(s\ t)(a\ i)](s\ t\ a)^2[(s\ t)(a\ k)]^{-1} = (s\ t\ i)
	\]
	está en $H$. Luego, $H$ debe contener todos los ciclos $(s\ t\ i)$ para $1 \geq i \geq n$. Por la proposición \ref{pro:alternante3ciclos}, estos 3-ciclos generan $A_n$; luego $H = A_n$.
\end{proof}
\begin{pro}
	\label{pro:3cicloinsubgralter}
	Para $n\geq 5$, todo $H \normsub A_n$ contiene un 3-ciclo.
\end{pro}
\begin{proof}
	Sea $e \not \eq \sigma \in H$, existen varias posibles estructuras de ciclos para $\sigma$.
	\begin{itemize}
		\item $\sigma$ es un 3-ciclo.
		\item $\sigma$ es el producto de ciclos disjuntos, $\sigma = \tau(a_1\ a_2 \cdots a_r)\in H$, con $r\geq 3$.
		\item $\sigma$ es el producto de ciclos disjuntos, $\sigma = \tau(a_1\ a_2\ a_3)(a_4\ a_5\ a_6)$.
		\item $\sigma = \tau(a_1\ a_2\ a_3)$, donde $\tau$ es el producto de 2-ciclos disjuntos.
		\item $\sigma = \tau(a_1\ a_2)(a_3\ a_4)$, donde $\tau$ es el producto de un número par de 2-ciclos disjuntos.
	\end{itemize}
	La demostración sigue con el desarrollo de cada uno de los casos, utilizando la normalidad de $H$ para ver que en todos los casos se llega a que $H$ contiene un 3-ciclo.
\end{proof}
\begin{thm}[Simplicidad del grupo alternante]
	Sea $(A_n, \circ)$ el grupo alternante de un conjunto de $n$ elementos. $A_n$ es simple $\forall n \geq 5$.
\end{thm}
\begin{proof}
	Sea $H$ un subgrupo normal no trivial de $A_n$, por la proposición \ref{pro:3cicloinsubgralter}, $H$ contiene un 3-ciclo. Por el teorema \ref{thm:subequalsalternate}, $H = A_n$; por tanto, $A_n$ no contienen ningún subgrupo normal que sea propio y no trivial para $n\geq 5$.
\end{proof}
Falta la semana fatídica de Estadística

% 20181029
Vez pasada considerabamos $G_1 \times G_2$ y fijado un homomorfismo de grupos $\phi: G_1 \to Aut(G_2)$ hacíamos lo siguiente. En $G_1 \times_{\phi} G_2$ viven los elementos $(a,b) \times_{\phi} (c,d)$ donde la operación cambiaba en la primera coordenada $(a \phi_b(c), bd)$. Probamos la última clase que $G_1 \times_{\phi} G_2$ era un grupo (probar la asociatividad no es trivial).

% 20181030

Observación:

\begin{align*}
\gamma: G \xrightarrow{Int} Aut(G)
\end{align*}
$\gamma$ es un homomorfismo de grupos que lleva cada elemento $g \in G$ al automorfismo conjugación $\gamma_g(x) = gx\inv{g}$. Observamos que si $N \normsub G,\ \forall g \in G, \gamma_g(N) = gN\inv{g} = N$.

\begin{pro}
	$N$ es normal en $G$ ($N \normsub G$) sí y solo sí al restringir $\phi_g$ a $N$ la imagen es $N$:
	\begin{align*}
	G \xrightarrow{\gamma_g} G \\
	N \xrightarrow{\gamma_g \vert_N} N
	\end{align*}
	Es decir, que si $N$ es normal, $\gamma_g\vert_N$ induce un isomorfismo $\gamma_g\vert_N : N \to N$.
\end{pro}

\begin{proof}
	Cristalina de la definición de subgrupo normal.
\end{proof}

En general, al restringir $\gamma_g$ a un subgrupo de $G$ no tenemos esta propiedad.

Además, si $N \normsub G$ tiene sentido restringir $\gamma: G \xrightarrow{Int} Aut(G)$ a $Aut(N)$ y la restricción da un homomorfismo.

\section{Nuevas estructuras de grupo en el producto directo}

Sean $G_1, G_2$ grupos, queremos definir nuevas estructuras de grupo en el producto $G_1 \times G_2$.
Para ello comenzaremos definiendo una operación $\ast_\alpha$. Fijamos un homomorfismo de grupos $\alpha:G_2 \longrightarrow Aut(G_1)$, con $Aut(G_1)$ el grupo de automorfismos de $G_1$.\\\\
Sean $(a,b),(c,d) \in G_1\times G_2$, definimos $\ast_\alpha$ como:
\[
	(a,b)\ast_\alpha(c,d) = (a\cdot\alpha(b)\cdot c, b\cdot d).
\]
Donde $b\in G_2,\ \alpha(b) \in G_1$ y $\alpha(b)\cdot c \in G_1$.\\\\
Vamos a ver que $(G_1 \times G_2, \ast_\alpha)$ es un grupo.
\begin{thm}[Grupo producto directo]
	$(G_1 \times G_2, \ast_\alpha)$ es un grupo.
\end{thm}
Vamos a demostrar cada una de las propiedades del grupo:
\begin{itemize}
	\item Asociatividad.
		\begin{proof}
			\begin{align*}
				(a\cdot\alpha(b)\cdot c, bd) \ast_\alpha (h,f) &= (a\cdot\alpha(b)\cdot c\cdot \alpha(bd)\cdot h, b\cdot d\cdot h)\\
				(a,b)\ast_\alpha(c\cdot\alpha(d)\cdot h, df) &= (a\cdot\alpha(b)\cdot c\cdot \alpha(d)\cdot h, b\cdot d\cdot h)
			\end{align*}
			Entonces, falta ver que $\alpha(d)\cdot h = \alpha(bd)\cdot h$. Definimos el isomorfismo de grupo:
			\begin{align*}
				\alpha(b) : G_1 &\longrightarrow G_1\\
				c &\longmapsto \alpha(b)\cdot c\\
				\alpha(d)\cdot h &\longmapsto \alpha(b)\cdot(\alpha(d)\cdot h) = \alpha(bd) \cdot h.
			\end{align*}
			Por tanto, son iguales y la operación es asociativa.
		\end{proof}	
	\item Existencia del elemento neutro.
		\begin{proof}
			Sean $e_1$ y $e_2$ elementos neutros de $G_1$ y $G_2$ respectivamente. Recordamos que por el argumento anterior $\alpha(b)\cdot e_1 = e_1$.
			\begin{align*}
				(a,b) \ast_\alpha (e_1, e_2) = (a \cdot \alpha(b) \cdot e_1, b \cdot e_2) = (a,b)
			\end{align*}
		\end{proof}
	\item Existencia del inverso.
		\begin{proof}
			Hemos de hallar $(c,d) \mid (a,b)\ast_\alpha(c,d) = (e_1,e_2)$.  Entonces, hemos de hallar $c$ y $d$ tal que:
			\begin{align*}
				a \cdot \alpha(b) \cdot c &= e_1\\
				b \cdot d &= e_2
			\end{align*}
			Es fácil ver que $\exists d$ y $d = b^{-1}$. Como $\alpha(b)$ es un isomorfismo $\implies \exists (\alpha(b))^{-1}$, entonces, $c = \alpha(b^{-1}) \cdot a^{-1} = a^{-1}$, por tanto $\exists c$ y $c = a^{-1}$.
		\end{proof}
		
\end{itemize}
Por tanto, el par $(G_1 \times G_2, \ast_\alpha)$ tiene estructura de grupo.\\

Vamos a ver ahora ciertas relaciones del producto cruz con la operación que acabamos de definir. Para abreviar, al par $(G_1 \times G_2, \ast_\alpha)$ lo denominaremos por $G_1 \times_\alpha G_2$.\\\\
Sean $G_1, G_2$ grupos finitos, definimos:
\begin{align*}
	G_1^\ast &= \{(a, e_2) \mid a \in G_1\}\\
	G_2^\ast &= \{(e_1, b) \mid a \in G_2\}
\end{align*}
Es fácil ver que $G_1^\ast < G_1 \times_\alpha G_2$ y $G_2^\ast < G_1 \times_\alpha G_2$. Además,
\begin{align*}
	|G_1^\ast\cdot G_2^\ast| &= \frac{|G_1^\ast|\cdot |G_2^\ast|}{|G_1^\ast \cap G_2^\ast|} = \frac{|G_1^\ast|\cdot |G_2^\ast|}{1} = |G_1|\cdot |G_2| = |G_1 \times_\alpha G_2|\\
	G_1^\ast \cap G_2^\ast &= {(e_1, e_2)}
\end{align*}
Y podemos probar que $G_1^\ast$ es normal, sean $g_1 \in G_1$ y $g_2 \in G_2$:
\begin{align*}
	(g_1, g_2) \ast_\alpha (a, e_2) \ast_\alpha (g_1, g_2)^{-1} = (g_1,g_2)\ast_\alpha(\ldots, e_2\cdot g_2^{-1}) = (\ldots, e_2).
\end{align*}
\begin{cor}
	\label{cor:propiedadesgrupdirecto}
	Por lo que acabamos de ver:
	\begin{itemize}
		\item $\hat{G_1}$ y $\hat{G_2}$ son subgrupos.
		\item $\hat{G_1}$ es normal.
		\item $G_1^\ast \cap G_2^\ast = \{(e_1,e_2)\}$
		\item $G_1^\ast \cdot G_2^\ast = G_1 \times_\alpha G_2$
	\end{itemize}
	Si ahora tomamos $G_1 = N, G_2 = H$ con $N \normsub G, H < G$, entonces:
	\begin{itemize}
		\item $H \cap N = \{e\}$
		\item $H \cdot N = G$
		\item $\alpha: H \longrightarrow Aut(N)$
		\item $G \cong H \times_\alpha N$
	\end{itemize}
\end{cor}
En particular, podemos definir:
\begin{align*}
	\phi : H &\longrightarrow Aut(N)\\
	h &\longmapsto \gamma_h\mid_N(n) = h\cdot n\cdot h^{-1}
\end{align*}
\begin{ej}
	Sea el famoso grupo $D_4 = \{1,B,B^2,B^3,A,AB,AB^2,AB^3\}$ (ver ejemplo \ref{ej:famosogrupod4}). Tomamos $N = \langle B \rangle =\{1,B,B^2,B^3\},\ H = \langle A \rangle =\{1,A\}$. Entonces:
	\begin{align*}
		\phi: H &\longrightarrow \autom{N}\\
		A &\longmapsto ABA^{-1} = B^3
	\end{align*}
Entonces como hemos visto: $D_4 \cong \{1,A\} \ast_\phi \{1,B,B^2,B^3\}$.
\end{ej}
\section{Clase de equivalencia por el grupo de biyecciones}
\begin{dfn}[Clase de equivalencia por el grupo de biyecciones]
Sea $(G, \ast)$ un grupo, $X$ un conjunto, $Biy(X) = \{f\mid f: X\longrightarrow X\ biyeccion\}$, y $\alpha: G \longrightarrow Biy(X)$ es un homomorfismo de grupos. Definimos la siguiente relación de equivalencia:
\begin{align*}
	a\mathcal{R}b\ si\ \exists g \mid \alpha(g)a=b
\end{align*}\\
Por esta relación, definimos la \textbf{clase} de equivalencia de un elemento $a \in X$ como:  \[cl(a)=\{ \alpha(g)(a)\mid g\in G \} \ni a \]
\end{dfn}
Para poder definirlo mejor nos gustaría saber cuantos elementos existen en $cl(a)$. Para ello nos ayudaremos del $centralizador\ de\ a$ (definición \ref{dfn:centralizador}). En nuestro caso particular, el centralizador es:
\[ C_G(a) \{g\in G \mid \alpha(g)(a) = a \} \]
Es fácil ver que si $g \in C_G(a)$ y $g' \in C_G(a)$ entonces $g \ast g' = C_G(a)$.
\begin{thm}[Orden de la clase de equivalencia de un elemento]
Sea $(G, \ast)$ un grupo, $C(a)$ el centralizador de $a$ y $cl(a)$ la clase de equivalencia de $a$:
\[ |cl(a)| = \left[ G:C(a) \right] \]
\end{thm}
%TODO: Demostración
\begin{itemize}
	\item Recordemos que fijado $\sigma \in S_5$ podemos dar una descomposición en ciclos $\sigma = (123)(45)$ que es única aunque los ciclos se escriban diferente (por ejemplo $(123) = (231)$).
	
	\item Fijado $\tau \in S_5$, $\tau \sigma \inv{\tau} = (\tau(1)\tau(2)\tau(3))(\tau(4)\tau(5))$ la descomposición se mantiene
	
	\item Si dos permutaciones $\sigma, \sigma'$ tienen descomposiciones del mismo tipo (un 3-ciclo y un 2-ciclo como antes) entonces existe un $\tau$ que hace pasar de una a otra.
\end{itemize}

\begin{ej}[Posibles descomposiciones en cíclos de $S_4$]$ $ \newline
	\begin{itemize}
		\item Para $(1234)$
		\begin{align*}
		cl((1234)) = \{\tau(1234)\inv{\tau} \mid \tau \in S_4\}
		\end{align*}
		\item A la hora de definir $\tau$ tenemos varias posibilidades. En este caso, si empezamos por el $1$, para fijar el segundo elemento solo tenemos 3 posibilidades, para el tercero 2 y para el último una. Por tanto
		\begin{align*}
		|cl((1234))| = 4
		\end{align*}
		
		\item Recordemos que el centralizador
		\begin{align*}
		C_{S_4}((1234)) = \{\sigma \in S_4 \mid \sigma (1234) \inv{\sigma} = (1234)\} < S_4
		\end{align*}
		
		\item Como $S_4$ tiene $|S_4| = 4! = 24$ y tenemos que $|cl((1234))| = [S_4 : C_{S_4}((1234))] = 6$ necesariamente $|C_{S_4}((1234))| = 4$.
		
		\item Nos proponemos calcular el grupo $C((1234))$. Un candidato para $\sigma \in C((1234))$ es $\sigma = (1234)$. En efecto $(1234)(1234)(1234) \in C((1234))$. Siempre ocurre que un elemento conmuta consigo mismo. Además, $\langle (1234) \rangle < C((1234))$ pero como $|\langle (1234) \rangle| = 4 = |C((1234))$ tiene que ocurrir que $\langle (1234) \rangle = C((1234))$. Es decir que de tipo 4 solo tenemos $(1234)$.
		
		\item ¿Qué tipos tenemos? Pues tantos como maneras de descomponer 4 en suma de números positivos, a saber
		\begin{itemize}
			\item (1234) de tipo 4
			\item (123) de tipo 3+1
			\item (12)(34) de tipo 2+2
			\item (12) de tipo 2+1+1
			\item $Id$ de tipo 1+1+1+1 (que es la única que tiene descomposición en 4 unos)
		\end{itemize}
		
		\item En general no es difícil calcular cuantos hay, por lo que a menudo utilizamos este argumento para calcular el grupo centralizador.
		
		\item Lo importante es que estamos descomponiendo $S_4$ de la siguiente manera:
		\begin{align*}
		S_4 &= cl((1234)) \cap cl((1223)) \cap cl((12)(34)) \cap cl((12)) \cap cl(Id) \\
		|S_4| &= |cl((1234))| \cap |cl((1223))| \cap |cl((12)(34))| \cap |cl((12))| \cap |cl(Id)|
		\end{align*}
		\item Ahora analizamos la clase $cl((123))$ de los ciclos de tipo 3+1. Lo primero es saber cuantos hay. Pues tenemos que elegir 3 elementos de entre 4 y luego ordenar los dos que nos quedan por tanto
		\begin{align*}
		|cl((123))| = \binom{4}{3} \times 2 = 8
		\end{align*}
		Por otro lado lo que sabemos es que $(123) \in C((123))$ (porque todos conmutan consigo mismos) y como antes $|C((123))| = 3$ (de la fórmula $|cl((123))| = [S_4:C((123))]$), luego $C((123)) = \langle (123) \rangle$.
		
		\item Igual es un poco más interesante la clase de tipo 2+2. \textbf{Pregunta de examen:} halla generadores del subgrupo centralizador del elemento (12)(34).
		\begin{itemize}
			\item Sabemos que el conjugado de un elemento de tipo 2 tiene que ser otro de tipo 2, por tanto tenemos que ver qué elementos distintos de tipo 2 tenemos. Pues fijamos el 1 por ejemplo y vemos qué parejas podemos hacer. Nos salen 3, a saber, 1 con 2, 1 con 3 y 1 con 4 de lo que concluímos que $|cl((12)(34))| = 3$.
			\item De la misma fórmula que antes sacamos que $|C((12)(34))| = 8$. De orden 8 sabemos que hay solo unos pocos grupos (ver la clasificación en \ref{gruposfinitosnotables}). Veamos con cuál de ellos es isomorfo.
			\item Como siempre sabemos que $(12)(34) \in C((12)(34))$. Tenemos que encontrar los demás $\tau$ que conmutan $\tau \sigma \inv{\tau} = \tau (12)(34) \inv{\tau} = (\tau(1)\tau(2))(\tau(3)\tau(4))$. Probamos con $\tau = (1324)$\footnote{La idea de probar con este viene de decir: pues a ver qué pasa si cambio el 1 con el 3 y el 2 con el 4, que nos daría la permutación (1324). En cualquier caso esto es prueba y error, y parar de probar cuando tengamos un grupo generado de orden 8.}.
			\begin{align*}
			(1324)&(12)(34)\inv{(1324)} \\
			&(34)(21)
			\end{align*}
			Que es el mismo, luego hemos probado que $\tau$ conmuta y por tanto $\tau \in C((12)(34))$. Lástima que no valga porque nos damos cuenta de que $\tau ^2 = (12)(34)$. Vaya. Drácula ha hecho chiste con esto y todo $(X,d)$.\footnote{Aquí se ve claramente que la elección del $\tau$ es casi al azar. Hemos elegido uno que prometía pero hemos tenido la mala suerte de que su cuadrado nos daba un elemento que suponíamos estaba en el grupo ($\tau^2 = (12)(34)$. Podríamos haber descartado este $\tau = (1324)$ pero hemos preferido descartar el elemento (12)(34) que sabíamos que estaba en el grupo. La razón de la sustitución de este último por el (12) es un misterio hasta la fecha.}
			
			Lo que hacemos es quitar el $(12)(34)$ y cambiarlo por el $(12)$. Para evitar $\tau^2 \neq (12)$. En resumen, ya tenemos $(12) \in C((12)(34))$ y $\tau = (1324) \in C((12)(34))$. Si vemos sus grupos generados:
			\begin{align*}
			\langle (1324)\rangle = \{(1324), (12)(23), (4321), Id\} \\
			\langle (12) \rangle = \{(12), Id\}
			\end{align*}
			La intersección de ambos subgrupos es solo la identidad y por la fórmula del producto libre averiguamos que $|\langle (1324)\rangle \langle (12) \rangle| = 8$ por lo $C((12)(34)) = \langle (1324), (12) \rangle$.
			
			Tiene toda la pinta de ser $D_4$ porque está generado por dos elementos, no es abeliano y los órdenes de los generadores son $o((1324)) = 4,\ o((12)) = 2$. Solo nos quedaría probar que se sigue cumpliendo la ecuación de la presentación de $D_4$:
			\begin{align*}
			BA = AB^3 \iff (1324)(12) = (12)(1324)^3
			\end{align*}
			Lo comprobamos y al final sale.
		\end{itemize}
		
		\item Ahora hacemos lo mismo con $C((12))$. Siguiendo un razonamiento similar, llegamos a que $C((12))$ es isomorfo con el grupo de Klein y por extensión con $\Z/2\Z \times \Z/2\Z$.
	\end{itemize}
\end{ej}


Falta la semana fatídica de Estadística

% 20181029
Vez pasada considerabamos $G_1 \times G_2$ y fijado un homomorfismo de grupos $\phi: G_1 \to Aut(G_2)$ hacíamos lo siguiente. En $G_1 \times_{\phi} G_2$ viven los elementos $(a,b) \times_{\phi} (c,d)$ donde la operación cambiaba en la primera coordenada $(a \phi_b(c), bd)$. Probamos la última clase que $G_1 \times_{\phi} G_2$ era un grupo (probar la asociatividad no es trivial).

% 20181030

Observación:

\begin{align*}
\gamma: G \xrightarrow{Int} Aut(G)
\end{align*}
$\gamma$ es un homomorfismo de grupos que lleva cada elemento $g \in G$ al automorfismo conjugación $\gamma_g(x) = gx\inv{g}$. Observamos que si $N \normsub G,\ \forall g \in G, \gamma_g(N) = gN\inv{g} = N$.

\begin{pro}
	$N$ es normal en $G$ ($N \normsub G$) sí y solo sí al restringir $\phi_g$ a $N$ la imagen es $N$:
	\begin{align*}
	G \xrightarrow{\gamma_g} G \\
	N \xrightarrow{\gamma_g \vert_N} N
	\end{align*}
	Es decir, que si $N$ es normal, $\gamma_g\vert_N$ induce un isomorfismo $\gamma_g\vert_N : N \to N$.
\end{pro}

\begin{proof}
	Cristalina de la definición de subgrupo normal.
\end{proof}

En general, al restringir $\gamma_g$ a un subgrupo de $G$ no tenemos esta propiedad.

Además, si $N \normsub G$ tiene sentido restringir $\gamma: G \xrightarrow{Int} Aut(G)$ a $Aut(N)$ y la restricción da un homomorfismo.

\section{Producto semidirecto}

Sea $G$ un grupo. Sea $N \normsub G$, $H < G$, $N \cap H = \{e\}$ y $NH = G$ (recordemos que por ser $N$ normal, $NH$ es grupo). Entonces $G \isom N \times H$.

Veamos quién es ese isomorfismo $\gamma : G \to N \times H$. Recordemos que considerando dos grupos $G_1, G_2$ y su producto directo $G_1 \times G_2$ existe un $\alpha : G_2 \to Aut(G_1)$. Veremos quien es este $\alpha$ para $H$ y $N$, es decir, quién es $\alpha: H \to Aut(N)$.

Construye $\alpha$ a partir de 4 isomorfismos.

\begin{proof}$ $\newline
	\begin{itemize}
		\item Comenzamos por definir una función $j: N\times H \to G,\ (n, h) \mapsto nh$. Es función está bien definida por teoría de conjuntos pero no es un homomorfismo de grupos\footnote{Ojo con por qué no es homomorfismo. Si tomamos $(n,h),(n', h') \in N \times H$ tenemos que $j((n,h)(n',h')) = nn'hh'$. Podríamos pensar que como $N$ es normal, podemos conmutarlo y obtener $nn'hh' = nhn'h' = j((n,h))j((n',h'))$. \textbf{Pero esto está mal.} Lo que significa ser normal es que para $h \in H$, se tiene que $nh = hn''$ para algún $n'' \in N$.}\footnote{Si los grupos son abelianos entonces sí es claro que es un homomorfismo. Lo que vamos a hacer es ver que dando una estructura especial, sí que es un homomorfismo de grupos incluso para grupos no abelianos}.
		\item Recordemos que por el teorema \ref{thm:cardinalidadproductolibre} tenemos que $|G| = |N||H| = |N \times H|$ por ser $N \cap H = \{e\}$.
		\item Volviendo a lo de la estructura especial. Dar una estructura especial es dar una operación para $N \times H$.
		\begin{itemize}
			\item Sea $A$ un conjunto. Es claro que si tenemos una biyección $\phi : A \to G$ podemos dotar a $A$ de alguna estructura para que sea un grupo.
			\item Para dotar a $A$ de estructura tenemos que definir la operación. Forzamos que para cada $a, a' \in A$ para los que se tiene $\phi(g) = a, \phi(g') = a'$ la operación sea $a a'  = \phi(gg')$.
			\item En este caso nuestro $A$ es $N \times H$. En lugar de utilizar la operación habitual del producto directo definimos otra operación. Para llegar a ella nos fijamos en $(n,h)(n',h') \mapsto nhn'h' = nhn'\inv{h}hh' = n(hn'\inv{h})hh' = nn'hh'$ (intercalamos el neutro, que es legal).
			\item Redefinimos la operación en $N \times H$ para que cuadre con este resultado. Llamaremos al nuevo grupo con la nueva operación $N \times_\phi H$: para $(n,h), (n',h')$ definimos $(n,h)\cdot (n',h') = (n(hn'\inv{h}), hh')$.
			\item Comprobamos que en este caso $j$ es un homomorfismo de grupos:
			\begin{align*}
			j : N \times_\phi H &\to G \\
			(n,h) &\mapsto nh \\
			(n',h') &\mapsto n'h' \\
			(n,h)\cdot(n',h') &\mapsto n(hn'\inv{h})hh' = nn'hh'
			\end{align*}
		\end{itemize} 
	\end{itemize}
\end{proof}

Es muy interesante por que es muy natural llegar a situaciones de esta manera. ¡Y les voy a dar una!\footnote{Sugerencia: leelo con voz de tomatito.}

\begin{ej}
	Sea $|G| = p \cdot q$ y supongamos $p < q$ primos. Por el teorema de Lagrange (\ref{thm:lagrange}) tenemos que existe un subgrupo $H_p < G$ con $|H_p| = p$ y análogamente $\exists H_q \mid |H_q| = q$. A primera vista podríamos pensar que puede haber varios grupos de orden $q$. Pues no.
\end{ej}

\begin{proof}
	Supongamos hay dos grupos $H, H'$ de orden $q$ distintos. La intersección tiene que dar un subgrupo y si los dos grupos tienen un número primo de elementos entonces la intersección solo puede ser el neutro, $H \cap H' = \{e\}$. Entonces por el teorema \ref{thm:cardinalidadproductolibre} tenemos que $|HH'| = q^2 > p\cdot q$ lo que es imposible. Luego sabemos que a lo sumo hay un grupo de orden $q$.
\end{proof}

(Sigue el ejemplo) Supongamos que ese único grupo de orden $q$ se llama $N$. Entonces $\phi_g(N) = N$ ya que un isomorfismo tiene que mandar un subgrupo de $q$ elementos en otro subgrupo de $q$ elementos y $N$ es el único. Por tanto $N \normsub G$. Aplicando el teorema de antes, tenemos que $G \isom N \times H$.

\begin{ej}
	Veamos un ejemplo con más pinta de problema. Demostrar que todo grupo de orden $77$ es cíclico.
\end{ej}

\begin{proof}
	Comenzamos por observar que $77 = 7 \cdot 11$. Por el teorema de Lagrange (\ref{thm:lagrange}) tenemos que existen $H, N < G \mid |H| = 7,\ |N| = 11$ y por lo visto en el ejemplo anterior, $N \normsub H$. Como antes llegamos a que $H \cap N = \{e\}$ y a que $|H N| = pq$. Para aplicar el teorema anterior vemos qué estructura tiene que tener $N \times_\phi H$, con $\phi:H\longrightarrow Aut(N)$.
	\\\\
	Vemos que $Aut(N) = Aut(\mathbb{Z}/11\mathbb{Z}) = \mathcal{U}(\mathbb{Z}/11\mathbb{Z}) = \mathbb{Z}/10\mathbb{Z}$, es decir, un grupo cíclico de 10 elementos.
	\\\\
	Entonces, $\phi$ es de la forma: $H = \mathbb{Z}/7\mathbb{Z} \longrightarrow Aut(\mathbb{Z}/11\mathbb{Z})$, por tanto, solo podemos definir el homomorfismo de grupos trivial. Esto hace que $N \times_\phi H$ es igual a $\mathbb{Z}/7\mathbb{Z} \times \mathbb{Z}/11\mathbb{Z}$.\\\\Por el corolario \ref{cor:propiedadesgrupdirecto} sabemos que $G \cong N \times_\phi H \implies G\cong \mathbb{Z}/7\mathbb{Z} \times \mathbb{Z}/11\mathbb{Z}$ que es cíclico por ser producto de cíclicos de órdenes coprimos.
\end{proof}

% !TeX root = ../apuntes-ea.tex

\chapter{Teoremas de Sylow}

% 20181031

Son muchos teoremas para grupos finitos en los que el orden se puede expresar como
\begin{align}
	|G| = p^s m,\ mcd(p, m) = 1, s \geq 1
\end{align}
Veremos y discutiremos algunos de ellos.

\begin{thm}
	[Primero de Sylow]
	Sea $G$ un grupo tal que $|G| = p^s m,\ mcd(p, m) = 1, s \geq 1$. Entonces existe $H < G$ con $|H| = p^s$.
\end{thm}

Hemos hecho mucho hincapié en los subgrupos normales y tenemos que si $N \normsub G$ entonces existe $\pi:G \to G/N$ homomorfismo de grupos\footnote{Por teoría de conjuntos tenemos que $\pi$ es una función que existe y está bien definida, pero aquí interesa que además es homomorfismo.}. Además teníamos que $|G| = |G/N| \cdot |N|$.

También establecíamos una biyección entre los submódulos de $G$ que contienen a $N$ y los submódulos de $G/N$. Si $K$ es uno de ellos entonces $N \normsub G \implies N \normsub K$,
\begin{align*}
	K/N = \overline{K} \subset K/N \\
	|K| = |\overline{K}||N|
\end{align*}

Vamos a discutir el teorema. Recordemos que dado $G$ el centro $Z(G)$ es el conjunto de los elementos que conmutan con todos (ver definición \ref{dfn:centro}). Recordamos además las proposiciones \ref{pro:centronormal} y \ref{pro:subcentronormal} que nos dicen que el centro es normal y que cualquier subgrupo del centro es abeliano y normal. El centro está bien pero tampoco es para tanto: suele ser muy pequeño. WTF.

% TODO restate theorems

Aquí en medio ha desvariado bastante, remontándose hasta el teorema \ref{thm:correspondenciasubgrupos}.

\begin{proof}[Demostración del teorema de Sylow]
	Procedemos por inducción [fuerte] en $|G|$.
	\begin{itemize}
		\item Si $|G| = 1$ no hay mucho que probar porque son grupos muy tontos.
		\item Suponemos que\footnote{[La clase en silencio]. \textit{Orlando: Se pueden callar por favor.} [El silencio se hace más hueco]. \textit{Orlando: No hagan ruiditos. Me cuesta concentrarme} [agita las manos]. [Sigue la demostración.]} el teorema es válido para $|G| < n$. Distinguimos los siguientes casos:
		\begin{enumerate}
			\item $|Z(G)| = 0$
			\item $|Z(G)| \neq 0$. Entonces $Z(G)$ es un grupo abeliano no trivial. Es decir que $Z(G) \isom \Z/n_1\Z \times \dots \times \Z/n_l\Z$. Como $p$ divide a $|Z(G)|$ podemos suponer que $p$ divide a $n_1$. Entonces $\overline{(n_1/p)} \in \Z/n_1\Z$ y por tanto
			\begin{align*}
				(\overline{\left(\frac{n_1}{p}\right)}, \overline{0}, \dots, \overline{0}) \text{ tiene orden } p
			\end{align*}
			Es decir que tenemos un $H < Z(G)$ con $|H| = p$.
			
			Teníamos de antes que $|G/H| |H| = |G|$. Por inducción existe $\overline{K} < G/H$ de orden $p^{s-1}$. Aplicamos $|K| = |\overline{K}||H|$ y como $|H| = p,\ |\overline{K}| = p^{s-1}$ tenemos que $|H| = p^s$.
		\end{enumerate}
	\end{itemize}

	Lo hemos probado para una hipótesis en concreto pero falta algo (no sé el qué). Seguimos con la demostración.
	\begin{align*}
		|G| = |Z(G)| + [G:C(a_{s+1})] + \dots  + [G:C(a_r)]
	\end{align*}
	$|G|$ es no nulo módulo $p$ y $|Z(G)$ es nulo módulo $p$, por lo que necesariamente tiene que ocurrir que alguno de los $[G:C(a_i)]$ sea no nulo módulo $p$. Supongamos que es el primero, es decir, supongamos que $[G:C(a_{s+1})]$ es no nulo módulo $p$. Además tenemos que
	\begin{align*}
		\underbrace{|G|}_{p^sm} = \underbrace{|C(a)|}_{p^sm'}\cdot \underbrace{[G:C(a)]}_{\text{ no divisible por p}}
	\end{align*}
	Como $[G:C(a)] \geq 2,\ |C(a)| = p^sm' < |G|$ por inducción el subgrupo $C(a_{s+1})$ tiene un subgrupo de orden $p^s$.
\end{proof}


% !TeX root = ../apuntes-ea.tex

\chapter{Anillos}


\begin{dfn}[Anillo]
	Un anillo es una terna $(A, +, \cdot)$ donde $+$ es una operación a la que llamamos suma, $\cdot$ es otra operación a la que llamamos producto y se verifican las siguientes propiedades
	\begin{enumerate}
		\item El par $(A, +)$ es un grupo abeliano
		\item El producto $\cdot$ es asociativo
		\item Se cumplen las propiedades distributivas:
		\begin{align}
			\forall a, b , c \in A,\ a\cdot (b + c) = a\cdot b + a \cdot c \\
			\forall a, b , c \in A,\ (a + b) \cdot c = a\cdot c + b \cdot c
		\end{align}
	\end{enumerate}
\end{dfn}

Con la operación $+$ tenemos las siguientes propiedades
\begin{enumerate}
	\item Asociatividad: $(a+b)+c = a+(b+c)$
	\item Elemento neutro aditivo: $\exists 0 \in A \mid 0+a = a$
	\item Elemento inverso aditivo: $\forall a \in A, \exists -a \in A \mid a + (-a) = 0$
	\item Conmutatividad aditiva: $\forall a, b \in A,\ a+b = b+a$
\end{enumerate}

Con la operación $\cdot$ tenemos las siguientes propiedades
\begin{enumerate}
	\item Asociatividad: $a\cdot (b \cdot c) = (a \cdot b) \cdot c$
	\item Elemento neutro multiplicativo: $\exists 1 \in A \mid a\cdot 1 = 1 \cdot a = a$
	\item No siempre existe inverso multiplicativo: $\inv{a} \mid a\cdot \inv{a} = 1$
	\item No siembre se da la conmutatividad multiplicativa: $a \cdot b = b\cdot a$
\end{enumerate}

\begin{dfn}[Unidades en anillos]
	Dado $(A, +, \cdot)$ anillo. El grupo de unidades es
	\begin{align}
		\uds{A} = (\{a \in A \mid \exists \inv{a} \in A,\ a\cdot \inv{a} = 1\}, \cdot)
	\end{align}
	Los elementos del grupo de unidades se llaman elementos invertibles.
\end{dfn}

\begin{ej}
	Las matrices cuadradas $2\times 2$ con coeficientes reales: $(M_{2\times 2}(\R), +, \cdot)$ es un anillo. Tiene unidades $\uds{A} = (GL_2(\R), \cdot)$
\end{ej}

\begin{ej}
	Los numeros enteros $(\Z, +, \cdot)$ es un anillo y tienen unidades $\uds{\Z} = (\{-1, 1\}, \cdot)$
\end{ej}

\begin{pro}
	Sea $-1$ el inverso aditivo del neutro multiplicativo $1$. Entonces $\forall a \in A$ el inverso aditivo es $-a = -1 \cdot a$ y se tiene $-1\cdot a + a = 0$.
\end{pro}

\begin{pro}
	Sea $A$ un anillo. El neutro aditivo $0$ verifica $0 \not\in \uds{A}$
\end{pro}

\begin{dfn}[Anillo conmutativo]
	Sea $A$ un anillo. $A$ es un anillo conmutativo $\iff \forall a, b \in A,\ a\cdot b = b \cdot a$.
\end{dfn}

\begin{pro}[Propiedad cancelativa]
	Sea $a \in \uds{A}$. Entonces $\forall b,c$ se tiene $b, c \in A \implies a\cdot b = a\cdot c \implies b = c$
\end{pro}

\begin{dfn}[Divisor de 0]
	Sea $(A, +, \cdot)$ un anillo. Diremos que $a \in A$ es divisor de $0 \iff a \neq 0 \land \exists 0 \neq b \in A \mid a\cdot b = 0$
\end{dfn}

\begin{ej}
	En $\Z/8\Z$ el elemento $\overline{2}$ tiene dimensión 0.
\end{ej}

\begin{pro}
	Sea $A$ un anillo. $\forall a \in A$ no divisor de 0 $\implies$ se cumple la propiedad cancelativa.
\end{pro}

\begin{proof}
	$ab = ac \implies b = c \iff ab + (-ac) = a(b -c) = 0$
\end{proof}

\begin{dfn}[Dominio de integridad]
	Un anillo que no tiene elementos divisores de 0 se llama dominio de integridad (DI).
\end{dfn}

\begin{ej}
	\begin{itemize}
		\item $\Z$ es un dominio de integridad ya que todo $a \in \Z, a \neq 0$ tiene un inverso multiplicativo $\inv{a}$.
		\item $\Z/p\Z$ con $p$ primo es un dominio de integridad.
		\item $\Z/n\Z$ con $n$ no primo no es un dominio de integridad ya que si $\overline n = ab$ con $a \neq n \land b \neq n$ se tiene $\overline{a} \cdot \overline{b} = \overline{n} = \overline{0}$ con $\overline{a} \neq 0 \land \overline{b} \neq 0$.
	\end{itemize}
\end{ej}

% -------------

\begin{thm}
	Dado el anillo $A$ y un ideal propio $I$
	\begin{align*}
		\pi: A \to A/I,\qquad I \subset \inv{\pi}(\overline{J}) \subset A,\qquad \overline{0} \in \overline{J} \subset A / I
	\end{align*}
	
	existe una identificación entre el retículo de ideales $A / I$ con el subretículo de ideales de $A$ que contienen a $I$. 
	
	Es decir, si $J$ es un ideale en $A/I$ entonces $\inv{\pi}(\overline{J})$ es un ideal en $A$ que contiene al ideal $I$.
\end{thm}


El ideal cero de $A/I$ tiene contraimagen $\inv{\pi}(\{0\}) = I$. Si $\overline{J}$ es un ideal en $A/I$
\begin{align*}
	\pi : A \to A/I \to (A/I) / \overline{J}
\end{align*}

es un homomorfismo de anillos (la composición de homomorfismos de anillos es un homomorfismo de anillos). $\inv{\pi}(\overline{J}) = \ker$ de la composición.

% --------------------

\begin{thm}
	Sea $\alpha: A \to B$ un homomorfismo de anillos.
	\begin{itemize}
		\item $\ker \alpha$ es un ideal
		\item $\ima \alpha$ es un subanillo
		\item $\alpha$ es sobreyectivo $\iff \ima \alpha = B$
		\item $\alpha$ es inyectivo $\iff \ker \alpha = \{0\}$
	\end{itemize}
\end{thm}

\begin{dfn}
	Un homomorfismo de anillos $\alpha: A \to B$ es un isomorfismo cuando es una biyección. En este caso decimos que $A$ y $B$ son isomorfos y lo notamos con $A \isom B$.
\end{dfn}

\begin{pro}
	Si $\alpha: A \to B$ es un homomorfismo de anillos y una biyección de conjuntos entonces $\inv{\alpha}:B \to A$ es nuevamente un homomorfismo de anillos.
\end{pro}

\subsubsection{Homomorfismos de anillos e ideales}

\begin{thm}
	Sea $\alpha: A \to B$ un homorfismo de anillos. Entonces
	\begin{enumerate}
		\item Si $J \subset B$ es un ideal en $B$ entonces $\inv{\alpha}(J)$ es un ideal en $A$.
		\item Si $\alpha$ es sobreyectiva entonces la imagen $\alpha(I)$ de un ideal $I \subset A$ es un ideal en $B$
	\end{enumerate}
\end{thm}

\begin{proof}
	\begin{figure}[h]
		\centering
		\begin{tikzpicture}
		\node (A) at (0,0) {$A$};
		\node (B) at (4,0) {$B$};
		\node (BJ) at (4, -3) {$B / J$};
		
		\draw[-{Latex[length=2mm]}] (A) -- (B) node[pos=.5, above] {$\alpha$};
		\draw[-{Latex[length=2mm]}] (B) -- (BJ) node[pos=.5, left]{$\pi$};
		\draw[-{Latex[length=2mm]}] (A) -- (BJ) node[pos=.5, below] {$\pi \circ \alpha$};
		\end{tikzpicture}
	\end{figure}
	\begin{enumerate}
		\item $\inv{\alpha}(J) = \ker(\pi \circ \alpha)$ y por tanto es un ideal.
		\item Probamos las propiedades de los ideales:
		\begin{enumerate}
			\item $\alpha(0) = 0 \in \alpha(I)$
			\item Sean $b_1, b_2 \in \alpha(I)$ tenemos que ver que $b_1 + b_2 \in \alpha(I)$. Sean $a_1, a_2 \in I$ tales que $b_1 = \alpha(a_1) \land b_2 = \alpha(a_2)$. Por ser $\alpha$ h. de anillos tenemos que $b_1 + b_2 = \alpha(a_1 + a_2) = \alpha(a_1) + \alpha(a_2)$.
			\item Sean $b \in B,\ b' \in \alpha(I)$. Tenemos que probar que $bb' \in \alpha(I)$. Sabemos que $b' \in \alpha(I) \iff b' = \alpha(a),\ a \in I$. Como $b \in B$ y $\alpha$ es sobre tiene que existir $d \in I \mid \alpha(d) = b$. Por tanto $\alpha(d\cdot a) = b \cdot b' \implies bb' \in \alpha(I)$.
		\end{enumerate}
	\end{enumerate}
\end{proof}

Fijado $I \subset A$ consideramos $\pi:A \to A/I$ que es un homomorfismo de anillos sobreyectivo.

\begin{enumerate}
	\item Si $\overline{J} \subset A /I$ es un ideal en $A / I$ entonces $\inv{\pi}(\overline{J})$ es un ideal en $A$ que contiene a $I$.
	\item Si $J$ es un ideal en $A$ entonces $\pi(J)$ es un ideal en $A/J$ y $J \subseteq \inv{\pi}(\pi(J))$ (es claro porque si $j \in J$ entonces $\pi(j) \in \pi(J)$).
	\begin{enumerate}
		\item Además, si $I \subseteq J$ entonces $J = \inv{\pi}(\pi(J))$.
		\begin{proof}
			Si $\delta \in \inv{\pi}(\pi(J)) \implies \delta \in J$. Además, $\delta \in \inv{\pi}(\pi(J)) \iff \pi(\delta) \in \pi(J) \iff \pi(\delta) = \pi(d_1),\ d_1 \in J \iff \delta - d_1 \in \ker \pi = I$. Tomamos
			\begin{align*}
				\delta = \underbrace{(\delta - j_i)}_{\in I} + \underbrace{j_i}_{\in J} \in J
			\end{align*}
			porque $I \subset J$.
		\end{proof}
	\end{enumerate}
\end{enumerate}


La siguiente proposición nos llevará al primer teorema de la isomorfía.
\begin{pro}
	Sea $\varphi: A \to B$ un homomorfismo de anillos con $\ker \varphi$ ideal en $A$. Sea $I$ un ideal en $A$ con $I \subset \ker \varphi$.
	\begin{itemize}
		\item Existe un único homomorfismo de anillos $\overline{\varphi}: A / I \to B$ tal que $\varphi = \overline{\varphi} \circ \pi$.
		\begin{figure}[h]
			\centering
			\begin{tikzpicture}
			\node (A) at (0,0) {$A$};
			\node (B) at (4,0) {$B$};
			\node (AI) at (0, -3) {$A / I$};
			
			\draw[-{Latex[length=2mm]}] (A) -- (B) node[pos=.5, above] {$\varphi$};
			\draw[-{Latex[length=2mm]}] (A) -- (AI) node[pos=.5, left]{$\pi$};
			\draw[-{Latex[length=2mm]}] (AI) -- (B) node[pos=.5, below] {$\overline{\varphi}$};
			\end{tikzpicture}
		\end{figure}
		\begin{proof}
			Definimos $\overline{\varphi}(\overline{a}) = \varphi(a)$. Aunque choque (porque el $\overline{a}$ puede venir de muchos $a$) aseguramos que $\overline{\varphi}$ está bien definida. Veamos por qué. Sabemos que $a'$ y $a$ definen el mismo elemento en $A / I \iff a' - a \in I$. Sopongamos que $I \subset \ker \varphi$. Entonces $\varphi(a - a') = 0 \iff \varphi(a) - \varphi(a') = 0 \implies \overline{\varphi}$ está bien definida como función.
			
			Veamos ahora que en efecto se cumple que $\overline{\varphi}$ es un homomorfismo de anillos, es decir que $\overline{\varphi}(\overline{a} + \overline{b}) = \overline{\varphi}(\overline{a}) + \overline{\varphi}(\overline{b})$. Recordando la definición que hemos dado de $\varphi$ y la propiedad $\overline{a} + \overline{b} = \overline{a + b}$ es claro que $\overline{\varphi}(\overline{a} + \overline{b}) = \overline{\varphi}(\overline{a +b}) = \varphi(a + b) = \varphi(a) + \varphi(b) = \overline{\varphi}(\overline{a}) + \overline{\varphi}(\overline{b})$. Es análogo para el producto ya que $\overline{a} \cdot \overline{b} = \overline{a \cdot b}$.
		\end{proof}
		\item $\ker \overline{\varphi} = \ker \varphi / I$
		\begin{proof}
			Sea $\overline{a} \in A / I$. Entonces $\overline{a} \in \ker \overline{\varphi} \iff \overline{\varphi}(\overline{a}) = 0 \iff \varphi(a) = 0 \iff a \in \ker \varphi$.
		\end{proof}
	\end{itemize}
\end{pro}

\begin{thm}[Primer teorema de la isomorfía (anillos)]
	Si $\alpha: A \to B$ es un homomorfismo de anillos sobreyectivo entonces $B \isom A / \ker \alpha$.
\end{thm}

	\begin{figure}[h]
	\centering
	\begin{tikzpicture}
	\node (A) at (0,0) {$A$};
	\node (B) at (4,0) {$B$};
	\node (AI) at (0, -3) {$A / \ker \alpha$};
	
	\draw[-{Latex[length=2mm]}] (A) -- (B) node[pos=.5, above] {$\alpha$};
	\draw[-{Latex[length=2mm]}] (A) -- (AI) node[pos=.5, left]{$\pi$};
	\draw[-{Latex[length=2mm]}] (AI) -- (B) node[pos=.5, below] {$\overline{\alpha}$};
	\end{tikzpicture}
\end{figure}

\begin{proof}
	Nos apoyamos en la proposición anterior tomando $I = \ker \alpha$. Como $\alpha$ y $\pi$ son sobreyectivas tenemos que $\overline{\alpha}$ es sobreyectiva. Aplicando el segundo resultado de la proposición anterior tenemos que $\ker \overline{\alpha} = \ker \alpha / \ker \alpha = \{ 0\} \implies \overline{\alpha}$ es inyectiva. Concluimos que $\overline{\alpha}$ es un isomorfismo de anillos y por tanto $B \isom A / \ker \alpha$.
\end{proof}

% ------- 20181217

\begin{thm}
	\begin{align*}
	D \text{ es un dominio de ideales principales (DIP) } \implies D \text{ es un dominio de factorización única (DFU)}
	\end{align*}
\end{thm}

El recíproco de este teorema no es cierto en general. Véase por ejemplo el caso de $\Z$ que es un dominio de ideales principales pero no se cumple que $\Z[X]$ es un dominio de factorización única. Si se cumpliera el recíproco entonces el siguiente teorema sería un simple corolario.

\begin{thm}
	\begin{align*}
	D \text{ es un dominio de factorización única (DFU) } \implies D[X] \text{ es un dominio de factorización única (DFU)}
	\end{align*}
\end{thm}

Este segundo teorema no lo vamos a probar. Probamos el primero.

\begin{dfn}[Asociados]
	Sea $D$ un domino, $a,a' \in D$. DIremos que $a$ y $a'$ son asociados $\iff \exists u \in \uds{D} \mid a = u a'$.
\end{dfn}

\begin{proof}
	Sea $D$ un dominio, $a \in D \mid a \neq 0 \land a \not\in \uds{D}$. Sabemos que $a, a' \in D$ son asociados si $\exists u \in \uds{D} \mid a = ua'$. Por ejemplo, los polinomios $3x-2$ y $x - 2/3$ en $\Q[X]$ son asociados.
	
	Observemos que si $a$ y $a'$ son asociados entonces $\gen{a} = \gen{a'}$. Si $u \in \uds{a}$ entonces $ua' = a \in \gen{a'}$. Análogamente $\inv{u}a = a' \in \uds{a}$. Luego tenemos $\gen{a} \subset \gen{a'} \land \gen{a'} \subset \gen{a} \implies \gen{a} = \gen{a'}$. Recíprocamente si $0 \neq \gen{a} = \gen{a'} \implies \exists u \in \uds{D} \mid a = ua'$. $a \in \gen{a'} \land a' \in \gen{a} \implies a = a't \land a' = as \implies a' = a'ts \implies 1 = ts \implies t,s \in \uds{D}$.
	
	Recordemos las hipótesis iniciales: $a \in D \mid a \neq 0 \land a \not\in \uds{D}$. Esto nos da que $0 \neq \gen{a} \land \gen{a} \subsetneq D$. Pensemos en qué significa que un elemento no nulo $a$ no sea una unidad. Supongamos $a = st$. Si $a$ no es una unidad podría ocurrir que $s$ es una unidad (por ejemplo $6 = (-1)(-6),\ -1 \in \uds{\Z}$). Lo que sí que está claro es que no puede ocurrir que a la vez $s$ y $t$ sean unidades. Es decir, tiene que ocurrir que al menos uno de los dos no es una unidad. Por tanto podemos suponer sin pérdida de generalidad que si expresamos $a = a' \cdot s$  entonces $a' \not \in \uds{D}$. Tenemos dos situaciones posibles
	\begin{enumerate}
		\item $s \in \uds{D} \implies \gen{a} = \gen{a'}$
		\item $s \not \in \uds{D} \implies \gen{a} \subsetneq \gen{a'}$ ya que $\gen{a} = \gen{a'} \iff a = a'u$ con $u \in \uds{D}$ pero hemos tomado $s \not\in \uds{D}$
	\end{enumerate}
\end{proof}

Aquí para de demostrar y empieza a dar definiciones.

\begin{dfn}
	Sea $D$ un dominio y $0 \neq a \not\in \uds{D}$. Diremos que $a$ es irreducible en $D \iff \forall a',s \in D,\ a' \not \in \uds{D},\ a = a's \implies s \in \uds{D}$
\end{dfn}

\begin{obs}
	Un elemento es irreducible $\iff$ cualquier asociado lo es.
\end{obs}

\begin{dfn}
	Sea $D$ un dominio. Diremos que $D$ es un dominio de factorización única (DFU) si se cumplen las siguientes condiciones $\forall a \in D$:
	\begin{itemize}
		\item $a \neq 0 \land a \not\in\uds{D} \implies a = p_1p_2\dots p_r$ donde $p_i$ es irreducible en $D$
		\item $a = p_1p_2\dots p_r,\ p_i$ irreducible y $a = q_1q_2 \dots q_s,\ q_i$ irreducible $\implies r = s$ y además $r_i$ y $q_i$ son asociados para $i = 1, \dots, r$ (la igualdad es un caso particular de el ser asociados).
	\end{itemize}
\end{dfn}

\begin{obs}
	Sea $I_1 \subseteq I_2 \subseteq I_3 \subseteq ...$ una cadena creaciente de ideales de un anillo $A$. Entonces $\bigcup I_i$ es un ideal.\footnote{Literalmente ha dicho que esto no viene a cuento. Que esto es una digresión de las suyas.}
\end{obs}

\begin{proof}
	Probamos las propiedades de los ideales.
	\begin{enumerate}
		\item $0 \in \bigcup I_i$
		\item $s,t \in \bigcup I_i \implies s+t \in \bigcap I_i$
		\item $s \in \bigcup I_i,\ a \in A \implies as \in \bigcup I_i$.
	\end{enumerate}
\end{proof}

\begin{dfn}
	[Propiedad de cadena creciente]
	
	Diremos que un anillo $A$ tiene la propiedad de cadena creciente $\iff$ toda cadena creciente $I_1 \subseteq I_2 \subseteq I_3 \subseteq \dots \subseteq I_n \subseteq \dots$ es finita. Es decir, que $\exists n \mid I_n = I_{n+1} = I_{n+2} = \dots$.
\end{dfn}

\begin{thm}
	Si $D$ es un DIP entonces $D$ tiene la propiedad de cadena creciente.
\end{thm}

La demostración es tan ingenua como uno quiera.

\begin{proof}
	Sea $I_1 \subseteq I_2 \subseteq I_3 \subseteq \dots \subseteq I_n \subseteq \dots$ una cadena de ideales. Sabemos que en cualquier anillo $\bigcup I_i$ es un ideal. Sea $J = \gen{d}$ para algún $d \in D$. Como $D$ es un DIP ocurre que $d \in \bigcup I_i \implies d \in I_{n_0} \implies \gen{d} \subset I_{n_0} \implies I_{n_0} = I_{n_0 + 1} = \dots$
\end{proof}




\part{Apendices}
\renewcommand{\thechapter}{\Alph{chapter}}
\setcounter{chapter}{0}



% !TeX root = ../apuntes-ea.tex

\chapter{Ejercicios}

\section{Hoja 1}

\begin{ex}[H1.2]
	Sean $a,b,c \in G = (-1,1)$. Probamos las propiedades de los grupos.
	\begin{itemize}
		\item \textbf{Asociatividad:}
		\begin{align*}
			(a \ast b) \ast c = \left(\frac{a+b}{1+ab}\right) \ast c = \frac{\left(\frac{a+b}{1+ab}\right) + c}{1 + \left(\frac{a+b}{1+ab}\right)c} = \frac{\frac{a+b+c+abc}{1+ab}}{\frac{1+ab+ac+bc}{1+ab}} = \frac{a+b+c+abc}{1+ab+ac+bc} \\
			a \ast (b \ast c) = a \ast \left(\frac{b+c}{1+bc}\right) = \frac{a+\left(\frac{b+c}{1+bc}\right)}{1+a\left(\frac{b+c}{1+bc}\right)} = \frac{\frac{a + b + c + abc}{1+bc}}{\frac{1+ab+ac+bc}{1+bc}} = \frac{a+b+c+abc}{1+ab+ac+bc}
		\end{align*}
		\item \textbf{Elemento neutro:} es el $0$ ya que $x\ast 0 = \frac{x+0}{1+x\cdot 0} = \frac{x}{1} = x$ y además $0 \ast x = \frac{0+x}{1+0\cdot x} = \frac{x}{1} = x$
		\item \textbf{Elemento inverso:} la ecuación
		\begin{align*}
			x \ast \inv{x} = 0 \iff \frac{x + \inv{x}}{1+x\inv{x}} = 0 \iff \inv{x} = -x
		\end{align*}
		siempre tiene solución y ocurre lo mismo para la ecuación $\inv{x} \ast x = 0 \iff \inv{x} = -x$
		\item \textbf{Clausura:} tenemos que probar que si $x,y \in (-1,1)$ entonces $x \ast y \in (-1,1)$. Consideramos $f(x,y) = x \ast y = \frac{x+y}{1+xy}$. Derivando tenemos que $\nabla f(x,y) = (\frac{1}{(1+xy)^2},\ \frac{1}{(1+xy)^2}) \neq 0, \forall x,y \in [-1,1]\times[-1,1]$. Si el máximo no se alcanza en ningún sitio de dentro del cuadrado $(-1,1)\times(-1,1)$ se tendrá que alcanzar en el borde.
		\begin{itemize}
			\item Fijado $x = 1$ tenemos que $f(1,y) = \frac{1+y}{1+y} = 1 \implies f(1, -1 < y < 1) < 1$ porque si $f(1, -1 < y < 1)$ tomara un valor mayor que $1$ habría un máximo en $(-1,1)\times(-1,1)$ y esto no puede ser pues $\nabla f$ no se anula en el cuadrado.
			\item Fijado $x = -1$ tenemos que $f(-1, y) = \frac{y-1}{1-y} = -1 \implies f(-1, -1 < y < 1) > -1$ por la misma razón que antes.
			\item Hacemos lo mismo fijando la $y$ y variando la $x$.
		\end{itemize}
		En el borde (que no está incluido) se alcanzan máximo y mínimo que acotan a $f$ en el cuadrado:
		\begin{align*}
			-1 < f(x,y) = x \ast y < 1,\quad \forall x, y \in G
		\end{align*}
	\end{itemize}
\end{ex}

\begin{ex}[H1.3]Hallar los inversos de los siguientes elementos, cada uno en su grupo correspondiente:
	\begin{enumerate}
		\item $o(\overline{11})$ en $\uds{\Z^\ast/23\Z}$ es $22$ porque $23 \cdot 22 \equiv 1 \mod 23$
		\item $o(\overline{5})$ en $\uds{\Z^\ast/31\Z}$ es $3$ porque $5 \cdot 3 \equiv 1 \mod 31$
	\end{enumerate}
\end{ex}

\begin{ex}[H1.33]
	\label{ex:h1.33}
	Sea $G$ un grupo. Suponed que existe un único $a \in G$ de orden 2. Demostrad que $a \in Z(G)$.
\end{ex}

\begin{proof}
	Recordamos que $a \in Z(G) \iff ga = ag,\ \forall g \in G$. Definimos el isomorfismo de conjugación $\phi_g (x) = gx\inv{g}$ para algún $g$. Como $\phi_g$ es isomorfismo lleva elementos de orden $n$ en elementos de orden $n$. Entonces $\phi_g(a) = a$ ya que $a$ es el único elemento de orden 2. Por tanto $ga\inv{g} = a \implies ga = ag \implies a \in Z(G)$.
\end{proof}

\section{Hoja 2}

\begin{ex}[H2.1]
	\label{ex:h2.1}
	Se considera el tercer grupo diédrico $D_3$. Se pide hallar lo siguiente:
	\begin{enumerate}
		\item Las clases de conjugación de cada uno de sus elementos.
		\begin{proof}
			Las clases dan una partición del grupo. Si un elemento pertenece a una clase, entonces la clase de ese elemento también es la clase a la que pertenece.
			\begin{itemize}
				\item $cl(e) = \{e\}$
				\item $cl(B)$? Sabemos que $|cl(B)| = [G:C(B)]$. Sabemos que $\gen{B} =\{1, B, B^2\} \subset C(B)$ luego $|C(B)| \geq 3$. Si hubiera más elementos en $C(B)$ tendríamos que $|C(B)| = 6$ pues $C(B) < D_3$. Esto no ocurre porque sabemos que $B$ no conmuta con todos los demás elementos. Por ejemplo $BA \neq AB$. Por tanto $|C(B)| = 3 \implies |cl(B)| = [D_3:C(B)]= 6/3 = 2$. Es claro que $B \in cl(B)$. Además, como $cl(B)$ contiene elementos transformados por el isomorfismo conjugación sabemos que el otro elemento que hay tiene orden 3. El único elemento que queda de orden 3 es $B^2 \implies cl(B) = \{B, B^2\}$.
				
				\item $cl(A)?$ Sabemos que $A$ no conmuta con todos ($A \not\in Z(D_3)$) luego $|C(A)| < 6$. Sabemos que $\gen{A} = \{1, A\} < C(A)$. Además, como $C(A)$ es un (sub)grupo sabemos que no puede haber más elementos porque si los hubiera, $|\gen{A}| \divides |C(A)| \implies C(A) \geq 6$ pero ya hemos visto que no puede ser. Es decir que $|cl(A)| = [D_3:D(A)] = 6 / 2 = 3$. Por tanto $cl(A)$ incluye los 3 elementos que nos quedan: $cl(A) = \{A, AB, AB^2\}$.
			\end{itemize}
		\end{proof}
	
		\item Los elementos de $\text{Int}(D_3)$.
		\item Los centralizadores $C_{D_3}(x)$ para cada $x \in D_3$
		\item Los normalizadores $N(H)$ para cada $H < D_3$.
	\end{enumerate}
\end{ex}

\begin{ex}[H2.2]
	
	\begin{proof}
		Obtenidas las clases en el ejercicio \nameref{ex:h2.1} se verifica que $|D_3| = |cl(e)| + |cl(B)| + |cl(A)| = 1 + 2 + 3 = 6$
	\end{proof}
	
\end{ex}

\begin{ex}[H2.6]
	Sea $G$ un grupo. ¿Verdadero o falso?
	\begin{enumerate}
		\item $H < G$ y $H$ conmutativo implica $H \normsub G$.
		\item $H < G$ y $|H| = 2$ implica $H \normsub G$.
		\item Si $\varphi: G \to G_1$ es un homomorfismo de grupos, entonces $\ima \varphi \normsub G$
		\item Si $H \normsub K$ y $K \normsub G$ entonces $H \normsub G$
		\item Si $H \normsub G$ y $|H| = m$ entonces $H$ es el único subgrupo de $G$ de orden $m$.
		\item Si $H \normsub G$ entonces $H < Z(G)$.
		\begin{proof}[FALSO]
			Contraejemplo: En $G = D_4$ tomamos $H = \gen{B^2} = \{1, B, B^2, B^3\} \not\subset Z(D_4) = \{1, B^2\}$.
		\end{proof}
	\end{enumerate}
\end{ex}

\begin{ex}[H2.10]
	\begin{proof}
		Fijado $n$ y definida $\alpha_n : G \to G,\ x \mapsto x^n$ tenemos que $\alpha_n$ es un homomorfismo de grupos. Además podemos expresar $H_2 = \ker \alpha_n \implies H_2 \normsub G$. Además también tenemos que $H_1 = \ima \alpha_n < G$. Veamos que $H_1 \normsub G$. Es decir, que $gH_1\inv{g} = H_1,\ \forall g \in G$. Para ello tomamos $x_1^n \in H_1$ y lo conjugamos $gx_1^n\inv{g} = (gx_1\inv{g})^n$ por ser $\alpha$ homomorfismo de grupos. En particular $(gx_1\inv{g})^n \in \ima \alpha \implies (gx_1\inv{g})^n \in H_1 \implies (gx_1\inv{g})^n = x_2^n$ para algún $x_2 \in H_1 \implies H_1 \normsub G$.
	\end{proof}
\end{ex}

\begin{ex}[H2.13] Si $A$ es un grupo abeliano con $n$ elementos y $k$ es un entero primo con $n$, demostrad que la aplicación $\varphi : A \to A$ definida por $\varphi(a) = a^k$ es un isomorfismo.
	\begin{itemize}
		\item $\varphi$ homomorfismo de grupos.
		\begin{proof}
			\begin{align*}
				\varphi(a)\varphi(b) = a^kb^k = (ab)^k = \varphi(ab)
			\end{align*}
		\end{proof}
		\item $\varphi$ biyectiva $\iff \varphi$ inyectiva ya que dominio y codominio coinciden
		\begin{proof}
			$\ker \varphi = \{a \in A \mid \varphi(a) = a^k = 1\}$. Probaremos que $a^k = 1 \iff a = 1$ y por tanto que $\ker \varphi = \{1\} \implies \varphi$ inyectiva. Sabemos que $a^k = 1 \iff o(a^k) = 1$. Sea $t = o(a) \divides n$. Distinguimos dos casos
			\begin{itemize}
				\item Si $t = 1$ entonces $a = 1$ y ya está
				\item Si $t > 1$ entonces $o(a^k) = \frac{t}{mcd(k, t)} = \frac{t}{1} > 1$ contradicción. Luego necesariamente $t = o(a) = 1$.
			\end{itemize}
		\end{proof}
	\end{itemize}
\end{ex}

\begin{ex}[H2.22]
	\label{ex:h2.22}
	Demostrad que si $G$ es un grupo no conmutativo y tiene orden $p^3$ ($p$ un número primo) entonces $Z(G)$ tiene orden $p$.
	
	\begin{proof}
		Sabemos que $Z(G) < G \implies |Z(G)| \divides |G| \implies |Z(G)| \in \{1, p, p^2, p^3\}$
		\begin{itemize}
			\item $|Z(G)| \neq p^3$ porque en tal caso $G$ sería conmutativo
			\item $|Z(G)| \neq 1$ porque $G$ es un p-grupo y por tanto su centro no es el trivial.
			\item Si $|Z(G)| = p^2$ entonces $|G/Z(G)| = p \implies G/Z(G)$ es cíclico lo que no es posible si $G$ no es abeliano.
		\end{itemize}
	
		Por descarte concluimos que $|Z(G)| = p$.
	\end{proof}
\end{ex}


\begin{ex}[H2.25]
	Sabemos que $\autom{\Z/12\Z} \isom \uds{\Z^\star/12\Z}$ donde $\Z^\star/12\Z$ es el grupo multiplicativo $(\{\overline{1}, \overline{2}, \overline{3}, \dots,  \overline{11}\}, \cdot)$. Queda $\uds{\Z^\star/12\Z} = (\{\overline{1}, \overline{5}, \overline{7}, \overline{11}\}, \cdot )$ y además da la casualidad que $\forall x \in \uds{\Z^\star/12\Z},\ o(x) = 2$ (todos los elementos son su propio inverso) por lo que no tenemos restricciones al definir $f: \Z/2\Z \to \autom{\Z/12\Z}$:
	\begin{align*}
		f:\Z/2\Z & \to \autom{\Z/12\Z} \isom \uds{\Z^\star/12\Z} \\
		e = \overline{0} &\mapsto 1 \\
		\overline{1} &\mapsto \{\overline{1}, \overline{5}, \overline{7}, \overline{11}\}
	\end{align*}
\end{ex}

\begin{ex}[H2.26]
	Sea $|G_1| = m, |G_2| = n,\ mcd(m, n) = 1$. Si $f:G_1 \to G_2$ es h. de g. sabemos que $o(f(a)) \divides o(a),\ \forall a \in G_1$. Además $o(a) \divides m \land o(f(a)) \divides n$ por el teorema de Lagrange (\ref{thm:lagrange}).
	\begin{align*}
		\begin{cases}
		o(a) \divides m \land o(f(a)) \divides n \\
		mcd(m, n) = 1 \\
		o(f(a)) \divides o(a)
		\end{cases} \implies o(a) = o(f(a)) = 1, \forall a \in G_1
	\end{align*}
	Por lo que solo puede haber un homomorfismo entre ellos y además es el trivial $f(a) = e_{G_2}$.
\end{ex}

\begin{ex}[H2.19]
	Definimos una función $f:[0, 2\pi] \subset \R \to \mathbb{S}^1, \alpha\mapsto \cos \alpha + i \sin \alpha$. Esta función tiene la propiedad de que $f(\alpha) \cdot f(\alpha') = \cos(\alpha + \alpha') + i\sin(\alpha + \alpha') = f(\alpha + \alpha')$ y por tanto es un h. de g.\footnote{A la izquierda (en $\R$) sumamos pero a la derecha (en $\mathbb{S}^1$) multiplicamos.} entre $R$ y $\mathbb{S}^1$.
	
	Un elemento de $\cos \alpha + i \sin \alpha \in \mathbb{S}^1$ es de torsión $\iff \exists n \mid (\cos \alpha + i \sin \alpha )^n = 1$. Ahora bien $(\cos \alpha + i \sin \alpha )^n = \cos n\alpha + i \sin n\alpha = 1 \iff n\alpha = k 2\pi$. 
\end{ex}

\section{Hoja 4}

\begin{ex}[H4.11] Hallar los subgrupos de Sylow de $S_5$.
	Sabemos que $|S_5| = 5! = 2^3\cdot3\cdot 5$ y por \nameref{thm:sylow1} tenemos lo siguiente:
	\begin{itemize}
		\item $\exists P_2,\ |P_2| = 2^3 = 8$.
		\item $\exists P_3,\ |P_3| = 3$. Además en $S_5$ hay $\binom{5}{3}2! =20 $ 3-ciclos y en cada $gP_3\inv{g} = \{1, a, a^2 \mid o(a) = 3\}$ hay 2 elementos de orden 3 distintos. Además, $g_1P_3\inv{g_1} \cap g_2P_3\inv{g_2} = \{e\}$ porque si su intersección fuera más grande entonces serían el mismo subgrupo (porque son cíclicos). Es por esto que tenemos que repartir 40 elementos dando 2 a cada 3-grupo con lo que obtenemos $n_3 = 20 / 2 = 10$ 3-subgrupos de Sylow en $S_5$.
		\item $\exists P_5,\ |P_5| = 5$. Además en $S_5$ hay $4! = 24$ 5-ciclos (elementos de orden 5) y en cada $gP_5\inv{g} = \{1, a, a^2, a^3, a^4 \mid o(a) = 5\}$ tenemos $4$ elementos de orden $5$ distintos. Además, $g_1P_5\inv{g_1} \cap g_2P_5\inv{g_2} = \{e\}$ porque si su intersección fuera más grande entonces serían el mismo. Así, tenemos $24$ 5-ciclos a repartir entre los diferentes $gP_5\inv{g}$ dando 4 5-ciclos a cada 1. Por tanto tenemos $n_5 = 24/4 = 6$ 5-subgrupos de Sylow en $S_5$.
	\end{itemize}
\end{ex}

\begin{ex}[H4.18]
	Demostrar que todo grupo de orden $|G| = 5^3 \cdot 7^3$ tiene un subgrupo normal de orden 125.
	
	\begin{proof}
		\nameref{thm:sylow1} $\implies \exists P_5 < G,\ |P_5| = 125$. \nameref{thm:sylow3} $\implies n_5 \divides 7^3 \land n_5 \equiv 1 \mod 5$ es decir $n_5 \in \{1, 7, 49, 343\} \land n_5 \in \{1, 6, 11, \dots\}$. Como ni $49$ ni $343$ son conguentes con $1$ módulo 5 tenemos que $n_5 = 1 \implies P_5 \normsub G$.
	\end{proof}
\end{ex}

\begin{ex}[H4.20] Hallad todos los grupos abelianos de órdenes 36, 64, 96 y 100.
	\begin{enumerate}
		\item $|G| = 36 = 2^2\cdot3^2$
		\begin{proof}
			\nameref{thm:sylow1} $\implies \exists P_2,\ |P_2| = 4 \land \exists P_3,\ |P_3| = 9$. Además $G$ abeliano $\implies P_2, P_3 \normsub G \implies G \isom P_2 \times P_3$. Estudiamos los grupos de orden 4 y de orden 9
			\begin{itemize}
				\item $|P_2| = 4$ entonces $P_2 \isom \Z/2\Z \times \Z/2\Z \lor P_2 \isom \Z/4\Z$
				\item $|P_3| = 9$ entonces $P_3 \isom \Z/3\Z \times \Z/3\Z \lor P_3 \isom \Z/9\Z$
			\end{itemize}
			Como $G \isom P_2 \times P_3$ tenemos 4 posibles grupos abelianos de orden 36.
		\end{proof}
	\end{enumerate}
\end{ex}

\begin{ex}[H4.22]
	Hallar todos los grupos abelianos de orden 175.
	
	\begin{proof}
		$|G| = 5^2\cdot 7$. Por el \nameref{thm:sylow1} tenemos que $\exists P_5, P_7 < G$ con $|P_5| = 25,\ |P_7| = 7$ y además por ser $G$ abeliano tenemos que $P_5, P_7 \normsub G \implies G \isom P_5 \times P_7$. Estudiamos los grupos de orden 25 y de orden 7:
		\begin{itemize}
			\item $|P_5| = 25 \land P_5$ abeliano $\implies P_5 \isom \Z/25\Z \lor P_5 \isom \Z/5\Z \times \Z/5\Z$. En ambos casos $P_5$ es producto directo de cíclicos pues $\Z/n\Z$ es cíclico.
			\item $|P_7| = 7 \land P_7$ abeliano $\implies P_7 \isom \Z/7\Z$. Ocurre lo mismo que con $P_5$.
		\end{itemize}
		Concluimos que $G \isom \Z/25\Z \times \Z/7\Z \lor G \isom \Z/5\Z \times \Z/5\Z \times \Z/7\Z$. Los dos casos son abelianos por ser producto directo de grupos cíclicos.
	\end{proof}
\end{ex}

\begin{ex}[H4.23]
	¿Cuántos elementos de orden 3 puede tener un grupo abeliano de orden 36?
	
	\begin{proof}
		$|G| = 36 = 2^2 3^2$. \nameref{thm:sylow1} $\implies \exists P_2, P_3 < G,\ |P_2| = 4,\ |P_3| = 9$. Estudiamos los grupos de órdenes 4 y 9:
		\begin{itemize}
			\item $|P_3| = 9 \implies P_3 \isom \Z/9\Z \lor P_3 \isom \Z/3\Z \times \Z/3\Z$
			% TODO: terminar
		\end{itemize}
	\end{proof}
\end{ex}

\chapter{Índices}

\renewcommand{\listtheoremname}{Lista de definiciones}
\listoftheorems[ignore={thm,ej,pro,cor,obs,lem,ex}]

\renewcommand{\listtheoremname}{Lista de teoremas}
\listoftheorems[onlynamed,ignore={dfn,ej,pro,cor,obs,lem,ex}]

\renewcommand{\listtheoremname}{Lista de ejemplos}
\listoftheorems[onlynamed,ignore={dfn,thm,pro,cor,obs,lem,ex}]

\renewcommand{\listtheoremname}{Lista de ejercicios}
\listoftheorems[onlynamed,ignore={dfn,thm,pro,cor,obs,lem,ej}]


\bibliographystyle{alpha}
\bibliography{apuntes-ea}


\end{document}