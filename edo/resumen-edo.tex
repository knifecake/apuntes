\newcommand{\R}{\mathbb{R}}
\renewcommand{\Re}{\operatorname{Re}}
\renewcommand{\Im}{\operatorname{Im}}
\newcommand{\A}{\mathbb{A}}

\hypertarget{ecuaciones-diferenciales-ordinarias-de-grado-n}{%
\section{\texorpdfstring{Ecuaciones diferenciales ordinarias de grado
\(n\)}{Ecuaciones diferenciales ordinarias de grado n}}\label{ecuaciones-diferenciales-ordinarias-de-grado-n}}

\begin{itemize}
\item
  Una ecuación diferencial de grado \(n\) se puede escribir
  \[x^{(n)} + a_{n-1}(t)x^{(n-1)} + \dots  + a_2(t)x'  + a_1(t) x = f(t)\]
  Si el coeficiente de \(x^{(n)}\) no es \(0\) se puede dividir por el
  para llegar a la forma anterior. Sino, se trata de una ecuación de
  grado \(n-1\).
\item
  Su ecuación homogénea asociada es
  \[x^{(n)} + a_{n-1}(t)x^{(n-1)} + \dots  + a_2(t)x' + a_1(t) x = 0\]
  Las soluciones de esta ecuación están en un espacio vectorial de
  dimensión \(n\). Esto significa que encontrando \(n\) soluciones
  linealmente independientes \(x_1, \dots, x_n\) ya tenemos todas las
  soluciones de la ecuación homogénea que podemos escribir como la
  solución general \[x_h(t) = c_1 x_1(t) + \dots + c_n x_n(t)\]
\item
  En general, no hay un método para resolver ecuaciones en las que los
  coeficientes \(a_1(t), \dots, a_{n-1}(t)\) no son constantes pero
  algunas admiten trucos que se describen más adelante.
\item
  En cualquier caso, la solución de una EDO (homogénea o no) de grado
  \(n\) siempre será \[ x(t) = x_h(t) + x_p(t)\] donde \(x_h\) es una
  solución general de la ecuación homogénea asociada y \(x_p\) una
  solución particular de la EDO original.
\item
  Si \(a_1(t), \dots, a_{n-1}(t)\) son en realidad constantes entonces
  la solución de la ecuación se obtiene con el siguiente procedimiento:

  \begin{enumerate}
  \def\labelenumi{\arabic{enumi}.}
  \item
    Solución de general de la homogénea, e.d., base de un e.v. de
    dimensión \(n\). Es de la forma
    \[x_h(t) = c_1 x_1(t) + c_2 x_2(t) + \dots + c_n x_n(t)\] para
    \(c_1, \dots, c_n \in \mathbb{R}\) y \(x_1, x_2, \dots, x_n\)
    soluciones particulares linealmente independientes. Estas últimas se
    pueden obtener a partir de las soluciones de la ecuación
    característica
    \(P(\lambda) = \lambda^n + a_{n-1}\lambda^{n-1} + \dots + a_2 \lambda + a_1 = 0\)

    \begin{itemize}
    \item
      Si todas las raíces son reales y distintas entonces la solución es
      de la forma
      \[x_h(t) = c_1 e^{\lambda_1 t} + c_2 e^{\lambda_2 t} + \dots + c_n e^{\lambda_n t}\]
    \item
      Si aparecen raíces \(\lambda\) con multiplicidad \(m > 1\)
      entonces la solución homogénea incluirá los términos
      \[c_1 e^{\lambda t} + c_2 t e^{\lambda t} + \dots + c_m t^{m-1} e^{\lambda t}\]
      para garantizar que el espacio de solucuiones siga teniendo
      dimensión \(n\).
    \item
      Si aparecen dos raíces \(\lambda_1, \lambda_2\) complejas entonces
      también serán conjugadas. Si decimos que \(\lambda_1 = a + bi\) (y
      por tanto \(\lambda_2 = a - bi\)), la solución tendrá términos de
      la forma
      \[c_1 e^{\lambda_1 t} + c_2 e^{\lambda_2 t} = e^{at}\left(\hat{c_1}\cos bt + \hat{c_2}\sin bt\right)\]
      En este caso todas las constantes son reales (la parte imaginaria
      se cancela).
    \end{itemize}
  \item
    Solución particular de la EDO:

    \begin{itemize}
    \item
      Por variación de constantes utilizando el Wronskiano
      \[ \left( \begin{array}{cccc}
       x_1(t)         & x_2(t)         & \dots  & x_n(t) \\
       x_1'(t)        & x_2'(t)        & \dots  & x_n'(t) \\
       \vdots         & \vdots         & \ddots & \vdots \\
       x_1^{(n-1)}(t) & x_2^{(n-1)}(t) & \dots  & x_n^{(n-1)}(t)
      \end{array}\right)\left(\begin{array}{c} c_1'(t) \\ c_2'(t) \\ \vdots \\ c_n'(t)\end{array}\right) = \left(\begin{array}{c}0 \\ 0 \\ \vdots \\f(t)\end{array}\right) \]
      e integrando \(c_i'(t)\) con respecto a \(t\).
    \item
      Por coeficientes indeterminados
    \end{itemize}
  \item
    Concluir que la solución general de la EDO es
    \(x(t) = x_h(t) + x_p(t)\).
  \end{enumerate}
\item
  Las constantes \(c_1, \dots, c_n\) de la solución general de la EDO
  original se determinan a partir de un PVI en el que aparecen \(n\)
  condiciones (por ejemplo de la forma
  \(x(0) = x_0, x'(0) = x_1, \dots, x^{(n - 1)}(0) = x_{n-1}\).
\end{itemize}

\hypertarget{sistemas-de-ecuaciones-lineales}{%
\section{Sistemas de ecuaciones
lineales}\label{sistemas-de-ecuaciones-lineales}}

\hypertarget{sistemas-lineales-con-coeficientes-constantes}{%
\subsection{Sistemas lineales con coeficientes
constantes}\label{sistemas-lineales-con-coeficientes-constantes}}

\begin{itemize}
\item
  Un sistema lineal de EDOs se ecribe \[X'(t) = \mathbb{A}X(t) + B(t)\]
  donde \(X(t)\) es una función vectorial de \(n\) variables, \(X'(t)\)
  su derivada con respeco de \(t\) y \(B\) un vector de funciones en
  \(t\).
\item
  Un sistema lineal es homogéneo si \(B(t)\) es nulo.

  \begin{itemize}
  \item
    Las soluciones de un sistema lineal homogéneo de \(n\) EDOs es un
    espacio vectorial de dimensión \(n\).
  \item
    Se obtiene una base de este espacio a partir de la ecuación
    característica en \(\lambda\). En el caso de los sistemas la
    ecuación característica es el polinomio característico de la matriz
    \(\mathbb{A}\) igualado a 0: \[\det (\mathbb{A}- \lambda I) = 0\]

    \begin{itemize}
    \item
      Para cada \(\lambda_i\) autovalor de \(\mathbb{A}\) con
      multiplicidad 1 se obtiene una solución
      \(X_i(t) = e^{\lambda_i t}V_i\) donde \(V_i\) es el autovector
      asociado a \(\lambda_i\).
    \item
      Para cada \(\lambda_i\) autovalor de \(\mathbb{A}\) con
      multiplicidad \(m\) se obtiene una solución de la forma
      \(X_i(t) = e^{\lambda_i t}(V_1 + tV_2 + \dots + t^{m-1}V_m\). Los
      vectores \(V_1, \dots, V_{m-1}\) se obtienen de plantear el
      sistema: \begin{multline*}
      \begin{cases}
      \frac{d}{dt} e^{\lambda_i t}V_1 = \mathbb{A}V_1 \\
      \frac{d}{dt} e^{\lambda_i t}(V_1 + tV_2) = \mathbb{A}(V_1 + tV_2) \\
      \dots \\
      \frac{d}{dt} e^{\lambda_i t}(V_1 + t V_2 + \dots + t^{m-1} V_m) = \mathbb{A}(V_1 + t V_2 + \dots + t^{m-1} V_m)
      \end{cases} \iff \\
      \begin{cases}
      \lambda_i V_1 = \mathbb{A}V_1 \\
      \lambda_i (V_1 + t V_2) = \mathbb{A}(V_1 + tV_2) \\
      \dots \\
      \lambda_i(V_1 + \dots  + t^{m-1}V_m = \mathbb{A}(V_1 + \dots + t^{m-1}V_m
      \end{cases}
      \end{multline*}
    \item
      Si \(\lambda_i\) es complejo
    \end{itemize}
  \item
    La solución general de un sistema lineal homogéneo puede darse por
    la matriz fundamental principal \(e^{t\mathbb{A}}\).
  \end{itemize}
\end{itemize}

\hypertarget{sistemas-lineales-con-coeficientes-variables}{%
\subsection{Sistemas lineales con coeficientes
variables}\label{sistemas-lineales-con-coeficientes-variables}}

\begin{itemize}
\item
  En general no se pueden resover por el mismo procedimiento.
\item
  Si en realidad son sistemas no ligados, sí que se pueden resolver, por
  ejemplo el ejercicio 12 de la hoja 4 en el que una ecuación es
  \(y_1' = y_1\).
\end{itemize}

\hypertarget{teorema-general-de-existencia-y-unicidad}{%
\section{Teorema general de existencia y
unicidad}\label{teorema-general-de-existencia-y-unicidad}}

Sea \(\{f_n\}_{n=1}^\infty\) una sucesión de funciones sobre un
intervalo \(I\)

\begin{itemize}
\item
  \(\{f_n\}\) converge punto a punto a \(f(x) \iff\)
  \[\forall x \in I,\ f_n(x) \xrightarrow{n\to \infty} f(x)\]
\item
  \(\{f_n\}\) converge uniformemente a
  \(f(x) \iff \forall \varepsilon > 0, \exists N \in \N\) tal que
  \[|f_n(x) - f(x)| < \varepsilon,\ \forall n > N,\ \forall x \in I\]
\end{itemize}
