\chapter{Basic tools for mathematical modelling}

Teaching for this chapter started on Monday, 2019.11.11 (week 46a) and ended on
Monday, 2019.11.18. This chapter corresponds to part of chapter 1 in
\cite{eck2017mathematical}.

In this introductory chapter we will introduce the mindset that we should have
when trying to \textbf{translate a specific problem} from the natural sciences,
the social sciences or technology into a \textbf{well-defined mathematical
problem}\footnote{This is the definition of \textit{mathematical modelling}
given in \cite[p.  1]{eck2017mathematical} with Kreisbeck's emphasis.}.

TODO: some context and general pointers would probably look good here.

\section{Case study: population dynamics}

Suppose we want to model the change in population (i.e. number of individuals)
in an environment over a period of time. First thing we need is to make some
assumptions about what's really happening here. We might, for example, make the
following assumptions\footnote{Stolen from \cite{kreisbeck2019slides}.}

\begin{enumerate}
  \item growth rate independent of population size (unlimited growth possible,
    neglecting e.g. limited resources)

  \item growth rate independent of time (neglecting time-dependence due to e.g.
    influence of enemies, economical or cultural changes)

  \item population within closed systems (neglecting e.g. migration)

  \item assuming an equal distribution of male and female, age distribution not
    considered

  \item continuous model with non-integer solutions (idealization reasonable
    for very large populations, for small populations stochastic effects have
    to be taken into account)
\end{enumerate}

After this, we name the quantities that intervene in our problem. We will use
$t$ for time, $x(t)$ for the number of individuals (population) at time $t$ and
$\frac{dx}{dt}(t)$ or $x'(t)$ for the rate of change in population. To model
the change we introduce the quantities
\begin{itemize}
  \item $b(t, \Delta t)$ for the increase of population during the time
    interval $(t, \Delta t)$, and
  \item $d(t, \Delta t)$ for the decrease of population during the time
    interval $(t, \Delta t)$.
\end{itemize}

Therefore the population at time $t + \Delta t$ is given by
\[
  x(t + \Delta t) = x(t) + b(t, \Delta t) - d(t, \Delta t).
\]
That $\Delta t$ desperately wants us to take the limit as $\Delta t \to 0$ and
so we do
\[
  \lim_{\Delta t \to 0} \frac{x(t + \Delta t) - x(\Delta t)}{\Delta t}
  = \lim_{\Delta t \to 0} \frac{b(t, \Delta t)}{\Delta t} - \lim_{\Delta t \to 0} \frac{d(t, \Delta t)}{\Delta t}.
\]

Note that here we're assuming the limit really does exist, which is quite a big
assumption\ldots\ Rename,
\[
  B(t) = \lim_{\Delta t \to 0} \frac{b(t, \Delta t)}{\Delta t} \text{ and }
  D(t) = \lim_{\Delta t \to 0} \frac{d(t, \Delta t)}{\Delta t}
\]
and use the definition of derivative to get
\[
  \frac{dx}{dt}(t) = x'(t) = B(t) - D(t)
\]
where $B(t)$ and $D(t)$ being the rates at which the population increasing,
resp. decreases at time $t$. Recall that we assumed that the rates of change in
the population were independent of time and population size. This is equivalent
to saying that $B(t)$ and $D(t)$ are really constants which gives us the final
model
\[
  x'(t) = \beta - \delta \implies x(t) = (\beta - \delta)t.
\]
The previous is a not-particularly-interesting ODE with solution\footnote{For
help with solving differential equations see \cite{math24}.}
\[
  x(t) = (\beta - \delta)x + C.
\]

This model has a lot of shortcomings, first of all, it does not account for the
size of the population in the rates of change. But, one might argue that the
more individuals there are in a population the greater the rates of change are.
We can go back and restate assumption one as \say{population increase, resp.
decrease in the time interval $(t, \Delta t)$ is directly proportional to the
population at time $t$ and the time passed}.  This in turn gives us
\[
  b(t, \Delta t) = \beta x(t) \Delta t \text{ and }
  d(t, \Delta t) = \delta x(t) \Delta t.
\]
Taking the limit as before leads us to the model
\[
  x'(t) = (\beta - \delta)x(t) = px(t).
\]
From now on, we shall let $p = \beta - \delta$ since we don't really need to
distinguish between changes in the population because of births and deaths---we
just care about the overall evolution of the population. This is another ODE,
this time a bit more interesting, with solution
\[
  x(t) = C e^{p t}.
\]

Although a bit better, you can probably see that this model explodes as time
passes since it does not include any provisions for when the population turns
stupidly large. Anyhow, it is common enough that it deserves its own name: the
\textbf{exponential growth model}.

A small step in the right direction would be to account for a population limit
in the system, i.e. number of individuals that flips the rate of growth. More
precisely, let's change assumption one to \say{there is a number $x_M$ that is
  the maximum population in the system (sometimes called the \textit{carrying
capacity} and that the rate of change in population $p(x(t))$ is positive if
$x(t) > x_M$ and negative if $x(t) > x_M$}. The easiest way to model this is
with a linear ansatz for $p(x)$, i.e. something of the form $p(x) = q(x_M - x)$
with a parameter\footnote{The meaning of $q$ is a bit more complex to explain,
but at heart it is just a proportionality constant.} $q > 0$. Notice how
\[
  \begin{cases}
    p(x) > 0 \tif x < x_M,\text{ and } \\
    p(x) < 0 \tif x > x_M
  \end{cases}
  .
\]
Plugging this into our previous model to get
\[
  x'(t) = q(x_M - x(t))x(t) = q x_M x(t) - q x^2(t),
\]
which is our final model for now and is called the
\textbf{logistic growth model}

This ODE can be solved exactly and the solution is
\[
  x(t) = \frac{x_M x_0}{x_0 + (x_M - x_0) e^{- x_M q (t - t_0)}},
\]
where $t_0$ is the initial time and $x_0 = x(t_0)$ is the initial population.


\begin{figure}
  [h]
  \incfig{logistic-growth-model}

  \caption{Solutions for the three iterations of the model, for different
  values for the parameters on each version.}
\end{figure}

We'll stop here for now, but keep in mind that we're missing the second half of
the solution---we still need to apply this models to the real world. This means
\textit{fitting} those curves to the specific problem at hand, in this case,
getting some data (at least two data points) to calculate the constants that
are present in our solutions. Also, recall that we made a lot of assumptions,
there are more population dynamics models that account for changes in the
environment, migrations, etc.



\section{Dimensional analysis and non-dimensionalisation}

The previous models had two or three parameters each, but as we work our way to
more complex examples the number of parameters will increase. Moving around all
those constants is cumbersome and draws our attention away from really
understanding the problem at hand. In addition, as we apply our models to
specific problems we will need to take into account the units of the quantities
we are dealing with.

\subsection{Dimensional analysis}

In addition to being a prerequisite to doing non-dimensionalisation,
dimensional analysis provides a sanity check for us to \textit{make sure we're
not adding apples to oranges}. When coming up with a model, we generally need
to specify the \textit{physical dimension} of the quantities involved, but not
necessarily want to specify the particular units that quantity is expressed in.
For this we will denote by $[c]$ the physical dimension of a quantity $c$. For
example if $t$ denotes time, when we write $[t]$ we mean the \textit{dimension
of time} or \textit{some units of time} but do not specify which.

Revisiting our population model we can define the characteristic units
\[
  T := \text{ time },\qquad \text{ and } \qquad N := \# \text{ of individuals }
\]
so that we get the following dimensions for the involved quantities
\[
  [t] = T,\ [x(t)] = N,\ [x'(t)] = \frac{N}{T}.
\]

We can also do this for the parameters by solving for them in the equation for
the model\footnote{Notice how $[x_M - x] = N$ and not something weird like $N -
N = 0$, since when you subtract apples from apples you still get apples.}
\[
  [x_M] = N,\ [t_0] = T,\ [x_0] = N,\ 
  [q] = \frac{[x']}{[x_M - x][x]} = \frac{N / T}{N \cdot N} = \frac{1}{T}. 
\]

\subsection{Non-dimensionalisation}

Once we know the physical units of all involved quantities in our model we are
ready to choose actual units for our model. For instance, for time we might
choose years, days or hours but most of the time it is better to choose
appropriate units for our problem. Non-dimensionalisation\footnote{Yes, this is
an accepted spelling although not very common in the literature.} is a recipe
for choosing the most appropriate units.

From the dimensional analysis of our population dynamics examples we now that
there are two physical dimensions and therefore we will choose two
characteristic quantities $\overline{t}$ and $\overline{x}$.

\subsection{Non-dimensionalisation when there are several options. The
projectile problem.}

\section{Asymptotic expansion method}

\subsection{Error estimation}

\section{Exercises}

\begin{ex}
  [Recap, part a]

  Solve the IVP
  \[
  \begin{cases}
    y'(t) = \frac{2t}{y(t) - 1} ,\ t > 0 \\
    y(0) = 2
  \end{cases}
  \]
\end{ex}

\begin{proof}
  We ommit the $t$ in $y(t)$ and just write $y = y(t)$ for short.
  Via separation of variables we get
  \[
    (y - 1)y' = 2t. 
  \]
  Integrating on both sides with respect to $t$ we get,
  \[
    \int (y - 1) y'\ dt = \int 2t\ dt
  \]
  and with the rule for substitution on integrals we rewrite it as
  \[
    \int (y - 1)\ dy = \int 2t\ dt.
  \]
  Solving the integrals we arrive at the equality
  \[
  \frac{y^2}{2} - y = t^2 + c
  \]
  and solving for $y$ yields
  \[
    y_\pm = 1 \pm \sqrt{1 - 2(t^2 +c)}.
  \]
  Substituting the initial condition $y(0) = 2$ we get
  \[
    1 \pm \sqrt{1 - 2c} = 2
  \]
  which can only be satisfied with $y_+$ and picking $c = 0$. Therefore, the
  final solution to the IVP is
  \[
   y(t) = 1 + \sqrt{1 - 2t^2}.
  \]
\end{proof}


\begin{ex}
  [Recap, part b]

  Compute the solution to the second-order IVP
  \[
  \begin{cases}
    y''(t) = y(t) + e^t,\ t > 0,\\
    y(0) = 1,\ y'(0) = 1.
  \end{cases}
  \]
\end{ex}

\begin{proof}$ $ \newline
  \begin{enumerate}
    \item Step 1: we find the general solution $y_h$ to the homogeneous ODE
      $y'' = y$. It follows from setting the characteristic polynomial
      $P(\lambda) = \lambda^2 - 1 = 0 \iff \lambda = \pm 1$. Therefore,
      \[
        y_h(t) = c_1 e^t + c_2 e^{-t}.
      \]
    \item Step 2: we find a particular solution $y_p(t)$ to the ODE $y'' = y +
      e^t$. One way to do it is with the method of indeterminate coefficients.
      We need a solution wich is independent from the homogeneous solution
      $y_h(t)$ so we make the ansatz
      \[
        y_p(t) = a t e^t.
      \]
      Computing the first and second derivatives of $y_p(t)$ and substituting
      them into the original ODE (without boundary conditions) yields the
      equality
      \[
      ae^t + ae^t + ate^t = ate^t + e^t \implies a = \frac{1}{2}.
      \]
    \item Step 3: the solution to the IVP is given by
      \[
        y(t) = y_h(t) + y_p(t) = c_1e^t + c_2e^{-t} + \frac{1}{2} e^t.
      \]
      Pluging in the initial values for $y$ and $y'$ we arrive at the system of
      linear equations
      \[
      \begin{cases}
        c_1 + c_2 = 1\\
        c_1 - c_2 + \frac{1}{2} = 1
      \end{cases}
      \iff
      \begin{cases}
        c_1 = \frac{3}{4} \\
        c_2 = \frac{1}{4}.
      \end{cases}
      \]
      Therefore, the solution to the IVP is given by the function
      \[
        y(t) = \frac{3}{4} e^t + \frac{1}{4} e^{-t} + \frac{1}{2} t e^t.
      \]
  \end{enumerate}
  
\end{proof}


\begin{ex}
  [1.4]
  We consider the model of limited growth of populations
  \[
    x'(t) = qx_M x(t) - q x^2(t),\ x(0) = x_0.
  \]
  \begin{enumerate}
    \item Nondimensionalize the model using appropriate units for $t$ and $x$.
      Which possibilities exist?
    \item What nondimensionalization is appropriate for $x_0 \ll x_M$ ($x_0$
      "much smaller than" $x_M$) in the sense that omitting small terms leads
      to a reasonable model?
  \end{enumerate}
\end{ex}

\begin{proof}
  The two main involved quantities in this model are $t$ with dimension $T$ and
  $x(t)$ with dimension $N$. We make the following change of variables:
  \[
    \tau = \frac{t}{\overline{t}}, \qquad
    y(\tau) = \frac{x(t) - x_0}{\overline{x}} = \frac{x(\tau \overline{t}) - x_0}{\overline{x}}.
  \]
  We solve for $x(t)$ and $x'(t)$ to be able to plug $y(\tau)$ into the model:
  \[
    x(t) = y(\tau)\overline{x} + x_0,\\
    x^2(t) = y^2(\tau)\overline{x}^2 + x_0^2 + 2x_0\overline{x}y(\tau), \\
    x'(t) = \ddt \left( \overline{x}y(\tau) - x_0\right) = \overline{x} y'(\tau) \cdot \frac{1}{\overline{t}}.
  \]
  Therefore,
  \[
  \frac{\overline{x}}{\overline{t}}y'(\tau) =
  qx_M \left(\overline{x}y(\tau) + x_0\right)
  - q\left(\overline{x}^2y'(\tau) + x_0^2 + 2x_0\overline{x}y(\tau)\right),
  \]
  or, equivalently,
  \[
    y'(\tau) = qx_M \overline{t} y(\tau) +
    qx_M \frac{\overline{t}}{\overline{x}} x_0
    - q \overline{t} \overline{x} y^t(\tau)
    - 2q x_0 \overline{t} y(\tau).
  \]
  As for the inital conditions we get
  \[
    y(0) = \frac{x(0 \cdot \overline{t}) - x_0}{\overline{x}} = 0
  \]
  as expected, because of the way we chose the change $y(\tau)$.

  There are many possibilities for the choice of $\overline{t}$ and
  $\overline{x}$. We will not discuss the now, but rather wait until part be.
  For an example of a detailed discussion without making extra assumptions see
  \autoref{ex:1.6}.

  If we make the assumption that $x_0$ is very small, we may discard some terms
  in the above nondimensionalised model to get
  \[
    \begin{cases}
      y'(\tau) = qx_M \overline{t} y(\tau) - q \overline{t} \overline{x} y^2(\tau), \\
      y(0) = 0.
    \end{cases}
  \]
  In this case, setting the cofficients for the function $y$ to equal $0$ we
  get the system
  \[
  \begin{cases}
    qx_M \overline{t} = 1, \\
    x\overline{t} \overline{x} = 1
  \end{cases}
  \iff
  \begin{cases}
    \overline{t} = \frac{1}{q x_M}, \\
    \overline{x} = \frac{x_M}{\overline{t}}.
  \end{cases}
  \]
  In this case there is only this posibility, since we have two restrictions on
  two characteristic quantities.
  
\end{proof}

\begin{ex}
  \textit{(Nondimensionalization, scale analysis)} A body of mass $m$ is thrown
  upwards in a vertical direction from the Earth’s surface with a velocity $v$.
  The air resistance is supposed to be taken into account by Stokes' law $F_R =
  -cv$ for the flow resistance in viscous fluids, which is reasonable for small
  velocities. Here $c$ is a coefficient depending on the shape and the size of
  the body. The motion is supposed to depend on the mass $m$, the velocity $v$,
  the gravitational acceleration $g$ and the friction coefficient $c$ with
  dimension $[c] = M/T$.

  \begin{enumerate}
    \item Determine the possible dimensionless parameters and reference values
      for height and time.
    \item The initial value problem for the height is assumed to take the form
      \[
      m x'' = c x' = -m g,\qquad x(0) = 0,\qquad x'(0) = v.
      \]
      Nondimensionalize the differential equation. Again different
      possibilities are available.
    \item Discuss the different possibilities of a reduced model if $\beta :=
      cv / (m g)$ is small.
  \end{enumerate}
\end{ex}

\begin{proof}
  TODO
\end{proof}



\begin{ex}
  [1.6]
  \label{ex:1.6}

  A model for the vertical throw on the Earth taking into account the air
  resistance is given by
  \[
    mx''(t) = -mg  - c\abs{x'(t)}x'(t),\ x(t_0) = 0, x'(t_0) = v_0.
  \]
  In this model the gravitational force is approximated by $F = -mg$, the air
  resistance for a given velocity $v$ is described by $-c\abs{v}v$ with a
  proportionality constant $c$ depending on the shape and size of the body and
  the density of the air. This law is reasonable for high velocities.
  \begin{enumerate}
    \item Nondimensionalize the model. What possibilities exist?
    \item Compute the maximal height of the throw for the data $m =
      \SI{0.1}{kg},\ g = \SI{10}{m\per s^2},\ v_0 = \SI{10}{m \per s}, c =
      \SI{0.01}{kg\per m}$ and compare the result with the corresponding result
      for the model without air resistance.
  \end{enumerate}
  
\end{ex}


\begin{proof}
  We make the change of variables
  \[
    \tau = \frac{t - t_0}{\overline{t}},\qquad
    y = \frac{x}{\overline{x}}
  \]
  and plug it into the IVP to get
  \[
      y''(\tau) = - \frac{\overline{t}^2 g}{\overline{x}}
        - \frac{c\overline{x}}{m} \abs{y'(\tau)} y'(\tau),\qquad
      y(0) = 0,\qquad
      y'(0) = \frac{\overline{t}}{\overline{x}} v_0.
  \]
  We would like as many coefficients as possible to equal 1 but we have 3
  coefficients and 2 characteristic quantities so we will have to make a
  compromise. There are three possibilities for the compromise:
  \begin{enumerate}
    \item Set $\frac{\overline{t}^2 g}{\overline{x}} = \frac{c\overline{x}}{m}
      = 1$ and leave $\frac{\overline{t}}{\overline{x}} v_0$ as is, which would
      yield $\overline{t} = \sqrt{\frac{m}{gc} },\ \overline{x} = \frac{m}{c} $, and
      \[
        y'' = -1 - \abs{y'}{y'},\qquad
        y(0) = 0,\qquad
        y'(0) = v_0\sqrt{\frac{c}{gm}}. 
      \]
    \item Set $\frac{\overline{t}^2 g}{\overline{x}} =
      \overline{t}/\overline{x} v_0 = 1$ and leave $\frac{c\overline{x}}{m}$ as
      is, which would yield $\overline{t} = \frac{v_0}{g},\ \overline{x} =
      \frac{v_0^2}{g}$, and
      \[
        y'' = -1 - \frac{cv_0^2}{mg} - \abs{y'} y',\qquad
        y(0) = 0,\qquad
        y'(0) = 1.
      \]
    \item Set $\frac{c\overline{x}}{m} = \overline{t}/\overline{x} v_0 = 1$
      and leave $\frac{\overline{t}^2 g}{\overline{x}}$ as is, which would
      yield $\overline{t} = \frac{m}{c v_0}, \overline{x} = \frac{m}{c}$, and
      \[
        y'' = - \frac{mg}{cv_0^2} - \abs{y'} y,\qquad
        y(0) = 0,\qquad
        y'(0) = 1.
      \]
  \end{enumerate}

  The second part is left to the reader.
\end{proof}


\begin{ex}
  [1.8]
  \textit{(Formal aymptotic expansion)}
  \begin{enumerate}
    \item For the initial value problem
      \[
        x''(t) + \varepsilon x'(t) = -1,\qquad
        x(0) = 0,\qquad
        x'(0) = 1
      \]
      compute the formal asymptotic expansion of the solution $x(t)$ up to the second order in $\varepsilon$.
    \item Compute the formal asymptotic expansion for the instance of time $t^*
      > 0$, for which $x(t^*) = 0$ holds true, up to first order in
      $\varepsilon$, by substituting the series expansion $t^* \sim t_0 +
      \varepsilon t_1 + O(\varepsilon^2)$ into the approximation obtained for
      $x$ leading to a determination of $t_0$ and $t_1$.
  \end{enumerate}
\end{ex}

\begin{proof}
  TODO
\end{proof}

\begin{ex}
  [1.9]

  A model already nondimensionalised for the vertical throw with \textit{small}
  air resistance is given by
  \[
    x''(t) = -1 - \varepsilon(x'(t))^2,\qquad
    x(0) = 0,\qquad
    x'(0) = 1.
  \]
  The model describes the throw up to the maximal height.

  \begin{enumerate}
    \item Compute the first two coefficientes $x_0(t)$ and $x_1(t)$ in the asymptotic expansion
      \[
        x(t) = x_0(t) + \varepsilon x_1(t) + \varepsilon^2 x_2(t) + \dots
      \]
      for small $\varepsilon$.
    \item Compute the maximal height of the throw up to terms of order
      $\varepsilon$ using asymptotic expansion.
    \item Compare the results from 2. for the data of \autoref{ex:1.6} with the
      exact result and the result neglecting the air resistance.
  \end{enumerate}
\end{ex}

\begin{proof}
  TODO
\end{proof}

\begin{ex}
  [1.10]

  \textit{(Multiscale approach)} The function $y(t)$ is supposed to solve the initial value problem
  \[
    y''(t) + 2 \varepsilon y'(t) + (1 + \varepsilon^2)y(t) = 0,\qquad
    y(0) = 0,\qquad
    y'(0) = 1,
  \]
  for $t > 0$ and a small parameter $\varepsilon > 0$.
  \begin{enumerate}
    \item Compute the approximation of the solution by means of formal
      asymptotic expansion up to first order in $\varepsilon$.
    \item Compare the function obtained in 1. with the exact solution
      \[
        y(t) = e^{-\varepsilon t} \sin t.
      \]
      For which times $t$ the approximation from 1. is good?
    \item To get a better approximation one can try the approach
      \[
        y \sim y_0(t, \tau) + \varepsilon y_1(t, \tau) + \varepsilon^2 y_2(t, \tau) + \dots,
      \]
      here $\tau = \varepsilon t$ is a slow time scale.
      Substitute this ansatz in the differential equation and compute $y_0$
      such that the approximation becomes better.
      \textit{Hint:} The equation of lowest order does not determine $y_0$
      uniquely and coefficient functions in $\tau$ apper. Choose them in a
      clever way such that $y_1$ is easily computable.
  \end{enumerate}
  
\end{ex}




