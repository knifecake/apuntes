\documentclass[a4paper]{book}

%\usepackage[utf-8]{inputenc}

\usepackage{amsmath}
\usepackage{amsfonts}
\usepackage{amssymb}
\usepackage{amsthm}

\usepackage{dirtytalk}
\usepackage{siunitx}


\usepackage[dvipsnames]{xcolor}
\usepackage{thmtools}

% inkscape figures
\usepackage{import}
\usepackage{pdfpages}
\usepackage{transparent}

\newcommand{\incfig}[2][1]{%
    \def\svgwidth{#1\columnwidth}
    \import{./figures/}{#2.pdf_tex}
}

% \pdfsuppresswarningpagegroup=1

\usepackage[hidelinks, backref]{hyperref}
\hypersetup{
  pdftitle={Notes on Mathematical Modelling},
	pdfauthor={Elias Hernandis},
	pdfpagemode=UseOutlines,
	bookmarksnumbered
}

\declaretheoremstyle[
bodyfont=\normalfont,
shaded={
	margin=8pt,
	bgcolor=White,
	rulecolor=Black,
	rulewidth=1pt
}]{mythm}


\declaretheoremstyle[
bodyfont=\normalfont,
spacebelow=1em,
spaceabove=1em,
]{myeg}

% DEFINICIONES DE ENTORNOS DE TEOREMAS
\declaretheorem[
name=Theorem,
refname={theorem,theorems},
Refname={Theorem,Theorems},
style=mythm,
numberwithin=chapter
]{thm}

\declaretheorem[
name=Lemma,
refname={lemma,lemmas},
Refname={Lemma,Lemmas},
style=myeg,
sibling=thm
]{lem}

\declaretheorem[
name=Remark,
refname={remark,remarks},
Refname={Remark,Remarks},
style=myeg,
sibling=thm
]{remark}

\declaretheorem[
name=Corollary,
refname={corollary,corollaries},
Refname={Corollary,Corollaries},
style=myeg,
sibling=thm
]{cor}


\declaretheorem[
name=Definition,
refname={definition,definitions},
Refname={Definition,Definitions},
style=myeg,
sibling=thm
]{dfn}

\declaretheorem[
name=Example,
refname={example,examples},
Refname={Example,Examples},
style=myeg,
sibling=thm
]{eg}

\declaretheorem[
name=Exercise,
refname={exercise,exercises},
Refname={Exercise,Exercises},
style=myeg,
numbered=no
]{ex}



\renewcommand{\complement}{c}
\newcommand{\abs}[1]{\left\vert{#1}\right\vert}
\newcommand{\norm}[1]{\left\Vert{#1}\right\Vert}
\newcommand{\N}{\mathbb{N}}
\newcommand{\Z}{\mathbb{Z}}
\newcommand{\R}{\mathbb{R}}
\newcommand{\tif}{&\text{ if }}
\newcommand{\inv}[1]{{#1}^{-1}}
\newcommand{\sgn}{\text{sgn}}
\renewcommand{\vec}[1]{\mathbf{#1}}
\newcommand{\calA}{\mathcal{A}}
\newcommand{\calI}{\mathcal{I}}
\DeclareMathOperator{\argmin}{argmin}
\newcommand{\0}{\mathbf{0}}
\newcommand{\ddt}{\frac{d}{dt}}
\newcommand{\ddx}{\frac{d}{dx}}

\author{Elias Hernandis}
\title{Notes on Mathematical Modelling}

\begin{document}
	\maketitle
	
	\section{Acknowledgements}
	
  These notes are based on the lectures of Carolin Kreisbeck
  (c.kreisbeck@uu.nl) during Fall of 2019 at Universiteit Utrecht. The lectures
  were intended to serve as a preparation for the reading of the texbook of the
  course \cite{eck2017mathematical}.
	
	Readers are asked to report errata to eliashernandis@gmail.com.
	
	
	\tableofcontents

  \chapter{Basic tools for mathematical modelling}

Teaching for this chapter started on Monday, 2019.11.11 (week 46a) and ended on
Monday, 2019.11.18. This chapter corresponds to part of chapter 1 in
\cite{eck2017mathematical}.

In this introductory chapter we will introduce the mindset that we should have
when trying to \textbf{translate a specific problem} from the natural sciences,
the social sciences or technology into a \textbf{well-defined mathematical
problem}\footnote{This is the definition of \textit{mathematical modelling}
given in \cite[p.  1]{eck2017mathematical} with Kreisbeck's emphasis.}.

TODO: some context and general pointers would probably look good here.

\section{Case study: population dynamics}

Suppose we want to model the change in population (i.e. number of individuals)
in an environment over a period of time. First thing we need is to make some
assumptions about what's really happening here. We might, for example, make the
following assumptions\footnote{Stolen from \cite{kreisbeck2019slides}.}

\begin{enumerate}
  \item growth rate independent of population size (unlimited growth possible,
    neglecting e.g. limited resources)

  \item growth rate independent of time (neglecting time-dependence due to e.g.
    influence of enemies, economical or cultural changes)

  \item population within closed systems (neglecting e.g. migration)

  \item assuming an equal distribution of male and female, age distribution not
    considered

  \item continuous model with non-integer solutions (idealization reasonable
    for very large populations, for small populations stochastic effects have
    to be taken into account)
\end{enumerate}

After this, we name the quantities that intervene in our problem. We will use
$t$ for time, $x(t)$ for the number of individuals (population) at time $t$ and
$\frac{dx}{dt}(t)$ or $x'(t)$ for the rate of change in population. To model
the change we introduce the quantities
\begin{itemize}
  \item $b(t, \Delta t)$ for the increase of population during the time
    interval $(t, \Delta t)$, and
  \item $d(t, \Delta t)$ for the decrease of population during the time
    interval $(t, \Delta t)$.
\end{itemize}

Therefore the population at time $t + \Delta t$ is given by
\[
  x(t + \Delta t) = x(t) + b(t, \Delta t) - d(t, \Delta t).
\]
That $\Delta t$ desperately wants us to take the limit as $\Delta t \to 0$ and
so we do
\[
  \lim_{\Delta t \to 0} \frac{x(t + \Delta t) - x(\Delta t)}{\Delta t}
  = \lim_{\Delta t \to 0} \frac{b(t, \Delta t)}{\Delta t} - \lim_{\Delta t \to 0} \frac{d(t, \Delta t)}{\Delta t}.
\]

Note that here we're assuming the limit really does exist, which is quite a big
assumption\ldots\ Rename,
\[
  B(t) = \lim_{\Delta t \to 0} \frac{b(t, \Delta t)}{\Delta t} \text{ and }
  D(t) = \lim_{\Delta t \to 0} \frac{d(t, \Delta t)}{\Delta t}
\]
and use the definition of derivative to get
\[
  \frac{dx}{dt}(t) = x'(t) = B(t) - D(t)
\]
where $B(t)$ and $D(t)$ being the rates at which the population increasing,
resp. decreases at time $t$. Recall that we assumed that the rates of change in
the population were independent of time and population size. This is equivalent
to saying that $B(t)$ and $D(t)$ are really constants which gives us the final
model
\[
  x'(t) = \beta - \delta \implies x(t) = (\beta - \delta)t.
\]
The previous is a not-particularly-interesting ODE with solution\footnote{For
help with solving differential equations see \cite{math24}.}
\[
  x(t) = (\beta - \delta)x + C.
\]

This model has a lot of shortcomings, first of all, it does not account for the
size of the population in the rates of change. But, one might argue that the
more individuals there are in a population the greater the rates of change are.
We can go back and restate assumption one as \say{population increase, resp.
decrease in the time interval $(t, \Delta t)$ is directly proportional to the
population at time $t$ and the time passed}.  This in turn gives us
\[
  b(t, \Delta t) = \beta x(t) \Delta t \text{ and }
  d(t, \Delta t) = \delta x(t) \Delta t.
\]
Taking the limit as before leads us to the model
\[
  x'(t) = (\beta - \delta)x(t) = px(t).
\]
From now on, we shall let $p = \beta - \delta$ since we don't really need to
distinguish between changes in the population because of births and deaths---we
just care about the overall evolution of the population. This is another ODE,
this time a bit more interesting, with solution
\[
  x(t) = C e^{p t}.
\]

Although a bit better, you can probably see that this model explodes as time
passes since it does not include any provisions for when the population turns
stupidly large. Anyhow, it is common enough that it deserves its own name: the
\textbf{exponential growth model}.

A small step in the right direction would be to account for a population limit
in the system, i.e. number of individuals that flips the rate of growth. More
precisely, let's change assumption one to \say{there is a number $x_M$ that is
  the maximum population in the system (sometimes called the \textit{carrying
capacity} and that the rate of change in population $p(x(t))$ is positive if
$x(t) > x_M$ and negative if $x(t) > x_M$}. The easiest way to model this is
with a linear ansatz for $p(x)$, i.e. something of the form $p(x) = q(x_M - x)$
with a parameter\footnote{The meaning of $q$ is a bit more complex to explain,
but at heart it is just a proportionality constant.} $q > 0$. Notice how
\[
  \begin{cases}
    p(x) > 0 \tif x < x_M,\text{ and } \\
    p(x) < 0 \tif x > x_M
  \end{cases}
  .
\]
Plugging this into our previous model to get
\[
  x'(t) = q(x_M - x(t))x(t) = q x_M x(t) - q x^2(t),
\]
which is our final model for now and is called the
\textbf{logistic growth model}

This ODE can be solved exactly and the solution is
\[
  x(t) = \frac{x_M x_0}{x_0 + (x_M - x_0) e^{- x_M q (t - t_0)}},
\]
where $t_0$ is the initial time and $x_0 = x(t_0)$ is the initial population.


\begin{figure}
  [h]
  \incfig{logistic-growth-model}

  \caption{Solutions for the three iterations of the model, for different
  values for the parameters on each version.}
\end{figure}

We'll stop here for now, but keep in mind that we're missing the second half of
the solution---we still need to apply this models to the real world. This means
\textit{fitting} those curves to the specific problem at hand, in this case,
getting some data (at least two data points) to calculate the constants that
are present in our solutions. Also, recall that we made a lot of assumptions,
there are more population dynamics models that account for changes in the
environment, migrations, etc.



\section{Dimensional analysis and non-dimensionalisation}

The previous models had two or three parameters each, but as we work our way to
more complex examples the number of parameters will increase. Moving around all
those constants is cumbersome and draws our attention away from really
understanding the problem at hand. In addition, as we apply our models to
specific problems we will need to take into account the units of the quantities
we are dealing with.

\subsection{Dimensional analysis}

In addition to being a prerequisite to doing non-dimensionalisation,
dimensional analysis provides a sanity check for us to \textit{make sure we're
not adding apples to oranges}. When coming up with a model, we generally need
to specify the \textit{physical dimension} of the quantities involved, but not
necessarily want to specify the particular units that quantity is expressed in.
For this we will denote by $[c]$ the physical dimension of a quantity $c$. For
example if $t$ denotes time, when we write $[t]$ we mean the \textit{dimension
of time} or \textit{some units of time} but do not specify which.

Revisiting our population model we can define the characteristic units
\[
  T := \text{ time },\qquad \text{ and } \qquad N := \# \text{ of individuals }
\]
so that we get the following dimensions for the involved quantities
\[
  [t] = T,\ [x(t)] = N,\ [x'(t)] = \frac{N}{T}.
\]

We can also do this for the parameters by solving for them in the equation for
the model\footnote{Notice how $[x_M - x] = N$ and not something weird like $N -
N = 0$, since when you subtract apples from apples you still get apples.}
\[
  [x_M] = N,\ [t_0] = T,\ [x_0] = N,\ 
  [q] = \frac{[x']}{[x_M - x][x]} = \frac{N / T}{N \cdot N} = \frac{1}{T}. 
\]

\subsection{Non-dimensionalisation}

Once we know the physical units of all involved quantities in our model we are
ready to choose actual units for our model. For instance, for time we might
choose years, days or hours but most of the time it is better to choose
appropriate units for our problem. Non-dimensionalisation\footnote{Yes, this is
an accepted spelling although not very common in the literature.} is a recipe
for choosing the most appropriate units.

From the dimensional analysis of our population dynamics examples we now that
there are two physical dimensions and therefore we will choose two
characteristic quantities $\overline{t}$ and $\overline{x}$.

\subsection{Non-dimensionalisation when there are several options. The
projectile problem.}

\section{Asymptotic expansion method}

\subsection{Error estimation}

  \chapter{Linear systems of equations}

\section{Modelling electrical netwroks}

  \chapter{Ordinary differential equations}

Teaching started on Monday 2019.11.25 (week 48a).

This chapter corresponds to part of chapter 4 in \cite{eck2017mathematical}.

\section{Quantitative analysis of models in population dynamics}

Recall from week 46 (chapter 1) that we had two models for population dynamics.

\begin{itemize}
  \item The first one, the exponential model was described by
    \[
      x'(t) = px(t),\quad p \in \R,
    \]
    where $p$ was the growth rate.
  \item The second, the constrained model was described by
    \[
      x'(t) = qx_M x(t) - q x^2(t),\quad q, x_M \in \R
    \]
    where $q > 0$ was the growth rate and $x_M$ was the maximum carrying
    capacity of the environment in number of individuals.
\end{itemize}

Both of these models share the common mathematical structure of an autonomous
equation, i.e. an equation of the form
\begin{align}
  \label{eq:autonomous-ode}
  x'(t) = f(x(t)),
\end{align}

where $f$ (read $x'$) does not depend explicitly\footnote{That is, $f$ cannot
\textit{unwrap} $t$ out of $x(t)$ and do anything with it alone, it has to work
on $x(t)$ as its variable.} on $t$.

In this section we will focus on the qualitative aspects of the model, i.e.
what information can we get from it without explicitly solving the equations
(which in this case we can, but in the next examples we won't).

Recall that a \textbf{stationary solution} of an ODE is one that stays constant
in time, i.e. of the form $x(t) = c$. How can we find them? Easy, if $x$ is
constant then we must have $x'(t) = f(x(t)) = 0$. For our previous models this means
\begin{itemize}
  \item $x(t) = 0$ for the exponential model, and
  \item $x(t) = 0$ or $x(t) = x_M$ for the second model. We get these two
    solutions from solving
    \[
      x'(t) = qx_M x(t) - qx^(t) = 0
    \]
    for $x(t)$ using the well known quadratic formula.
\end{itemize}

We are interested in these solutions because they are predictable and
\textit{don't blow up} as time passes. Later in this chapter we will formally
define the concept of stability and quantify how stable solutions are based on
how close they are to the stationary solutions.

\subsection{Introduction to linear stability analysis}

For now we will settle with something called \textbf{linear stability
analysis}. The main idea is to linearise the solution (i.e. Taylor expand up to
degree 1) a stationary solution. Let $x^*$ be a stationary solution to an
autonomous problem of the form \ref{eq:autonomous-ode}. The linear expansion we are talking about is
\[
  f(x) = f(x^*) + f'(x^*)(x - x^*) + O(\abs{x - x^*}).
\]
To make things easier, let us take $y(t) = x(t) - x^*(t)$. We shall ignore the
error term $O(\abs{x - x^*}) = O(y(t))$ and thus we get
\begin{align}
  \label{eq:stationary-linear-exp}
  y'(t) = f'(x^*) y(t).  
\end{align}
Now, \ref{eq:stationary-linear-exp} is trivial to solve explicitly---it is a
linear homogeneous equation with constant coefficients
\[
  y(t) = c e^{f'(x^*) t}.
\]
Intuitively, as $t \to \infty$ we have
\[
  \abs{y(t)} \to 0  \implies \abs{x(t) - x^*(t)} \to 0 \iff x(t) \to x^*(t),
\]
i.e. the linearised solution $y(t)$ converges to the stationary solution
$x^*(t)$.

More on this later.

\section{Predator--prey models}

\subsection{Derivation of the Lotka--Volterra equations}

Now we turn our attention to environments where there are two species and one
eats/hunts/harvests the other. Let us model this from scratch to get yet
another example of how things work in Mathematical modelling. For this
derivation we shall use the following assumptions.

\begin{enumerate}
  \item The prey population has unlimited resources available for its growth all the time.
  \item The predator population feeds exclusively on the prey population.
  \item TODO: Something i cant remember.
  \item The rate of growth of the populations is proportional to their size.
  \item The environment is stable over time.
\end{enumerate}


We now proceed with the standard recipe for deriving models.

\begin{enumerate}
  \item \textbf{Name the quantities involved in the problem.} We are trying to
    model how two populations change over time so we need
    \begin{align*}
      \begin{array}
        {cl}
        t &:= \text{ time} \\
        x_1(t) &:= \text{ size of prey population at time } t \\
        x_2(t) &:= \text{ size of predator population at time } t
      \end{array}
    \end{align*}

  \item \textbf{Find relations between the quantities.} From assumption X we
    now that the growth of both species is directly proportional to the size of
    the populations, in other words
    \[
      x_1' = p_1 x_1\text{ and } x_2' = p_2 x_2.
    \]
    A priori, we don't know if $p_1$ depends only on $t$ or on $x_2(t)$ or on
    both. The possibility that $p_1$ depends on $x_1(t)$ is ruled out by the
    assumption that growth is proportional to size. Looking at the assumptions
    once more we find that $p_1$ cannot depend on $t$ since ``the environment
    is stable over time''. There fore it must be that $p_1$ is a function only
    of $x_2(t)$, which really makes sense, since the size of the prey species
    depends on how many individuals are being eaten by the predator species. A
    similar argument for $p_2$ yields
    \[
      p_1(x_2(t))\text{ and } p_2(x_1(t)).
    \]
    
    But what do these functions $p_1$ and $p_2$ look like? Well, the prey
    population naturally grows since we assumed unlimited resources but at the
    same time it is being eaten at some rate by the predator population.
    Similarly, the predator population naturally dies unless they can feed on
    the prey population. We introduce the parameters $\alpha, \beta, \gamma,
    \delta > 0$ and formalise these relations with
    \begin{align*}
      p_1(x_2(t)) = - \beta x_2(t) + \alpha 
      \qquad \text{ and } \qquad
      p_2(x_1(t)) = \delta x_1(t) - \gamma.
    \end{align*}

    Finally we get our model, commonly referred to as the Lotka--Volterra
    equations, derived independently by both authors from around 1920 to around
    1925 \cite{lotka-volterra}.
   
    \begin{align}
      \label{eq:lotka-volterra}
      \begin{cases}
        x_1' &= (\alpha - \beta x_2) x_1 \\
        x_2' &= (\delta x_1 - \gamma) x_2
      \end{cases}
      .
    \end{align}
\end{enumerate}

The mathematical structure of this problem is that of an autonomous planar
system of ODEs. We may rewrite it as
\begin{align}
  \label{eq:autonomous-planar-sys}
  \begin{cases}
    \vec{x}'     &= f(\vec{x}) \\
    \vec{x}(t_0) &= \vec{x_0}
  \end{cases}
\end{align}
with $f : \Omega \subseteq \R^2 \to \R^2$. If $f$ is sufficiently nice (i.e.
locally Lipschitz) then an initial value problem of the form in
\ref{eq:autonomous-planar-sys} has locally unique solutions. However, it is not
the explicit solutions that interest us right now, but rather the qualitative
aspects of their behaviour.

\subsection{Qualitative analysis of the Lotka--Volterra equations}

We look into the stationary solutions to later look at stability. Once more,
setting $x_1, x_2 = 0$ in \ref{eq:lotka-volterra} gives us the stationary
solutions

\begin{align*}
  \left(
    \begin{array}
      {c}
      x_1 \\ x_2
    \end{array}
  \right) = \left(
    \begin{array}
      {c}
      0 \\ 0
    \end{array}
  \right) \qquad \text{ and } \qquad \left(
    \begin{array}
      {c}
      x_1 \\ x_2
    \end{array}
  \right) = \left(
    \begin{array}
      {c}
      \alpha / \beta \\ \gamma / \delta
    \end{array}
  \right).
\end{align*}

There are too many parameters to work comfortably with this solutions. Let us
non-dimensionalise before moving on to get rid of as many parameters as we can.
Very quickly, we choose the fundamental dimensions $T$ for time and $N$ for
number of individuals and carry out a dimensional analysis to get
\[
\begin{array}{rlcrl}
  [t]                   & = T                                   & \qquad & [x_1] = [x_2]                                & = N \\
  \left[ \alpha \right] & = \left[ \gamma \right] = \frac{1}{T} &        & \left[ \beta \right] = \left[ \delta \right] & = \frac{1}{NT}
\end{array}
.
\]
We choose the characteristic quantities $\overline{x_1}, \overline{x_2}$ and
$\overline{t}$ and set up the change of variables
\[
  z_1 = \frac{x_1}{\overline{x_1}},\qquad
  z_2 = \frac{x_2}{\overline{x_2}} \qquad \text{ and } \qquad
  \tau = \frac{t}{\overline{t}}.
\]

Substitute with care in \ref{eq:lotka-volterra} (careful with the derivatives) to get
\[
  \begin{cases}
    z_1' &= \overline{t} \alpha z_1 - \beta \overline{x_2} \overline{t} z_1 z_2 \\
    z_2' &= \delta \overline{x_1}\overline{t} z_1 z_2 - \gamma \overline{t} z_2
  \end{cases}.
\]
Notice how we have four different coefficients for $z_1$ and $z_2$ but only
have three characteristic quantities. This means we'll need to make a
compromise. Which one to make is dictated by our taste and the mathematical or
biological interpretation of the parameters we choose. We will not do all four
options here but the Lotka--Volterra equations often come with
\[
  \overline{t} \alpha = 1,
  \qquad \beta \overline{x_2} \overline{t} = 1 \qquad \text{ and } \qquad
  \delta \overline{x_1} \overline{t} = \gamma \overline{t},
\]
which in turn give us the \textbf{non-dimensionalised version of the
Lotka--Volterra equations}
\begin{align}
  \label{eq:non-dim-lotka-volterra}
  \begin{cases}
    z_1' &= (1 - z_2) z_1 \\
    z_2' &= a (z_1 - 1) z_2
  \end{cases},
\end{align}
where there is only one parameter $a = \gamma / \delta$. 


  \chapter{Calculus of variations}

\begin{dfn}
  [Variational problem]

  Let $X$ be a real vector space (of functions, possibly infinite dimensional),
  $\calA \subset X$ a set of admisible functions and $\calI : X \to \R$ a
  functional that assigns a real number for each $u \in X$.

  A variational problem is the task
  \[
    \text{minimise } \calI(u),\quad u \in \calA.
  \]
\end{dfn}

In particular we are concerned with variational problems of integral form, i.e.
those where
\begin{align}
  \label{eq:cv-integral}
  \calI(u) = \int_a^b f(x, u(x), u'(x))dx,
\end{align}
where $u : [a, b] \to \R^m$ and $f: [a,b] \times \R^m \times \R^m \to \R$.

As mathematicians, we are immediately concerned about the exitence and
uniqueness of solutions. Until 1850, it was thought that minimisers for
integral problems always existed but Weierstrass gave a counter example. From
that moment on, a new theory for solving variational problems, now known as the
direct method was developed. In this section we will mostly concentrate on the
classical method for finding minimisers which basically replicates the process
of finding minima for functions in vector calculus. The reason for this is that
a formal treatment of the direct method requires advanced mathematical tools
from functional analysis, which is not a prerequisite for this course.

\section{The classical method}

This method was developed by Euler and Lagrange during the 18th century. The
strategy for dealing with variational problems of the form
\eqref{eq:cv-integral} is to just go ahead and find the minima of the function
$\calI(u)$. For this, let us recall the approach taken in vector calculus to minimise a function $f: \R^m \to \R^n$.
\begin{enumerate}
  \item Find $\overline{x} \in \R^m$ such that $Df(\overline{x}) = \0$,
  \item Find $D^2f(\overline{x})$, and
  \item Check that $D^2f(\overline{x})$ is positive definite.
\end{enumerate}

The problem with this strategy is that we don't know if $Df$ or $D^2f$ will
exist for our functional $f = \calI$. Therefore, we will introduce a weaker
notion of derivative, the variation, which is easier to work with in the
context of these problems.

\begin{dfn}
  [Admissible perturbation]

  Let $X$ and $\calA$ be a function space and a set of admissible functions,
  resp. Let $u \in \calA$. For any $\varphi \in X$ we say $\varphi$ is an
  admissible perturvation of $u$ iff there exists a $\varepsilon_0 > 0$ such
  that
  \[
    u + \varepsilon \varphi \in A,\quad \text{ for every } \varepsilon \in
    (-\varepsilon_0, \varepsilon_0).
  \]
\end{dfn}


\begin{dfn}
  [Variation]

  Let $X = \{ f : A \to B \}$ be a function space, $u, \varphi \in X$ and $x
  \in A$. We define the variation of $u$ at $a$ in the direction of $\varphi$
  as
  \begin{align}
    \label{eq:variation}
    \delta u(a)(\varphi) = \left. \frac{d}{d\varepsilon}\right|_{\varepsilon =
    0} u(a + \varepsilon \phi).
  \end{align}
\end{dfn}


\section{Exercises}

\begin{ex}
  [Dido's problem]

  Let $L > 0$ be a given length. We consider the maximisation problem
  \[
    \text{maximise } \int_0^L u(s)\sqrt{1 - u'(s)^2}ds, \text{ for } u \in \calA,
  \]
  where $\calA = \{C^1((0, L)) \cap C^0([0, L]) : u(0) = 0, u(L) = 0,
  \abs{u'(s)} \text{ for } s \in (0, L)\}$, which emerges from modeling Dido's
  problem.

  \begin{enumerate}
    \item Determine the corresponding Euler-Lagrange equation and find a
      non-negative solution $\overline{u} \in \calA \cap C^2((0, L))$.
    \item Sketch the curve $\{(\varphi(s), \overline{u}(s)) : s \in [0, L]\}$
      with $\varphi(s) = \int_0^s \sqrt{1 - \overline{u}'(\tau)}d\tau$ for $s
      \in [0, L]$.
    \item Interpret b) in the context of Dido's problem.
  \end{enumerate}

  \textit{Hint for a):} \footnote{In the original statement for this problem,
  the hint was only a simple implication, but by proving it one realised that
it was a double implication.}Prove and use the following statement: Let $a, b
\in \R$ with $a < b,\ u \in C^2((a, b))$ and $f \in C^2(\R \times \R)$ such
that $\partial_p f(u, u') \in C^1(a, b)$. Then
  \[
    \ddx \partial_p f(u, u') = \partial_z f(u, u')\text{ in } (a, b)
  \]
  if, and only if, there exists $c \in \R$ such that
  \[
    f(u, u') - u' \partial_p f(u, u') = c\text{ in } (a, b).
  \]
\end{ex}

\begin{proof}
  [Proof of the hint]
  Not very precise but something along the lines of
  \begin{align*}
         & \ddx \partial_p f(u, u') = \partial_z f(u, u') \\
    \iff & \partial_z f(u, u') - \ddx \partial_p f(u, u') = 0 \\
    \iff & \int \partial_z f(u, u') - \int \ddx \partial_p f(u, u') = \int 0 \\
    \iff & f(u, u') - u' \partial_p f(u, u') = c.
  \end{align*}

  Or, with derivatives,
  \begin{align*}
    & f(u, u') - u'\partial_p f(u, u') = c \\
    \iff & \ddx f(u, u') - \ddx u' \partial_p f(u, u') = \ddx c \\
    \iff & \partial_z f(u, u') u' + \partial_p f(u, u')u'' - u'' \partial_z f(u, u') - u' \ddx \partial_p f(u, u') = 0 \\
    \iff &u' \left( \partial_z f(u, u') - \ddx \partial_p f(u, u')\right) = 0.
  \end{align*}
  In this case, if $u' \neq 0$ we have
  \[
  \partial_z f(u, u') - \ddx \partial_p f(u, u') = 0 \iff
  \ddx \partial_p f(u, u') = \partial_z f(u, u').
  \]
  Otherwise, we go back to the original equation
  \[
    f(u, u') - 0 \partial_p f(u, u') = c \iff  f(u, u') = c
  \]
  which means that $f$ is constant in $u$ and therefore in $x$ so it is trivial to see that
  \[
    \partial_p f(u, u') = 0 = \partial_z f(u, u'),
  \]
  which gives the first equation.
\end{proof}


\begin{proof}
  For the purposes of determining the Euler-Lagrange equation we have $x = s,\
  z = u(x),\ p = u'(s)$ and $f(x, z, p) = z\sqrt{1 - p^2}$. The involved
  partial derivatives are
  \[
    \partial_z f(x, z, p) = \sqrt{1 - p^2} \text{ and }
    \partial_p f(x, z, p) = - \frac{zp}{\sqrt{1 - p^2}},
  \]
  and thus the Euler-Lagrange equation becomes
  \[
    \ddx \frac{- u u'}{\sqrt{1 - u'^2}} = \sqrt{1 - u'^2}.
  \]
  This looks complicated so we apply the hint:
  \begin{align*}
    & u\sqrt{1 - u'^2} + u' \frac{uu'}{\sqrt{1 - u'^2}} = c \\
    \iff & u \left( \sqrt{1 - u'^2} + \frac{u'^2}{\sqrt{1 - u'^2}}\right) = c \\
    \iff & u \left( \frac{1 - u'^2 + u'^2}{\sqrt{1 - u'^2}}\right) = c \\
    \iff & u = c\sqrt{1 - u'^2} \\
    \iff & u' = \sqrt{1 - \left( \frac{u}{c} \right)^2}.
  \end{align*}
  That last equation desperately screams for separation of variables and a trigonometric change of variable:
  \begin{align*}
    & \frac{du}{ds} = \sqrt{1 - \left( \frac{u}{c} \right)^2} \\
    \iff &\frac{du}{\sqrt{1 - (u/c)^2}} = ds \\
    \iff & \int \frac{1}{\sqrt{1 - (u/c)^2}} du = \int ds.
  \end{align*}
  Change $u / c = \sin y \implies u = c\sin y \implies du = c\cos y dy$ to get
  \begin{align*}
    \int \frac{1}{\sqrt{1 - (u/c)^2}} du
  & = \int \frac{1}{1 - \sin^2 y} c \cos y dy \\
  &= c \int \frac{\cos y}{\cos y} dy = cy \\
  &= c\arcsin\frac{u}{c}
  \end{align*}
  Plugging it back into the hint,
  \[
    s + k = c\arcsin\frac{u}{c} \implies u(s) = c\sin \frac{k+s}{c},
  \]
  which we rewrite picking different constants,
  \[
    u(s) = k_1 \sin( k_2 s + k_3),
  \]
  where the constants $k_1, k_2, k_3$ are obtained by enforcing $u \in \calA$.
  More specifically, we require
  \[
    u(0) = 0 \implies k_3 = 0\text{ and } u(L) = 0 \implies k_2 L = n\pi.
  \]
  Moreover, we require $u$ to be non-negative, therefore $n = 1$ and thus $k_2
  = \frac{\pi}{L}$ (otherwise the sine would go negative). Finally, we want
  \[
    \abs{u'(s)} < 1 \implies \abs{k_1\cos(k_2s + k_3) k_2} < 1
    \implies k_1 k_2 < 1 \implies k_1 < \frac{L}{\pi},
  \]
  since we require that $k_1$ is positive so that $u(s)$ also is. This final
  parameter is fixed by maximising $\calI(u)$:

  TODO
\end{proof}


\begin{ex}
  [Geodesics in $\R^2$]

  Let $A$ and $B$ be two points in the plane. What is the shortest connection between $A$ and $B$?

  \begin{enumerate}
    \item Set up the variational problem to model the situation.
    \item Solve the problem and interpret the result.
  \end{enumerate}
\end{ex}

\begin{proof}
  Let $X = C^1([0,1]; \R^2)$ and $\calA = \{ u \in X \mid u(0) = A,\ u(1) =
  B\}$. We define our functional $\calI$ as the length of the parametrised
  curve $u$ as follows
  \[
    \calI(u) = \int_0^1 \norm{u'(t)} dt.
  \]
  Our variational problem is
  \[
    \text{minimise } \calI(u) \text{ for } u \in \calA.
  \]

  To solve it we use the Euler-Langrange method. We have $f(t, u(t), u'(t) =
  \norm{u'(t)}$ and, in the form of the Euler-Langrange equations, we get $f(x,
  z, p) = \norm{p}$. Therefore,
  \[
    \delta_z f = 
    \begin{pmatrix}
      0 & 0
    \end{pmatrix}
    \text{ and } \delta_p f =
    \begin{pmatrix}
      \frac{p_1}{\norm{p}} & \frac{p_2}{\norm{p}}
    \end{pmatrix}.
  \]
  Notice how $p = u' : [0, 1] \to \R^2$ so by $\delta_p f$ we really mean the
  last two numbers in $Df$ (which is a row matrix since $f$ is real valued). We
  arrive at the following Euler-Lagrange equation
  \[
    \ddt \delta_p f = \delta_z \iff \ddt
    \begin{pmatrix}
      \frac{u_1}{\norm{u}} & \frac{u_2}{\norm{u}} 
    \end{pmatrix}
    =
    \begin{pmatrix}
      0 & 0
    \end{pmatrix}.
  \]
  Instead of taking the derivative with respect to $t$, we may simply rewrite
  this as
  \[
  \begin{pmatrix}
    \frac{u_1}{\norm{u}} & \frac{u_2}{\norm{u}}
  \end{pmatrix}
  =
  \begin{pmatrix}
    c_1 & c_2
  \end{pmatrix},
  \]
  where $c_1, c_2 \in \R$ are constants.

  Therefore
  \[
    u(t) = \int u'(t) dt = (c_1 t, c_2 t) + u_0,\ u_0 \in \R^2,
  \]
  which is a parametrisation for a curve. The parameters $c_1, c_2$ and $u_0$
  are determined by enforcing $u \in \calA$:
  \[
    u(0) = u_0 = A,\ u(1) = (c_1, c_2)  + u_0 = B.
  \]
\end{proof}

	
	
	% \listoftheorems[ignore={dfn,ex,eg,remark,lem,cor}]
	
	\bibliographystyle{IEEEtran}
	\bibliography{IEEEabrv,mm-notes}
\end{document}