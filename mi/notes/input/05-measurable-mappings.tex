% !TeX root = ../mi-notes.tex

\chapter{Measurable mappings}

In mathematics mappings between sets are a central topic. Moreover, when sets have a specific structure, we want mappings that preserve the that same structure between the two sets. For instance we have
\begin{itemize}
	\item groups and homomorphisms, which hold the group operation: $f(a \cdot b) = f(a) \cdot f(b)$;
	\item topological spaces and continuous functions, which hold the topology of the spaces in question: for any open set $V$ and continuous function $f$, $\inv{f}(V)$ is also an open set;
	\item and naturally we wish that between measurable spaces, there are appropriate mappings that preserve the measurable structure (\siga).
\end{itemize}

Let's dive right into it.

\section{Definition. Properties.}

\begin{dfn}[Measurable map]
	Let $(X, \sa)$ and $(X', \sa')$ be measurable spaces. A map $T : X \to X'$ is said to be $\sa/\sa'$-measurable (or measurable unless this is too ambiguous) if
	\begin{align*}
		\inv{T}(A') = \{x \in X \mid T(x) \in A'\} \in \sa,\qquad \forall A' \in \sa',
	\end{align*}
	i.e. if the preimage of every measurable set in $\sa'$ is a measurable set in $\sa$.
\end{dfn}

In the next chapter we will particularise this to mappings $T:X \to \R$ and measurable spaces $(X, \sa)$ and $(\R, \borel(\R^n))$ to pave the way for integration.

\begin{remark}$ $\newline
	\begin{enumerate}
		\item In probability theory, a measurable map is usually called a random variable.
		\item Consider the collection $\inv{T}(\sa') = \{\inv{T}(A') \mid A' \in \sa'\}$. Recall \autoref{eg:preimage-siga}, in which we proved that the preimage of a \siga is a \siga. We can rephrase the definition of measurability as
		\begin{align*}
			T \text{ is measurable } \iff \inv{T}\sa' \subset \sa
		\end{align*}
		\item If we write $T: (X, \sa) \to (X', \sa')$ then we usually mean that $T$ is $\sa/\sa'$-measurable.
		
		\item If $T: (\R^n, \borel(\R^n)) \to (\R, \borel(\R))$, then we simply say $T$ is Borel-measurable.
	\end{enumerate}
\end{remark}

\begin{lem}[Measurability can be checked only on the generators]
	\label{lem:measurable-generators}
	Let $(X, \sa)$ and $(X', \sa')$ be two measurable spaces with $\sa' = \sigma(\calG')$. A map $T: X \to X'$ is $\sa/\sa'$-measurable $\iff \inv{T}(G') \in \sa,\ \forall G' \in \calG' \iff \inv{T}(\calG') \subseteq \sa$.
\end{lem}

\begin{proof}
	If $T$ is $\sa/\sa'$ measurable then we have that $\inv{T}(A') \in \sa,\ \forall A' \in \sa$ so, in particular, $\inv{T}(G') \in \sa,\ \forall G' \in \calG \subset \sa'$.
	
	For the converse let us define
	\begin{align*}
		\Sigma' := \{ A' \subset X' \mid \inv{T}(A') \in \sa \}
	\end{align*}
	As usual we will check that $\Sigma'$ is itself a \siga on $X'$ and therefore
	\begin{align*}
		A' = \sigma(\calG') \subseteq \sigma(\Sigma') = \Sigma' \implies \inv{T}(A') \in A,\ \forall A' \in \sa'.
	\end{align*}
	
	Let's verify that $\Sigma'$ is indeed a \siga on $X'$.
	\begin{enumerate}
		\item $X' \in \Sigma'$ since $\inv{T}(X') = X \in \sa$
		\item For any $A' \in \Sigma'$ we have that $\inv{T}(A'^\complement) = \inv{T}(X' \setminus A') = \inv{T}(X') \setminus \inv{T}(A') = X \setminus \inv{T}(A') \in \sa$ since $\inv{T}(A') \in \sa$ by hypothesis.
		\item For any collection $(A'_n)_{n\in\N} \subset \Sigma'$ we have that
		\begin{align*}
			\inv{T}\left(\bigcup_{n\in\N} A'_n\right) = \bigcup_{n\in\N} \inv{T}(A_n') \in \sa,
		\end{align*}
		since $\inv{T}(A_n') \in \sa$ by hypothesis.
	\end{enumerate}
\end{proof}

So it is enough to check measurability on the generators only. This leads us to the following remark.

\begin{remark}
	Any continuous function $f: \R^n \to \R^m$ is a measurable mapping.
\end{remark}

\begin{proof}
	Recall that $f$ is continuous if and only if for any open set $V \subset \R^m$ then $\inv{f}(V) \in \R^n$ is also open. Recall that $\borel(\R^m)$ is also generated by the open sets $\OS^m$ hence if $f$ is continuous, $\inv{f}(V) \in \OS^n = \borel(\R^n) \implies f$ is a measurable map.
\end{proof}

Keep in mind that not every measurable map is continuous.

\begin{eg}[A non-continuous measurable map.]
	Let $f:(\R, \borel(\R)) \to (\R, \borel(\R))$ be a map defined by $f(x) = \ind_{A}$ for some $A \subset \R$. This map is clearly not continuous as $\ind_A$ takes only two values, 0 and 1. However, it is indeed measurable. Take any $a \in \R$. Then,
	\begin{align*}
		\inv{\ind_A}((a, \infty)) = \{x \in \R \mid \ind_A(x) > a\} =
		\begin{cases}
		\emptyset \tif a \geq 1 \\
		A \tif 0 \leq a < 1 \\
		\R \tif a < 0
		\end{cases} \in \borel(\R)
	\end{align*}
	
	By \autoref{lem:measurable-generators} we have that $f$ is Borel measurable.
\end{eg}

Before we move on, lets prove that the composition of measurable maps is another measurable map, as we would do with any other structure-preserving map in any other field of mathematics.

\begin{lem}
	Let $(X_i, \sa_i)$ be measurable spaces for $i = 1, 2, 3$ and $T: X_1 \to X_2,\ S: X_2 \to X_3$ be two $\sa_1/\sa_2$- resp. $\sa_2/\sa_3$-measurable maps. Then $S \circ T: X_1 \to X_3$ is $\sa_1/\sa_4$-measurable.
\end{lem}

\begin{proof}
	We need to show that $\inv{(S \circ T)}(A) \in \sa_1$ for any $A \in \sa_3$. Now $\inv{(S \circ T)}(A) = \inv{T}(\inv{S}(A))$ with $\inv{S}(A) \in \sa_2$ and hence $\inv{T}(\inv{S}(A)) \in \sa_1$.
\end{proof}

\section{\sigas in relation to measurable maps. Image measures.}

When dealing with measurable maps, we often have a \siga on the codomain but none is given for the domain. Naturally, we want to know which \sigas on $X$ render a map $T:X \to X'$ measurable when $(X', \sa')$ is the destination measurable space.

It is clear that if we take $\sa = \powerset(X)$ to be the \siga on $X$ then any map is measurable since
\begin{align*}
	\inv{T}(A') \subset X \implies \inv{T}(A') \in \powerset(X),\ \forall A' \in \sa'.
\end{align*}
Also, from the examples on chapter 1 know that the preimage of a \siga is another \siga. This would then be the smallest \siga on $X$ that makes $T$ measurable but we cannot remove any sets from $\inv{T}(\sa')$ without endangering the measurability of $T$.

What if we have many mappings to different measurable spaces that share the same origin set? We cannot guarantee that taking the union of all the \sigas which individually render each mapping measurable is again a \siga. Let us formalise this.

\begin{dfn}
	Let $(T_i)_{i \in I}$ be arbitrarily many mappings $T_i: X \to X'$ from the same set $X$ into measurable spaces $(X_i, \sa_i)$. The smallest \siga on $X$ that makes all $T_i$ simultaneously measurable is
	\begin{align*}
		\sigma(T_i : i \in I) := \sigma \left(\bigcup_{i \in I} \inv{T_i}(\sa_i)\right).
	\end{align*}
	
	We say that $\sigma(T_i : i \in I)$ is generated by the family $\inv{T_i}(\sa_i)$.
\end{dfn}

\begin{lem}
	The previous definition makes sense, i.e. $\sa = \sigma(T_i \mid i \in I)$ is indeed a \siga which renders all $T_i$ simultaneously $\sa/\sa_i$ measurable.
\end{lem}

\begin{proof}
	Let $A_i \in \sa_i$ for any $i \in I$. Then, clearly, $\inv{T_i}(A_i) \in \inv{T}(\sa_i)$, so any $T_i$ is measurable. But is $\sa$ a \siga? Well, that is the reason why we included the $\sigma$-hull in the above definition, to guarantee that the union does not break the \siga structure.
\end{proof}

To tie it back to measures we will give the following definition:

\begin{dfn}[Image measure]
	Let $(X, \sa),\ (X', \sa')$ be measurable spaces and $T:X \to X'$ be an $\sa/\sa'$-measurable map. Then, for every measure $\mu$ on $(X, \sa)$, we define the image measure of $\mu$ under $T$, denoted by $T(\mu)$ or $\mu \circ \inv{T}$, by
	\begin{align*}
		T(\mu)(A') = \mu(\inv{T}(A')),\quad \forall A' \in \sa'
	\end{align*}
\end{dfn}

\begin{lem}
	The image measure is indeed a measure.
\end{lem}

\begin{proof}
	We just have to check the two properties of measures.
	\begin{enumerate}
		\item Clearly $T(\mu)(\emptyset) = \mu(\inv{T}(\emptyset)) = \mu(\emptyset) = 0$.
		\item For any pairwise disjoint collection $(A_n')_{n \in \N} \subset \sa'$ we have
		\begin{align*}
			T(\mu)\left(\bigcupdot_{n \in \N} A_n'\right) &= \mu\left(\inv{T}\left(\bigcupdot_{n\in\N} A_n'\right)\right)\\
			&= \mu\left(\bigcupdot_{n \in \N} \inv{T}( A_n')\right) \\
			&= \sum_{n \in \N} \mu(\inv{T}(A_n')) = \sum_{n \in \N} T(\mu)(A_n')
		\end{align*}
	\end{enumerate}
\end{proof}

\section{Exercises}

\begin{eg}
	This is a version of exercise 7.6 that I think is more precisely formulated, partly inspired by \cite{MO1}.
	
	Let $(X, \sa)$ and $(X', \sa')$ be measurable spaces and $T: X \to X'$ be a surjective, $\sa/\sa'$-measurable map. Then $T(\sa)$ is a \siga $\iff \inv{T}$ is $\sa'/\sa$ measurable.
\end{eg}

\begin{ex}
	Let $T:(X, \sa) \to (X, \sa')$ be a measurable map. Under which circumstances is the family of sets $T(\sa)$ a \siga?.
\end{ex}

\begin{proof}
	We will see that the first and third properites of \sigas hold more or less trivially. It is the closing under complements that gives us problems.
	
	We claim that $T(\sa)$ is a \siga $\iff \inv{T}: X' \to X$ is a measurable map. Let us assume that $T$ is surjective, as when we define $\inv{T}:X' \to X$ we are implicitly denoting that. Otherwise, we can just redefine $X' = T(X)$ and move on.
	
	For the reverse implication let's verify the properties of a \siga over $X'$ on $T(\sa)$.
	
	\begin{enumerate}
		\item First, we need to show that $X' \in T(\sa)$. This is clear from the fact that $X \in \sa$ and $T$ is surjective.
		
		\item For any $A' \in T(\sa)$ we need to show that $A'^\complement \in T(\sa)$. Since $T$ is $\sa/\sa'$ measurable, $A = \inv{T}(A') \in \sa$. Since $\sa$ is a \siga, we have that $A^\complement$ is in $\sa$. Moreover, since $\inv{T}$ is measurable then $\inv{T}(A^\complement) = \inv{T}(\inv{T}(A)^\complement) = \inv{T}(\inv{T}(A^\complement)) = T(A^\complement) \in T(\sa)$.
		
		\item Finally, for any collection $(A_n')_{n \in \N} \subset T(\sa)$ we need to show that $\bigcup_{n \in \N} A_n' \in T(\sa)$. For each $A_n' \in T(\sa)$ we have a set $A_n \in \sa$ such that $A_n' = T(A_n)$. Thus,
		\begin{align*}
			\bigcup_{n \in \N} A_n' = \bigcup_{n \in \N} T(A_n) = T\left(\bigcup_{n \in \N} A_n\right) \in T(\sa),
		\end{align*}
		since $\bigcup_{n \in \N} A_n \in \sa$.
	\end{enumerate}
\end{proof}