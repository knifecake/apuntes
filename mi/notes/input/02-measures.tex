% !TeX root = ../mi-notes.tex

\chapter{Measures}

\section{Definition. Properties.}

\begin{dfn}[Measure]
	\label{dfn:measure}
	Let $(X, \sa)$ be a measurable space. A set function $\mu: \sa \to [0, \infty)$ is called a \underline{measure} (on X) if
	\begin{enumerate}
		\item $\mu(\emptyset) = 0$ and
		\item \textbf{($\sigma$-aditivity)} if $(A_n)_{n\in \N} \subset \sa$ is a pairwise disjoint (i.e. $A_i \cap A_j = \emptyset$ if $i \neq j$) sequence, then
		\begin{align*}
			\mu\left(\bigcupdot_{n \in \N} A_n\right) = \sum_{n\in \N} \mu(A_n)
		\end{align*}
	\end{enumerate}
\end{dfn}

And, analogously to \sigas

\begin{dfn}[Measure space]
	Let $(X, \sa)$ be a measurable space and $\mu$ a measure. The triple $(X, \sa, \mu)$ is called a \underline{measure space}.
\end{dfn}

In the definition of measure (\ref{dfn:measure}), the symbol $\bigcupdot$ indicates that the union is disjoint.

Some special measures get cool names, for instance:

\begin{itemize}
	\item If $\mu(X) < \infty$ we say that $\mu$ is \textbf{finite} and that $(X, \sa, \mu)$ is a \textbf{finite measure space}.
	\item IF $\mu(X) = 1$ then $(X, \sa, \mu)$ is a probability space and we usually denote it by $(\Omega, \sa, \Pro)$.
\end{itemize}

From the two conditions on \ref{dfn:measure}, one can derive many properties, but before, let us introduce some notation.

Let $(A_n)_{n \in \N}$ be a sequence of sets. We say that $(A_n)$ is an \textbf{increasing}, resp. \textbf{decreasing}, \textbf{sequence} and denote it by $A_n \uparrow A$, resp. $A_n\downarrow A$ according to the following definition.
\begin{align}
	A_n \uparrow A \iff A_1 \subseteq A_2 \subseteq \dots &\text{ and } A = \bigcup_{n \in \N} A_n \\
	A_n \downarrow A \iff A_1 \supseteq A_2 \supseteq \dots &\text{ and } A = \bigcap_{n \in \N} A_n
\end{align}

We say that an increasing sequence $A_n \uparrow A$ is an \textbf{exhausting sequence} if $A_n \subseteq X$ and $A = X$. We write $A_n \uparrow X$ for an exhausting sequence.


\begin{dfn}[$\sigma$-finite measure]
	\label{dfn:sigma-finite}
	A measure $\mu$ is called $\sigma$-finite if there exists an exhausting sequence $(A_n)_{n \in \N} \subset \sa$ such that $\mu(A_n) < \infty,\ \forall n \in \N$. A measure space with this kind of measure is called a $\sigma$-finite measure space.
\end{dfn}

\begin{remark}
	Finiteness is stronger than $\sigma$-finiteness\footnote{By $\sigma$-finiteness we mean that a measure satisfies \autoref{dfn:sigma-finite}, not that it is $s$-finite, which we wont see in this course. See \cite{WSF} for more details.}. Namely, any finite measure is $\sigma$-finite since one can choose the sequence $A_n = X,\ \forall n \in \N$ which is an exhausting sequence $A_n\uparrow X$ and $\mu(A_n) < \infty,\ \forall n \in \N$. On the other hand, an example of a $\sigma$-finite measure which is not finite is the Lebesuge measure (see \autoref{eg:lebesgue-measure}).
\end{remark}

\begin{thm}[Properties of measures]
	\label{thm:prop-measures}
	Let $(X, \sa, \mu)$ be a measure space and $A, B, A_n, B_n \in \sa,\ \forall n \in \N$. Then,
	\begin{enumerate}
		\item \textbf{(finite additivity)} $A \cap B = \emptyset \implies \mu(A \cupdot B) = \mu(A) + \mu(B)$,
		\item \textbf{(monotonicity)} if $A \subseteq B$ then $\mu(A) \leq \mu(B)$,
		\item if $A \subset B$ and $\mu(A) < \infty$ then $\mu(B\setminus A) = \mu(B) - \mu(A)$,
		\item \textbf{(strong additivity)} $\mu(A \cup B) + \mu(A \cap B) = \mu(A) + \mu(B)$,
		\item \textbf{(finite subadditivity)} $\mu(A \cup B) \leq \mu(A) + \mu(B)$,
		\item \textbf{(continuity from below)} if $A_n \uparrow A$ then $\mu(A) = \mu(\bigcup_{n \in \N} A_n) = \sup_{n\in\N} \mu(A_n) = \lim_{n\to \infty} \mu(A_n)$,
		\item \textbf{(continuity from above)} if $\mu(A) < \infty, \forall A \in \sa$ and $A_n \downarrow A$ then $\mu(A) = \mu(\bigcap_{n \in \N} A_n) = \inf_{n\in\N} \mu(A_n) = \lim_{n\to \infty} \mu(A_n)$, and
		\item \textbf{(sigma subadditivity)} $\mu(\bigcup_{n \in \N} A_n) \leq \sum_{n\in \N} \mu(A_n)$.
	\end{enumerate}
\end{thm}

\begin{proof}$ $\newline
	\begin{enumerate}
		\item Let $(A_n)_{n\in\N} \subset \sa$ with $A_1 = A,\ A_2 = B$ and $A_i = \emptyset$ for $i > 2$. It is clear that $A_n$ are disjoint since $A\cap B = \emptyset$ and $A_n \cap \emptyset = \emptyset,\ \forall n \in \N$.
		\item Write $B = (B\setminus A) \cupdot A$. Then, because of $\sigma$-aditivity we have
		\begin{align*}
		\mu(B) = \mu(B\setminus A) + \mu(A) \geq \mu(A) \text{ since } \mu(B\setminus A) \geq 0
		\end{align*} 
		\item As previously write $\mu(B) = \mu(B\setminus A) + \mu(A)$. Because $\mu(A) < \infty$ we can subtract it on both sides to get $\mu(B) -\mu(A) = \mu(B\setminus A)$.
		\item Write $A \cup B = A \setminus B \cupdot B \setminus A \cupdot A \cap B$ hence $\mu(A \cup B) = \mu(A \setminus B) + \mu(B \setminus A) + \mu(A \cap B)$. Add $\mu(A \cap B)$ on both sides and group terms:
		\begin{align*}
		\mu(A \cup B) + \mu(A \cap B) = \underbrace{\mu(A \setminus B) + \mu(A \cap B)}_{\mu(A)} + \underbrace{\mu(B \setminus A) + \mu(A \cap B)}_{\mu(B)}
		\end{align*}
		\item $\mu(A\cap B) \leq \mu(A \cap B)+ \mu(A \cup B) = \mu(A) + \mu(B)$
		\item Define the sequence $(B_n)_{n\in\N} \subset \sa$ by $B_1 = A_1$, $B_n = A_n\setminus A_{n-1}$ for $n > 1$. It is clear that $B_n$ is pairwise disjoint and that $\bigcupdot_{n \in \N} B_n = \bigcup_{n \in \N} A_n = A$. Hence
		\begin{align*}
		\mu(A) &= \mu(\bigcup_{n \in \N} A_n) = \mu(\bigcupdot_{n \in \N} B_n) = \sum_{n\in \N} \mu(B_n) = \lim_{m \to \infty} \sum_{n = 1}^m\mu(B_n) \\
		&= \lim_{m \to \infty} \mu (\bigcupdot_{n = 1}^m B_n) = \lim_{m \to \infty} \mu (A_m) = \sup_{n\in\N} \mu(A_n)
		\end{align*}
		since $\mu(A_n)$ is an increasing sequence. The introduction of the limits in the previous chain of equalities has to be done carefully, as we are building on the definition of limits for sequences of numbers. The equality between the first and the second lines comes from $\sigma$-additivity.
		\item Let $D_n = A_1 \setminus A_n,\ \forall n \in \N$. Then $D_N$ is an increasing sequence with
		\begin{multline*}
		\bigcup_{n\in\N} D_n = \bigcup_{n \in \N} (A_1\setminus A_n) = \bigcup_{n \in \N} (A_1\cap A_n^\complement) = A_1 \cap \bigcup_{n \in \N} A_n^c \\
		= B_1 \cap \left(\bigcap_{n \in \N} A_n\right)^\complement = A \setminus \bigcap_{n\in\N} A_n
		\end{multline*}
		Thus,
		\begin{multline*}
		\mu(A \setminus \bigcap_{n\in\N} A_n) \overset{(3)}{=} \mu(A) - \mu(\bigcap_{n\in\N} A_n) = \mu(\bigcup_{n \in \N} D_n) \\
		\overset{(4)}{=} \lim_{n\to \infty}\mu(D_n) = \lim_{n\to \infty} \mu(A_1 \setminus A_n) \overset{(3)}{=} \lim_{n\to \infty} (\mu(A) - \mu(A_n)) = \mu(A) - \lim_{n\to \infty} \mu(A_n)
		\end{multline*}
		Substracting $\mu(A) <\infty$ from both sides we have
		\begin{align*}
		\mu(\bigcap_{n \in \N} A_n) = \lim_{n\to \infty} \mu(A_n)
		\end{align*}
		\item Let $(A_n)_{n\in\N}$ be any countable subcollection of $\sa$. Define
		\begin{align*}
		E_n = \bigcup_{m = 1}^n A_m \in \sa
		\end{align*}
		Then $E_n \uparrow \bigcup_{n\in\N} E_n = \bigcup_{n\in\N} A_n$. Thus, by (6) we have
		\begin{align*}
		\mu(\bigcup_{n\in\N} A_n) = \mu(\bigcup_{n\in\N} E_n) = \lim_{n\to \infty} \mu(E_n) = \lim_{n\to \infty} \mu\left(\bigcup_{m = 1}^n A_m\right) \overset{(5)}{\leq}\lim_{n\to \infty} \sum_{m=1}^{n}\mu(A_m)
		\end{align*}
	\end{enumerate}
\end{proof}


\section{Examples}

Throughout this section, let $(X, \sa)$ be a measurable space.

\begin{eg}[Dirac measure]
	\label{eg:dirac-measure}
	We define the \textbf{Dirac measure} for a given $x_0 \in X$ as follows:
	\begin{align}
	\delta_{x_0}(A) = \begin{cases}
	0 \tif x_0\not\in A \\
	1 \tif x_0 \in A
	\end{cases},\qquad \forall A \in \sa
	\end{align}
	Clearly $\delta_{x_0}$ is a measure since it satisfies the two properties. First, $\delta_{x_0}(\emptyset) = 0$ since $\forall x_0 \in X,\ x_0 \not\in \emptyset$. Second, for any pairwise disjoint collection of sets $(A_n)_{n\in\N} \subset \sa$ we have two possibilities:
	\begin{itemize}
		\item If $x_0 \not\in \bigcupdot_{n \in \N} A_n$ then clearly $x_0 \not\in A_n,\ \forall n \in N$ so
		\begin{align*}
		0 = \mu\left(\bigcupdot_{n \in \N} A_n\right) = \sum_{n\in \N} \mu(A_n) = 0
		\end{align*}
		\item Otherwise, if $x_0 \in \bigcupdot_{n \in \N} A_n$ then there must be only one $n_0 \in \N$ such that $x_0 \in A_{n_0}$ since $(A_n)$ is a pairwise disjoint collection. Thus,
		\begin{align*}
		1 = \mu\left(\bigcupdot_{n \in \N} A_n\right) = \sum_{n\in \N} \mu(A_n) = \mu(A_{n_0}) + \sum_{n \neq n_0} \mu(A_n) = 1 + 0
		\end{align*}
	\end{itemize}
\end{eg}

\begin{eg}[Counting measure]
	Let $X = \R$ and choose $\sa = \{A \subset \R \mid A\text{ is countable or } A^\complement \text{ is countable }\}$. We already saw on the Chapter 1 that $\sa$ is a \siga. Now define $\mu:\sa \to [0, \infty)$ as
	\begin{align}
		\mu(A) = \begin{cases}
		0 \tif \# A \leq \# \N \\
		1 \tif \# A^\complement \leq \# \N
		\end{cases}
	\end{align}
	We have that $\mu(\emptyset) = 0$ since $\#\emptyset$ is countable. As for $\sigma$-additivity we must recall from set theory that the union of countable sets is also countable, so:
	\begin{itemize}
		\item If $\#A_n \leq \#\N,\ \forall n \in \N$ then
		\begin{align*}
			0 = \mu(\bigcupdot_{n \in \N} A_n) = \sum_{n\in \N} \mu(A_n) = 0
		\end{align*} 
		\item If there exists an $n_0 \in \N$ such that $A_{n_0}^\complement$ is countable then
		\begin{align*}
			\left(\bigcupdot_{n \in \N} A_n\right)^\complement = \bigcap_{n \in \N} A_n^\complement \subseteq A_{n_0}^\complement
		\end{align*}
		and hence $\left(\bigcupdot_{n \in \N} A_n\right)^\complement$ is countable. Furthermore, since the collection is pairwise disjoint, $\forall n \in \N, n \neq n_0$ we have $A_n^\complement \subseteq A_{n_0}$ so $\#A_n^\complement,\ \forall n \neq n_0$. Thus
		\begin{align*}
			1 = \mu(\bigcupdot_{n \in \N} A_n) = \sum_{n\in\N} \mu(A_n) = \mu(A_{n_0}) + \sum_{n \neq n_0} \mu(A_n) = 1 + 0
		\end{align*}
	\end{itemize}
\end{eg}

\begin{eg}[Discrete probability measure]
	Let $(\Omega, \sa, \Pro)$ be a measure space where $\Omega = \{\omega_1, \omega_2, \dots \}$ is a countable set, $\sa = \powerset(\Omega)$ (which is of course a \siga) and let $(p_1, p_2, \dots)$ be a probability vector where $\sum_{n\in \N} p_n = 1$ (and $p_i$ is the probability of $\omega_i$). Define the measure $\Pro : \sa \to [0, \infty)$ where
	\begin{align*}
		\Pro(A) = \sum_{\omega_i \in A} p_i = \sum_{i=1}^\infty p_i \delta_{\omega_i}(A)
	\end{align*}
	
	Let's verify that $\Pro$ is a measure. First, $\Pro(\emptyset) = 0$ since $\emptyset$ cannot contain any $\omega_i$. As for $\sigma$-additivity, we have
	\begin{align*}
		\Pro\left(\bigcupdot_{n = 1}^\infty A_n\right) = \sum_{i=1}^\infty p_i \delta_{x_i}\left(\bigcupdot_{n = 1}^\infty A_n\right) = \sum_{i = 1}^\infty p_i \sum_{n = 1}^\infty \delta_{x_i}(A_n) \\
		= \sum_{n = 1}^\infty \left(\sum_{i = 1}^\infty p_i\delta_{x_i} A_n\right) = \sum_{n = 1}^\infty \Pro(A_n)
	\end{align*}
	Nota that here we can exchange the summations because all the terms are non-negative so the convergence problems (oscillating convergence, that is) are eliminated.
\end{eg}

\begin{eg}[Lebesgue measure]
	\label{eg:lebesgue-measure}
	For now we shall define this measure only on $n$-dimensional rectangles. The generalisation to arbitrary Borel sets will come in chapter 4.
	
	Consider the measure space $(\R^n, \borel(\R^n), \lambda^n)$ where $\R^n$ and $\borel(\R^n)$ are the usual suspects and $\lambda^n : \borel(\R^n) \to [0, \infty)$ is defined as:
	\begin{align}
		\lambda^n\left(\bigtimes_{i = 1}^n [a_i, b_i)\right) = \prod_{i=1}^{n} (b_i - a_i)
	\end{align}
	or, more informally, the hypervolume of the rectangle in question.
	
	We are not ready to verify that $\lambda^n$ is a measure over $\borel(\R^n)$ yet. In fact, we will need to wait until the end of Chapter 4, where, once we have the generalisation to arbitrary Borel sets, we shall prove that it is a measure.
	
	We are however, in a position to prove that $\lambda^n$ is $\sigma$-finite. Let $A_i = [-i, i)^n$ for all $n \in \N$. Clearly, $(A_i)_{n \in \N}$ is an exhausting sequence since $A_i \uparrow \R^n$. Also, $\lambda^n(A_i) = (2i)^n < \infty, \forall n \in \N$. Therefore $\lambda^n$ is $\sigma$-finite.
	
	Note that $\lambda^n$ is not finite, since $\lambda(R^n) \not< \infty$.
\end{eg}